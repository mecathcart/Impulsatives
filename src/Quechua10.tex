\chapter{Cusco Quechua}

\section{Introduction}

Cusco Quechua is one of seven Quechuan languages are spoken in Western South America, specifically in the countries of Ecuador, Bolivia, Peru, Columbia, Chile and Argentina. There are over 10 million speakers of these languages, the largest group residing in Peru, with Cusco Quechua being spoken by over a million speakers.  Cusco Quechua is part of the Quechua II grouping, on the Southern Quechua branch. Its default word order is SOV and it is an agglutinating language with overt case-marking on nouns and often extensive derivational and inflectional suffixation on verbs \citep*{Cusihuaman:2001}.
	

Cusco Quechua, like other Quechua languages, has a desiderative affix \citep*{Muysken:1988a} which appears to be an impulsative, shown below: 

\singlespace

\exg. Noqa-ta tusu-{\bf naya}-wa-n.\\
I-ACC dance-IMPU-1OM-3SG \\
`I feel like dancing.' 
\label{quechua}

\doublespace

In this construction, the suffix \emph{-naya} added to a verbal stem V gives the meaning `feel like/be in the mood to V'.  The goal of this chapter is to provide a syntax and semantics of impulsatives in Cusco Quechua.   In the introduction, I showed the impulsative affix in Cusco Quechua differed from volitional desiderative affixes in other languages.  In this chapter, I will show that  within Cusco Quechua, impulsatives differ from periphrastic desiderative constructions.  Then I begin to study the nature of the suffix {\it naya} itself.  With regards to event structure, the impulse head is an event-introducing predicate.  Furthermore, the impulse head introduces an experiencer argument and is responsible for assigning it case.  The impulse head selects as its complement a vP. 

  \ex. \doublebr{Impulse} = $\lambda$P$_{<e,vt>}$$\lambda$x.$\lambda$e.$\lambda$w.$\forall$w'[w' is compatible with what x has an impulse to do in e in w ] $\rightarrow$[$\exists$e' in w'.P(x)(e')] 

Semantically, the null impulse head will do several things. First, it will provide intensionality by quantifying over possible worlds. Secondly, it has an event argument. Finally, it introduces an experiencer argument
and links it with the agent of the internal predicate in the modal world. Syntactically, the impulse head will license and case-mark an experiencer argument and select for an aspectual projection.   The
structure for example $\ref{quechua}$ is shown in the tree below.


\ex. \Tree  [.ImpulseP {Ricardo-ACC}  [.ImpulseP' {naya}  [.vP PRO [.v' v  [.VP [.V tusu ] ] ] ] ] ] 

The rest of the  chapter is organized as follows.  In the second section I differentiate impulsatives from desideratives in Quechua.  In the third section I discuss the event structure of Cusco Quechua impulsatives and show that impulsatives are a predicate over events.  In the fourth section, I argue that impulsatives introduce an experiencer argument.  In section five, the possibilities for the complement of the impulse head are reviewed.  Impulsatives in Cusco Quechua select for a control structure headed by vP.  Finally section six concludes the chapter and provides a full derivation and analysis.



\section{Impulsatives vs. Volitional Desideratives}
\label{definition}

Cusco Quechua has a construction that is known in the literature as a \emph{desiderative}.  However as I demonstrated in the Introduction, there are two types of desideratives;  impulsatives and volitional desideratives..  In the following section, I demonstrate how this construction in Cusco Quechua differs from volitional desideratives in Cusco Quechua.   Cusco Quechua does not have a desiderative affix but rather a periphrastic desiderative much like English `want'.

\exg. Noqa tusu-y-ta muna-ni.\\
I.NOM dance-INF-ACC want-1SG\\
`I want to dance.' 
\label{munay1}

In example $\ref{munay1}$ differs from $\ref{quechua}$ is several ways.  First, the subject of $\ref{munay1}$ is volitional and is often translated as `want'.   As previously mentioned, impulsatives are always translated with non-volitional meanings such as `feel like' or `have an urge to'.   Furthermore, it is possible for the desire and impulse to be two separate things.  Consider the following examples.
     

\exg. Noqa-ta pu\~nu-naya-wa-n. \\
I-ACC sleep-IMPU-1OM-3SG\\
`I feel like sleeping.'
 \label{on purpose}
 
 
 
\exg. Noqa pu\~nuyta munani.\\
I.NOM sleep-INF-ACC want-1SG\\
`I want to sleep.' 
\label{munay}
 
\doublespace


Example $\ref{on purpose}$ is salient in a context where I am tired even though I may not want to sleep.  For instance, it may be New Year's Eve and I want to be awake at midnight but am very tired.  However, this sentence cannot be used when what I want is not what I am feeling.  The sentence cannot be used in a context where I have a busy day the next day and want to get a good night's rest but cannot fall asleep.   On the other hand, example $\ref{munay}$  cannot be used in the same contexts as $\ref{on purpose}$.  It infelicitous to say $\ref{munay}$  when one is tired on New Year's Eve but wants to stay awake.  On the other hand, it is natural to say this sentence when one cannot fall asleep but wants to be rested for the next day.  Whereas example $\ref{on purpose}$ refers to the uncontrollable urge to sleep despite ones's desires, $\ref{munay}$ refers to one's desire to sleep as opposed to one's ability to sleep. 







 
  A second difference between impulsatives and volitional desideratives is case of the subject. Subjects of volitional desideratives in Cusco Quechua are nominative as seen in example $\ref{munay}$.   Subjects of impulsatives in Cusco Quechua, on the other hand, are accusative, as seen in $\ref{on purpose}$. While nominative case is the case that characterizes subjects generally in Cusco Quechua,  accusative case is the case marker for experiencers in Cusco Quechua.  A certain class of experiencer verbs mark their subjects with accusative case in Cusco Quechua.  


\exg. Noqa tusu-ra-ni.\\
I-NOM dance--PST-1SG\\
`I danced.'
\label{nom}
  
\exg.  Noqa-ta chiri-wan. \\
I-ACC cold-1OM-3SG\\
`I am cold.'
\label{lex1}

\exg. Ricardu-ta rayqan. \\
Ricardu-ACC hungry-3OB\\
`Ricardo is hungry.'
  \label{lex2}
  
  
 In example $\ref{nom}$, the canonical agentive subject is marked with the null nominative case.  In contrast, in examples $\ref{lex1}$ and $\ref{lex2}$ the arguments {\it Noqa} and {\it Ricardu} are marked by the accusative case marker {\it ta}.  These subjects are experiencer arguments of the lexical experiencer predicates {\it chiriy} and {\it rayqan}.  
   
  Furthermore,  impulsatives and desideratives affect verbal agreement differently.  Subject of volitional desideratives trigger canonical subject agreement.  Whereas, the subject in impulsatives fails to agree with the verb.  
  
  \exg. Noqa tusu-ni. \\
  I-NOM dance-1SG \\
  `I dance.' 
  \label{agr}
  
  
  \exg. Noqa-ta *tusu-naya-ni/tusu-naya-wa-n. \\
  I.ACC dance-IMP-1SG/dance-IMP-1OM-3SG \\
  `I feel like dancing.'
\label{no agr}


Example $\ref{agr}$ has the same agreement as $\ref{munay}$ since both have first person subject. However, Example $\ref{agr}$ and $\ref{no agr}$ exhibit the contrast between a subject induces agreement as in $\ref{agr}$ and a subject that doesn't as in the impulsative example $\ref{no agr}$. The verb instead carries object marking agreeing with the experiencer, which typically agrees with the objects of transitive verbs.


\exg. Magda qan-ta moqa-sunki. \\
Magda you-ACC hit-2OM.3SG \\
`Magda hits you.'
\label{you}


\exg. Qan-ta tusu-naya-sunki.\\
you.ACC dance-IMP-2OM.3SG\\
`You feel like dancing.'
\label{you2}


In examples $\ref{you}$ and $\ref{you2}$ the predicates host second person object marking {\it sunki}. While example $\ref{you}$ is the canonical use of object marking in Cusco Quechua, example $\ref{you2}$ exhibits the pattern of impulsatives, wherein the experiencer argument triggers the object marking.   Although it is called object marking because objects always trigger this agreement, it can be triggered by any argument that is affected.

\exg. Noqa-q wasi-y watumu-wa-ra-nki. \\
I-GEN house-1SG.PX visit-1OM-PST-2SG\\
`You visited my house.'
\label{OM}

In example $\ref{OM}$, even though the object is {\it wasiy} `my house',  the verb still receives the first person object marking {\it wa} because I am affected by you visiting my house.  Therefore, it is not just objects that can trigger object marking, but rather arguments that are affected.  None of the other languages in this dissertation have object marking; this is a characteristic unique of Cusco Quechua impulsatives.  


Thus, volitional desideratives in Cusco Quechua differ from the constructions with the impulsative affix.  While both express desire and introduce intensionality, impulsatives lack volition while volitional desideratives denote a willing subject.   Moreover, syntactically they behave as opposites.  While volitional desideratives contain normal subject case marking and agreement, impulsatives do not.  Instead, subjects in impulsatives receive accusative case marking, the same case marking other experiencer arguments in Cusco Quechua receive.  Furthermore, instead of triggering subject agreement, they trigger object agreement, which is the agreement pattern that affected arguments in the language trigger. Consequently, not only does the impulsative affix in Cusco Quechua differ from volitional desiderative affixes in other languages, they also differ from the periphrastic desiderative in Cusco Quechua.  

 
 
 \section{Event Structure}
 
 In this section I discuss the event structure of Cusco Quechua impulsatives. Cusco Quechua impulsatives are intensional in that they introduce possible worlds.  Additionally,  I argue that impulsatives in Cusco Quechua are bi-eventive and thus event introducers.
 
 Cusco Quechua impulsatives are intensional. First, impulsatives in Cusco Quechua do not preserve the truth value when a coreffering term is substituted.  
 
 \singlespace
 \ex. \ag. Noqa-ta Alan Garc\'ia-ta watuku-naya-wa-ran. \\
 I-ACC Alan Garc\'ia-ACC visit-IMPU-1OM-PST.3SG\\
 `I feel like visiting Alan Garc\'ia.' \\
 \label{Clark Kent} 
 \bg. Noqa-ta presidenti-ta watuku-naya-wa-ran.\\
  I-ACC president-ACC visit-IMPU-1OM-PST.3SG\\
`I feel like visiting the president.' \\
 \label{Superman}
 
 
 \doublespace
 
 Example $\ref{Clark Kent}$ does not mean the same as $\ref{Superman}$.  It is possible that the person uttering $\ref{Clark Kent}$ does not know who the president of Peru is and therefore has no desire to see Mr. Alan Garc\'ia.  Since the two terms {\it Clark Kent} and {\it Superman} co-refer in the real world, the sentences are only different in the possible worlds created by the impulsative element. 
 
 
 
  Furthermore, intensional sentences can be true when they a non-existing term like {\it Superman}.  This contrasts with a non-intensional sentence which is necessarily false if Superman does not exist.

\singlespace

 \ex. \ag. Noqa-ta superman-ta watuku-naya-wa-ran.\\
  I-ACC superman-ACC visit-IMPU-1OM-PST.3SG\\
`I feel like visiting the Superman.' \\
 \label{Superman1}
 \bg. Noqa superman-ta watukurani.\\
 I-NOM Superman-ACC visit-PST-1SG\\
 `I visited Superman.' 
 \label{Superman2}
 
 \doublespace
 
 Example $\ref{Superman2}$ is false, because Superman doesn't exist.  However, even though Superman does not exist $\ref{Superman1}$ could still be true.  This is because there is a possible world wherein Superman does exist that has been created by the intensional context introduced by the impulse head {\it naya}.  
 
 
 Since impulsatives quantify over possible worlds, one might hypothesize that impulsatives are analogous to circumstantial modals. Circumstantial modals have a modal base that quantifies over worlds compatible with a certain set of facts in the evaluation world. In other words, worlds that are determined by a specific set of circumstances \citep*{Kratzer:1991}.  Similar constructions in Polish have been analyzed as circumstantial modals \citep*{Rivero:2010}.   \citet*{Hacquard:2006} argues that modals are not predicate of events but instead have an accessibility relation that relates a set of possible worlds to an event.  Consequently, modals do not generate presuppositions with adverbs that apply to predicate of events such as {\it again} \citep*{vonStechow:1996, Bale:2006}.  
 
\exg. Yapamanta Gilbertu llank'a-na-n \\
again Gilberto-NOM work-OBL-3SG\\
`Gilberto has to work again.'
\label{na}

\exg. Yapamanta Gilbertu llank'a-ma-n \\
again gilberto-NOM work-COND-3SG\\
`Gilberto would work again'
\label{man}

Examples $\ref{na}$ and $\ref{man}$ both obligatorily presuppose that Gilberto has worked before. However, they do not generate any presuppositions associated with the modals.  Example $\ref{na}$ cannot mean that Gilberto had to work yesterday but didn't and thus has to work again today\footnote{This presupposition may be available in the English translation for some speakers, possibly because it has an infinitival complement}.  Likewise, $\ref{man}$ cannot be used in a context where yesterday Gilberto would have worked (in the fields) if it was sunny, but it was rainy and today he would work again (if it is sunny).   In contrast, impulsatives are ambiguous when associated with the adverb {\it yapamanta} `again'.


\exg. Noqa-ta yapamanta llank'a-naya-wan.\\
I.ACC again work.IMPU.1OM\\
`I feel like working again.'
\label{quechua again}


 In example $\ref{quechua again}$ there are two possible presuppositions: one in which I had the feeling before, and the second where I worked (maybe begrudgingly) before.  This indicates that {\it naya} is more than a modal.  Modals do not yield additional presuppositions because they do not introduce an event that can be presupposed.  This suggests that the impulse head introduces its own event.
 
 
 
   However the ability to generate presuppositions with `again' does not indicate a bi-eventive structure. Some predicates undergo Event Identification \citep*{Kratzer:1996}.  Under event identification, two separate events become the same event.  When there is only one event, two conflicting time adverbs create a contradiction.  However, if the structure is bi-eventive, each adverb can modify each event separately \citep*{Murasic:2006}.  
   
   

\singlespace


\exg. *Qaynap'unchaw lloqsi-ra-ni khunan p'unchaw-paq\\
Yesterday go.out-PST-1SG now day-for\\
`Yesterday, I am going out today.' 
\label{adverbs2}


\exg. Qaynap'unchaw lloqsi-naya-wa-ra-n khunan p'unchaw-paq\\
Yesterday go.out-IMPU-1OM-PST-3SG now day-for\\
`Yesterday, I felt like going out today.'
\label{adverbs}





\doublespace

In example $\ref{adverbs}$, the adverb {\it Qaynap'unchaw} `yesterday' modifies the time of the impulse whereas the adverb {\it khunan p'unchaw} modifies the time of the going out.  Thus there is no conflict in time because there are two separate events.  In contrast, example $\ref{adverbs2}$ is ungrammatical because the two opposing time adverbs create a contradiction.  This indicates that the impulsatives are bi-eventive and do not undergo Event Identification like the voice projection in $\ref{adverbs2}$.  



In this section, I have discussed the event properties of impulsatives in Cusco Quechua.  First, I showed that impulsatives are intensional.  Next, I demonstrated that although they are intensional, they are not like modals but instead event-introducing predicates.  Finally, I showed that impulsatives are bi-eventive.





 

 \section{Argument Structure}
 
 In this section, I discuss the argument structure of impulsatives; namely, the nature of the experiencer argument introduced by the impulse head.  First, I review a previous analysis of Imbabura Quechua by \citet*{Hermon:1985}.  Next, I compare Imbabura Quechua and Cusco Quechua concluding that while both experiencer arguments behave like subjects the former is assigned structural case while the latter is assigned ``quirky" case.  Finally, I discuss the proximate reading available with weather predicates.
 
		
While little work has been done on impulsatives in Cusco Quechua, significant work has been done on impulsatives in Imbabura Quechua.   Impulsatives are discussed at length in \citet*{Hermon:1985}, but the focus of her work is the nature of the logical subject rather than the syntax and semantics of the impulse head.  \citet*{Hermon:1985} observed that in Imbabura Quechua, like Cusco Quechua, impulsatives have an argument that is marked accusative.
	
	\singlespace
	
	\exg. (\~Nuka-ta) aycha-ta miku-naya-wa-n-mi. \\
	me-ac meat-acc eat-desid-1-OM-pr-3val\\
	`I desire to eat meat.' \footnote{Hermon's glossing may be slightly different from the glossing I use.} \\
\citep*[Example 1]{Hermon:1985}	
\label{acc}	
	
	\doublespace
	
	In example $\ref{acc}$, the argument {\it \~nuka} is affixed with the accusative case marker {\it -ta}, despite it being understood as the subject of the sentence.  \citet*{Hermon:1985} investigated the nature of the argument and determined that it behaved like a subject.  Based on evidence from diagnostics such as wh-extraction, and control she concluded that the argument begins as an object and moves to subject position at LF.   Under her analysis, the argument receives structural accusative case as the object of the affix {\it naya}.   When the argument appears as the subject of a matrix predicate than it appears with nominative case rather than accusative case.   This is demonstrated by embedding the predicate under a raising predicate.  	
	
	\singlespace

 \exg. Kan-\o-ga pu\~nu-naya-y yarin-gi \\
You-NOM-TOP sleep-DES-INF seem-2-SG-PRES \\
`You seem to want to sleep' \\
\citep*[Example 83]{Hermon:1985}	
\label{raising0}

\doublespace

In example $\ref{raising0}$, the argument {\it Kan} does not receive accusative case but rather is marked null nominative case marking as subject of raising verb {\it yarin-gi}. Moreover, when embedded beneath a control predicate, the argument also receives nominative case.  

\singlespace

\exg. \~Nuka pu\~nu-naya-chi-ni.\\
I-nom sleep-desid-pers-pr1\\
`I desire to sleep.' \\
\citep*[Example 70]{Hermon:1985}	
\label{nayachi}


\doublespace

In example $\ref{nayachi}$, the subject does not receive accusative case.  This is due to the personalizing morpheme {\it chi} which assigns nominative case to its argument. In this case, {\it \~nuka} is the argument of {\it chi} rather than an argument of the impulse head {\it naya}.  Additionally,  \citet*{Hermon:1985} observes that the personalizing morpheme {\it chi} has a semantic effect on the subject.  Example $\ref{nayachi}$ denotes a wish to perform the action, rather than an overwhelming urge to sleep.  

Impulsatives in Imbabura Quechua differ from lexical experiencer predicates that do not allow for the arguments to be raised.

\singlespace

\exg. *Kan-ga yarja-y yari-ngi.\\
you-top hunger-inf seem-pr-2\\
`You seem to be hungry.'\\
\citet*[Example 83b]{Hermon:1985}	
\label{raising lex}



\doublespace

 In example $\ref{raising lex}$, the argument of the lexical experiencer {\it kan} `you' is raised to the subject position of the raising predicate{\it yari} and is ungrammatical.  
 
 
 Like Imbabura Quechua, Cusco Quechua experiencer arguments also behave like subjects in that they are targeted by raising predicates.  However, the experiencer argument in impulsatives in Cusco Quechua retains its case even after being raised.
 
 \exg. Noqa tusu-y-ta qallari-ni.\\
 I-NOM dance-INF-ACC begin-1SG\\
 `I begin to dance.'
 \label{begin}
 
 \exg. Noqa-ta chiri-wa-y-ta qallari-n. \\
I-ACC cold-1OM-INF-ACC begin-3SG\\
`I begin to get cold�
\label{begin1}

\exg. Noqa-ta tusu-naya-wa-y-ta qallari-n \\
I-ACC dance-IMP-1OM-INF-ACC begin-3SG \\
`I am beginning to feel like dancing.'  
\label{begin2}

 
Example $\ref{begin}$ is a raising predicate with a canonical verb that assigns nominative case and triggers agreement.  Example $\ref{begin1}$ shows that a lexical experiencer retains its accusative case despite being the subject of a raising predicate.  Similarly, in example $\ref{begin2}$, the argument {\it noqa} `I' must receive accusative case marking {\it -ta}.   Therefore, when an argument raises to subject of {\it qallariy} `begin' its case is preserved.  In Cusco Quechua, the experiencer argument of an impulsatives behaves like the argument of a lexical experiencer predicate.  This suggests that the impulse head in Cusco Quechua assigns inherent case like experiencer predicates do.  In contrast, in Imbabura Quechua, the impulse head assigns structural case to its argument.  

This differences raises the question as to why the languages would differ in this particular fashion.  One hypothesis is that the languages differ with regards to what is considered a subject.  Recall that in Imbabura Quechua,  lexical experiencer predicates were not permitted to be embedded under raising predicates.  If it were strictly a case of assignment of inherent case vs. structural, the argument should be allowed to raise.  However, the sentence is ungrammatical.  This may be due to the fact that the argument is not considered a subject in Imbabura Quechua because it is not in Spec of TP \citep*{Bobaljik:1996}.  Having been assigned inherent case, the argument does not need to move to the spec of TP to receive case and stays {\it in situ}.  Thus raising predicates cannot target this argument because it is not in Spec of TP, and therefore not a subject.  In Cusco Quechua on the other hand, both impulsatives and lexical experiencers can have their arguments raised by a raising predicate.  This suggests that both arguments are considered subjects in Cusco Quechua.  Cusco Quechua may consider subjects to be the highest NP rather than the NP in spec of TP.  

\subsection{The Proximate}  
 Because impulsatives assign inherent case to its argument, it seem reasonable to posit that impulsatives are responsible for the introduction of the experiencer argument.  However, {\it naya} can be hosted by a weather predicate \citep*{Cusihuaman:2001} that has no external argument at all.

\exg. Para-naya-n\\
rain.IMP.3SG\\
`It's about to rain.'
\label{rain1}

%\exg. Para-na-n\\
%rain.OBL.3SG\\
%`It has to rain.'

%\exg. Para-n-man.\\
%rain.3SG.COND\\
%`It would rain.'

 However, there are several reasons to believe that this instance of {\it naya} differs from that of the canonical impulsative use.  First, the meaning of {\it -naya} with a weather predicate is distinct from that of the meaning of {\it -naya} when affixed to non-weather predicates.  In example $\ref{rain1}$, {\it -naya-} contributes an proximate reading  \citep*{Heine:1994}.  Instead of meaning `It feels like raining' it means `It's about to rain.'  The proximate reading is not available when {\it -naya-} is affixed to a non-weather predicate as in example $\ref{eat}$.
\singlespace

\exg. Pay-ta mihu-naya-n.\\
s/he-ACC eat-IMP-3SG\\
`S/he feels like eating.' \\
$*$`S/he is about to eat.' \footnote{Object marking on third person is null.}\\
\label{eat}


\doublespace


In contrast to example $\ref{rain1}$, example $\ref{eat}$ does not have the proximate reading available.  It cannot mean `S'he is about to eat.'  It only has the truly impulsative reading `S/he feels like eating.'  Nevertheless, this reading is available in Lamas Kechwa \citep*{Sanchez:2003}.
\exg. Miku-naya-ni \\
eat-des-1p\\
`I want to/am about to eat'
\label{toad} 

Example $\ref{toad}$ is ambiguous between a desiderative and proximate reading.  It is important to note that in this language, the subject is triggering agreement.  This means that, syntactically, it  is no longer an impulsative. \citet*{Romaine:1999} discusses how verbs such as `want'  can get grammaticalized to receive proximate readings.  She describes the process as the volition being backgrounded when volition does not make sense.

  Although it appears that impulsatives can occur without an experiencer argument, I show that instead this is a similar but unrelated reading, the proximate, of the same morpheme.  Thus the impulse head introduces an experiencer argument.  However, while in Imbabura Quechua it was assigned structural case, in Cusco Quechua it receives inherent case instead.
  
  
  
  
  
  
  
  
  
  
%
%However, even volitional predicates can induce a proximate reading with weather predicates. The verb {\it munay} as shown in section $\ref{definition}$, clearly has a volitional meaning.  However when it embeds a weather predicate, it also receives the proximate reading.

%\exg.  Para-y-ta muna-sha-n.\\
%rain-INF-ACC want-PROG-3SG\\
%`It's about to rain.'
%\label{rain}









%Another difference between these two uses of {\it -naya-} is that while $\ref{eat}$ allows an argument,  weather predicates do not allow an argument as in $\ref{no rain}$.

%
%\exg. *Pay(-ta) para-naya-n\\
%S/he(-ACC) rain.IMP.3SG\\
%`It's about to rain.'
%\label{no rain}

%Some may take $\ref{no rain}$ as evidence that {\it -naya} does not introduce its own argument.   However, there are several reasons to believe that {\it -naya-} does introduce its own argument and assigns it case.  This is evidence that {\it -naya} is not a modal, since modals do not introduce their own arguments or have the ability to assign case  \citep*{Bhatt:1998,Wurmbrand:1999a}.  First, as mentioned before, when a verb hosts {\it -naya-}, its argument receives accusative case rather than the nominative that would ordinarily appear.  Second, when embedded under a raising predicate the argument still receives accusative case.  Inherent or `Quirky' case is unaffected by raising predicates. 

%
%\exg. noqa-ta chiri-wa-y-ta qallari-n. \\
%I-ACC cold-1OM-INF-ACC begin-3SG\\
%`I begin to get cold�
%\label{begin1}

%\exg. Noqa-ta tusu-naya-wa-y-ta qallari-n \\
%I-ACC dance-IMP-1OM-INF-ACC begin-3SG \\
%`I am beginning to feel like dancing.'  
%\label{begin}

%

%

%
%Example $\ref{begin1}$ shows that a lexical experiencer retains its accusative case despite being the subject of a raising predicate.  Similarly, in example $\ref{begin}$, the argument {\it noqa} `I' must receive accusative case marking {\it -ta}.   Therefore when an argument raises to subject of {\it qallariy} `begin' its case is preserved This effect can also be seen with other modals.

%\exg. Noqa-ta tusu-naya-wa-na-n\\
%I-ACC dance-IMP-1OM-OBL-3SG\\
%`I have to feel like dancing.�
%\label{nayana}

%\exg. Noqa-ta tusu-naya-wa-n-man. \\
%I-ACC dance-IMP-1OM-3SG-COND\\
%`I would feel like dancing.�
%\label{nayaman}

%
%In examples both $\ref{nayana}$ and $\ref{nayaman}$ the subject {\it noqata} receives accusative case.
% 
% 
 \section{Syntactic Structure}
 
 
 In this section, I discuss the syntactic structure and I posit a semantic denotation for impulsatives in Cusco Quechua. First, impulsatives in Cusco Quechua can attach to nouns.  I argue that in these cases there is a null `have' and therefore, {\it naya} always attaches to a verb.   This indicates that the impulse head takes a vP as a complement.  As previously established, the impulse head also introduces an experiencer argument.  These facts allow us to detail a semantic denotation and composition for impulsatives in Cusco Quechua.

 
 

\subsection{Attaching to Nouns}

In this section, I discuss what type of predicates {\it naya} attaches to.  One type of such predicate is certain nouns in Cusco Quechua.  However, I argue that in those instances, there is a null `have' verb much like English intensional complements\citep*{Larson:1997}. I propose that impulse head {\it naya} can only attach to verbs.  

One fact that differentiates {\it naya} from covert impulsatives is that it can attach to nouns.


\exg. Aq'a-naya-wa-n.\\
chicha-IMP-1OM-3SG\\
`I feel like having chicha'\footnote{Chicha is an alcoholic drink made of fermented corn.}
\label{chicha1}

\exg. Ricarduta warmi-naya-n.\\
Ricardo-ACC woman-IMP-3SG\\
`Ricardo desires a woman (sexually)'
\label{woman1}


Based on this, one  could postulate that {\it naya} is a verbalizing suffix.  However, there are reasons to believe that {\it naya} isn't directly attaching to nouns and acting as a verbalizer.  Instead, I argue that there is a null `have' in these instances.  First, is that this pattern is not fully productive, but limited to a restricted set of commonly desired items.

\exg. *Pizza-naya-wa-n.\\
pizza-IMP-1OM-3SG\\
`I feel like having pizza'
\label{pizza}

Example $\ref{pizza}$ is ungrammatical, because the noun {\it pizza} is outside of the class of nouns that  {\it naya} is allowed to attach to. Secondly, all nouns that can appear with {\it -naya} can also appear as an object of the verb {\it kan} `have'.

\exg. Noqaq aq'a-y kan \\
I-GEN chicha-VAL have-3SG\\
`I have chicha.'
\label{chicha}

\exg. Rodolfoq warmin kan \\
Rodolfo-GEN woman-VAL have-3SG\\
`Rodolfo has a woman (wife).'
\label{woman}


The nouns appearing as objects of the predicate {\it kan} in exmaples $\ref{chicha}$ and $\ref{woman}$ are the same nouns from $\ref{chicha1}$ and $\ref{woman1}$.   However, there a few syntactic differences, namely the subject must have genitive case as shown in the above examples and {\it kan} is fully productive.  


\exg. Noqaq pizza-y kan \\
I-GEN pizza-VAL have-3SG\\
`I have pizza.'
\label{pizza1}

The verb {\it kan} can take {\it pizza} as its object as in example $\ref{pizza1}$.  However, the verb {\it kan} cannot form an impulsative.

\exg.  *Noqa-ta pizza-ta ka-naya-wa-n.\\
I-ACC pizza-ACC have-IMP-!OM-3SG\\
`I feel like having pizza.'
\label{kanaya}


The verb {\it kan} is Cusco Quechua has many uses existentials and locatives such as the predicates in \citep*{Freeze:1992}.  In these types of verbs both argument are internal arguments and do not have any voice projection.  As I will later explain, the impulse head in Cusco Quechua selects for vP. 



It is because of these differences that I posit that the null `have' is not the same as an elided {\it kan}.  Instead it is a null predicate. Evidence for this is based upon  bi-eventivity and the occurrence with other functional heads and attitude verbs.
%First,  the null `have'   can be elided\citep*{Larson:1997}. 

%\ex. \ag. Aqa-naya-sunki-chu? \\
%chicha-IMP-2OM-Q\\
%`Do you want chica'
%\bg. Manan ati-ni-chu\\
%No, can-1SG-NEG\\
%`No, I can't (have)'
%\label{ellide}

%Example $\ref{ellide}$, is understood as `I can't have chicha.'  It does not mean `I cannot feel like chicha.'  In this context, the person desires the chicha, but crucially cannot have any.  The elided predicate is {\it kan} `have' rather than the impulsative {\it naya}.  Thus because the affixation of {\it naya} to nouns is limited to only certain nouns, and those nouns can appear with the verb {\it kan} `have' and because of the interpretation of ellipsis, I conclude that there is a null `have' when {\it naya} attaches to nouns.

First, impulsatives attached to nouns are bi-eventive.   This can be seen by the use of time conflicting adverbs.

\exg. Khunan p'unchaw noqa-ta aq�a-naya-wa-n mincha p'unchaw-paq. \\
now day I-ACC chicha.IMPU-1OM-3SG few day-FOR\\ 
`Today I feel like having chicha in a few days.'
\label{ferment}



Example $\ref{ferment}$ is very salient because chicha takes a few days to ferment.  Therefore, it is understood that today, I feel like making the chicha so that I could have the chicha in a few days.   This is unexpected if the only event introducer in the sentence is the impulse head.  However, if there is a null `have', it can introduce the second event and allow the bi-eventive interpretation.  

Finally, other suffixes can be added to these nouns and induce the meaning of a null `have', such as the causative.



\exg.  Noqa aq'a-chi-ni.  \\
 I-NOM chicha-CAUS-1SG \\
`I am making chicha.'
\label{caus chicha}

In example $\ref{caus chicha}$ the causative morpheme {\it chi} is attached directly to the noun. Without the null `have' this sentence it would not be clear what event I am causing to happen.  Furthermore, attitude predicates, such as the periphrastic desiderative discussed earlier can take object complements.

\exg. Noqa aq'a-ta  muna-ni.\\
i-NOM chicha-ACC want-1SG\\
`I want chicha.'

Thus both the causative and the periphrastic desiderative take object complements.  I conclude that impulsatives that attach to noun include a null `have'.  This null have selects only a few common nouns in Quechua.  Evidence of a null have comes from ellipsis, and bi-eventivity. 




\subsection{Attaching to Verbs}

  This section discusses what verbal or functional category the Impulse head selects for.  It does not select as high as TP, as tense always occurs outside of the impulsative morpheme.
 
 \singlespace
 \ex. \ag. *Noqa-ta pu\~nu-ra-naya-wa-n. \\
 I-ACC sleep-PST-IMPU-1OM-3SG\\
 `I feel like I slept.' \\
\bg. Noqa-ta pu\~nu-naya-wa-ra-n. \\
 I-ACC sleep-IMPU-1OM-PST-3SG\\
 `I felt like sleeping' \\
\label{tense}

\doublespace

In example $\ref{tense}$, the past tense morpheme {\it ra} must occur after the impulsative {\it naya}.  Furthermore, the past tense morpheme modifies the impulse rather than the internal predicate {\it pu\~nu} `sleep', adhering to the mirror principal \citep*{Baker:1985a}.


This indicates that the complement of the impulse head is smaller than a TP.  Impulsatives in Cusco Quechua also do not select for Mood.  Future and conditional morphemes must occur outside of the impulsative.

\singlespace
\ex. \ag.*Noqa-ta tusu-qa-naya-wan\\
I-ACC dance.FUT-IMP-1OM-3SG\\
`I feel like I will dance'\\
\label{nqa naya}
\bg. Noqa-ta tusu-naya-wa-n-qa\\
I-ACC dance-IMP-1OM-3SG-FUT\\
`I will feel like dancing' \\
\label{naya nqa}

\doublespace


In example $\ref{naya nqa}$, the third person future morpheme {\it nqa} occurs after the morpheme. {\it -naya}.  However, example $\ref{nqa naya}$ is ungrammatical when the affix {\it qa} precedes the impulsative morpheme.

\singlespace
\ex. \ag.*Noqa-ta tusu-man-naya-wa\\
I-ACC dance.COND-IMPU-1OM-3SG\\
`I  feel like I would dance'\\
\label{man naya}
\bg. Noqa-ta tusu-naya-wa-n-man\\
I-ACC dance-IMPU-1OM-3SG-COND\\
`I would feel like dancing' \\
\label{naya man}

\doublespace



In example $\ref{naya man}$, the third person future morpheme {\it nqa} occurs after the morpheme. {\it -naya}.  However, example $\ref{man naya}$ is ungrammatical when the affix {\it man} precedes the impulsative morpheme.This indicates that the impulsative morpheme selects for a complement smaller than MoodP.  Lastly, the impulse head does not select for AspectP.


\ex. \ag.  Noqa-ta tusu-naya-wa-sha-n\\
I-ACC dance.IMPU-1OM.-PROG-3SG\\
`I am feeling like dancing.'
\label{naya sha}
\bg. tusu-naya-sha-wa-n \\
dance-IMPU-PROG-1OM-3SG \\
`I am feeling like dancing.'
\label{naya sha1}
\bg. *Noqa-ta tusu-sha-naya-wan\\
I-ACC dance-PROG-IMPU-1OM-3SG\\
`I am feeling like dancing.'
\label{sha naya}

In examples $\ref{naya sha}$ and $\ref{naya sha1}$, the progressive morpheme {\it sha} occurs after the morpheme. {\it -naya}.  However, example $\ref{sha naya}$ is ungrammatical when the progressive affix {\it sha} precedes the impulsative morpheme. 


  Therefore, I propose that the syntactic structure of Cusco Quechua impulsatives is that of a control or restructuring structure.


\ex. \Tree  [.ImpulseP {Ricardo-ACC}  [.ImpulseP' {naya} [.vP PRO [.v' v  [.VP [.V tusu ] ] ] ] ] ]
\label{tree}



In the structure above $\ref{tree}$, the impulse head selects for an argument, assigns it case and an experiencer theta role.  In addition this argument controls the subject of the internal predicate.  The impulse head has the following denotation.

\ex. \doublebr{Impulse} = $\lambda$P$_{<e,vt>}$$\lambda$x.$\lambda$e.$\lambda$w.$\forall$w'[w' is compatible with what x has an impulse to do in e in w ] $\rightarrow$[$\exists$e' in w'.P(x)(e')] 
\label{denotation1}

\singlespace

\ex. \a. \doublebr{dance} = $\lambda$e. dance(e) \\
 \doublebr{v} = $\lambda$x$\lambda$e. Agt(e,x) \\
 Event Identification \\
 \b. \doublebr{v'} = $\lambda$x.$\lambda$e.dance(e) \& Agt(e,x) \\ 
 \doublebr{PRO} = $\lambda$x. g(x) \\
 Functional Application \\
 \b. \doublebr{vP} = $\lambda$x.$\lambda$e.dance(e) \& Agt(e,g(x)) \\ 
    \doublebr{Impulse} = $\lambda$P$_{<e,vt>}$$\lambda$x$\lambda$e.$\lambda$w.$\forall$w' [w' is compatible with what x has an impulse in to do e in w] $\rightarrow$ [$\exists$e' in w'.P(x)(e')]\\
  Function Application \\
\b. \doublebr{ImpulseP'} = $\lambda$x$\lambda$e.$\lambda$w.$\forall$w' [w' is compatible with what x has an impulse to do in e in w ] $\rightarrow$ [$\exists$e' in w'.dance(e') \& Agent(e',x))] \\
 \doublebr{Ricardo} = Ricardo \\
 Function Application \\
\b. \doublebr{ImpulseP} = $\lambda$e.$\lambda$w.$\forall$w' [w' is compatible with what Ricardo has an impulse to do in e in w ] $\rightarrow$ [$\exists$e' in w' .dance(e') \& Agent(e',Ricardo))]\\



\doublespace



 
 \section{Conclusion}
 
 The goal of this chapter was to describe the event and argument structure of impulsatives in Cusco Quechua.  This will serve as a basis to which the covert impulsatives can be compared.  With regards to event structure, impulsatives are intensional and quantify over possible worlds.  However unlike a circumstantial modal, impulsatives introduce their own event.  Furthermore, they are bi-eventive.   With regards to argument structure, the impulse head introduces its own experiencer argument.  However, in Cusco Quechua, the Impulse head also assigns it case, while Imbabura Quechua does not.  Finally, with regards to syntactic structure, the Impulse head selects for a vP.  These facts lead us to posit the denotation in $\ref{denotation1}$ for the Impulse head.  In the following chapters I will show that this denotation also accounts for impulsatives in Bulgarian, Albanian and Finnish.
 
 

 
 
 
 
 