\chapter{Conclusion}

\section{Summary}


In this dissertation, I discussed the syntax and semantics of impulsatives.  I argued that we need to distinguish between two types of desideratives, volitional and non-volitional.  The non-volitional desideratives behaved differently both syntactically and semantically from volitional desideratives.  The former, impulsatives, can be divided into two types, overt and covert impulsatives.  The existence of overt impulsatives motivates the positing of a null impulse head in the languages that lack a dedicated impulsative morpheme.  Additionally, these languages provide independent evidence for positing a null impulse head, such as bi-eventivity and selectional restrictions.  Furthermore, by positing a null impulse head in Bulgarian, Albanian and Finnish, I am able to provide a unified analysis for impulsatives cross-linguistically.



This investigation began with the observation that impulsatives shared semantic and morpho-syntactic properties.  I defined impulsatives as constructions that denoted a non-volitional desire with an oblique experiencer argument and a verb that does not agree with this argument.  However, as the investigation continued, impulsatives revealed deeper commonalities.  As a result of shared event and argument structure, I was able to provide a unified semantic denotation for the impulse head.  The impulse head introduces intensional or modal semantics.  In the modal world, the experiencer is the agent of the internal predicate. The experiencer argument is also introduced by the impulse head. In addition, it is responsible for assigning the oblique case to the experiencer argument.  In Bulgarian and Albanian, the impulse head assigns dative case, in Finnish it assigns partitive case and in Cusco Quechua, it assigns accusative.   The impulsative is also an event introducer.   In Bulgarian, Albanian Finnish and Cusco Quechua, this event is syntactically independent from the event introduced by the internal predicate.  


\section{The Nature of Impulsatives}


The fact that impulsatives cross-linguistically share semantic and morpho-syntactic properties raises an important question about the nature of impulsatives.  Is it necessary that these two properties always come together as a predesignated package?  In other words, are the non-canonical case and agreement pattern and expression of involuntary desire inextricably linked?  


 As discussed in the introduction, it is common for experiencer predicates to assign `quirky' case to their arguments \citep*{McCawley:1976a, Dixon:2001, Sigurdsson:2002}.   This can be linked to various properties such as stativity, affectedness and volition. Volition is exclusively recognized as the determining factor in non-canonical subject marking in the South Asian languages by  \citet*{Klaiman:1980, Bhatt:1993}.  Therefore, it would be reasonable to postulate that the semantic and morpho-syntactic properties in impulsatives come hand in hand.  Furthermore,  in the cases of covert impulsatives, if there were canonical subject case and agreement, then there would be no cues for the impulsative interpretation.  The impulsative would look like an ordinary non-active sentence and consequently make many more sentences ambiguous.   Therefore, not only do the semantic and morpho-syntactic properties coincide, they are indispensable in covert impulsatives.  
 
 
However, if the impulsative meaning is linked to the case and agreement configuration, then it should not be possible for a language to have a impulsative construction with a nominative subject.   Nevertheless, it is possible to achieve an impulsative meaning without non-canonical case and agreement.  For instance, English is a language that always assigns nominative case to its subjects.  Thus in English, in order to express an impulsative meaning, the predicate must take a nominative subject.

\ex.  I feel like dancing.

Because of cases like English, it is not desirable to inextricably link the impulsative meaning with non-canonical case and agreement. . If the impulsative meaning and the syntax were not linked, it would predict that it should be possible for a language to have a dedicated impulsative affix with a nominative subject.  Unfortunately, much of the literature on desideratives does not explicitly discuss whether the constructions express a volitional or involuntary desire.  Thus it is difficult to determine whether the prediction is borne out or not.  



In essence, there is a contradiction.  On the one hand, to allow for cases like English, impulsative meaning and syntax cannot be linked.  On the other hand, it seems there must be a link between the two to allow for languages with covert impulsatives.  The answer to this problem may lie in the properties of the languages themselves. Languages with covert impulsatives are languages which exhibit a variety of case marking as well as verbal agreement; these languages are ``pre-equipped" in a sense to express an impulsative meaning using a non-canonical case and agreement configuration. English, however, has a limited case marking system and exhibits comparatively less verbal agreement. Thus, the only way to express an impulsative meaning in English is through a canonical case and agreement configuration. On the other end of the spectrum, isolating languages such as Mandarin Chinese have no overt case and agreement marking at all. Like English, Mandarin does not have a system which allows for an impulsative meaning to be expressed through the syntax.  The general pattern suggests that, languages that assign oblique cases to experiencer arguments are candidates for monoclausal impulsatives.  At the same time, languages that  do not assign oblique case to experiencer arguments or do not have any overt case and agreement at all, will not have monoclausal impulsatives.  I conclude then that there is no need to posit any external requirement linking the syntax and the semantics; rather the link can be viewed as a natural consequence of the properties of the languages themselves.
 
 In sum, even though impulsatives frequently appear with similar semantic and morpho-syntactic properties, nothing explicitly forces these properties to always come together. While cross-linguistically, it is common for languages to assign non-canonical case marking to the subject of experiencer predicates, many languages like English or Mandarin Chinese treat experiencer predicates like canonical predicates in the language.  Consequently, the link between the impulsative meaning and its case and agreement patterns are not universal.  Rather, the link can be viewed as a result of the syntactic properties of the languages which have impulsatives.
 
 
\section{Other Possible Impulsatives}

While almost every language has a periphrastic way of expressing non-volitional desire, dedicated impulsative morphemes appear quite rare.  Many of the Quechua languages, including Imbabura, Huanca, Ancash and Cusco have an attested form \citep*{Hermon:1985}, however, as far as I know, there is only one other attested case, Tohono O'odham \citep*{Zepeda:1987}.  

\singlespace

Tohono O'odham
\exg. S-n0-bisc-{ \bf im}-c �at \\
s-object-sneeze-DES-CAUSE Aux \\
`I feel like sneezing' or `Something makes me feel like sneezing' \\
\citep*{Zepeda:1987}
\label{tohono}

\doublespace

In example $\ref{tohono}$ the desiderative morpheme in Tohono O'odham is {\it -im-}.  Interestingly, impulsatives in Tohono O'odham come accompanied by the causative morpheme {\it c} which makes it appear similar to Finnish impulsatives \citep*{Pylkkanen:1999}.  Although \citet*{Zepeda:1987} claims that these are truely causative, she does not provide any syntactic tests such as those I have provided with Finnish impulsatives.







\section{Further Implications}

When positing a new grammatical category it is necessary to consider what other possible interpretations may overlap with the category.  Semantic meaning across languages is rarely entirely uniform and semantic shifts are inevitable.  Therefore, it is logical to conclude that categories with similar meanings would appear with the same morpho-syntactic realizations.    As far as I can discern, there are two possible overlaps:  the proximate and necessity readings.

  As shown in Cusco Quechua, the impulsative morpheme can sometimes generate proximate readings with weather predicates.  This is unsurprising, given that volitional desideratives often extend to future interpretations.  If a non-volitional desire were to extend to a future reading it would be that of a non-volitional future.  Proximate readings are precisely that: they indicate the future without positing any volition on the part of the subject \citep*{Romaine:1999}.  
  
  The necessity reading, like the impulsative reading, is also intensional and non-volitional.  This can observed in Hindi \citep*{Bhatt:1998}.
  
  
\exg. mujhe pad-na he  \\
1sg.DAT study-NOM PR\\
`I need to study.'
\label{Hindi}


Example $\ref{Hindi}$ has an oblique argument and a verb that does not agree with the subject, thus, morpho-syntactically behaving like a covert impulsative.   In addition, native speakers report that salient contexts are those involving bodily functions as in $\ref{pee}$.\footnote{Data received from Karthik Durvasula and friends.}

\exg. mujhe bohot zoor se muut-na he \\
me.DAT a.lot force with pee-NOM PR \\ 
 `I really need to pee.'
\label{pee} 


When it comes to bodily functions, there is not much difference between an uncontrollable urge and necessity. Thus, in addition to both being non-volitional and intensional, the meanings overlap in some contexts.  Therefore, the category of impulsatives may extend to necessity constructions.

\section{Further Issues}

One major unresolved issue in this dissertation is the role of aspect,  particularly in Bulgarian, where impulsative readings could not be attained in the imperfect form.  \citet*{Hacquard:2006}'s dissertation is dedicated to the intersection of aspect and modality.  She argues that aspect can bind the event variable in the accessibility relation in modals.  In addition, she argues that the possible combinations of event binders and accessibility relation is limited by restrictions on event type.  A further look might reveal what is restricting the imperfective aspect in Bulgarian impulsatives.   I will leave this as future research.

A second unresolved issue is whether the impulsatives undergo restructuring.  It is quite possible that impulsatives start out as truly bi-clausal and undergoes a process uniting the two clauses to yield one clause.   While established diagnostics for restructuring in the Romance and Germanic languages exist \citep*{Burzio:1986, Kayne:1989c, Rizzi:1982, Wurmbrand:2001}, restructuring in the languages covered in this dissertation has been less studied.  Thus more work needs to be done to determine whether impulsatives are a restructured construction or not.

Lastly, a related issue is whether the impulse head is functional or lexical. Functional categories do not assign theta roles to arguments, are subject to rigid ordering and co-occurence restrictions and allow only one type of complementation.  Lexical categories on the other hand, establish thematic relationships with arguments, are not subject to syntactic ordering and show optionality in complementation \citep*{Wurmbrand:2001}.  Impulsatives behave like a  functional head in that they only have one type of complementation in each language, and have rigid ordering.  However, they establish a thematic relationship with their argument like a lexical category.  Additionally, they introduce their own event like a predicate rather than a modal.  Therefore, it is not clear whether impulsatives are lexical or functional.  



