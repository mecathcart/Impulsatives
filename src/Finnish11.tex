%\documentclass{article}
%\usepackage{covington}
%\usepackage{fullpage}
%%\usepackage{xyling}
%\usepackage{qtree}
%\usepackage{amsfonts}
%\usepackage{marvosym}
%\usepackage{tipa}
%\usepackage{hyperref}
%\usepackage{linguex}
%\usepackage{setspace}
% \usepackage{natbib}
% \bibpunct{(}{)}{;}{a}{,}{,}

%\newcommand{\doublebr}[1]{[\hspace{-.02in}[{\bf#1}]\hspace{-.02in}]}
%\newcommand{\doublebrexpand}[1]{$\left[\hspace{-.06in}\left[#1\hspace{-.5in}\right]\hspace{-.06in}\right]$}

%

%\author{}
%\title {Finnish Chapter}
%\begin{document}
%\pagenumbering{arabic}
%\maketitle

%\doublespace


\chapter{Finnish}

\section{Introduction}



Finnish is spoken by approximately 5 million speakers \citep*{Ethnologue:2005}.  It is a member of the Finnic language family and morphologically between fusional and agglutinating.  Nouns are inflected for number and case while verbs are inflected for number, person, and tense.   

In this chapter, I will discuss Finnish impulsatives.  The goal of this chapter is to provide a syntactic structure and semantic derivation for impulsatives in Finnish.  Like their Albanian and Bulgarian counterparts, a dedicated impulsative morpheme is absent from the construction.  In Finnish, impulsatives are composed of a partitive argument and an unergative verb with the causative morpheme.


\exg. Maija-a laula-tta-a.\\
Maija-PART sing-CAUS-3SG \\
`Maija feels like singing.'
\label{phenom}

In example $\ref{phenom}$, the argument {\it Maija} is marked with partitive case and the verb {\it laula} `sing' is marked with the causative morpheme {\it tta}.  The sentence has the unexpected intensional meaning that Maija feels like singing.  This is the impulsative reading.  The impulsative reading is intensional because Maija does not actually have to sing in order for this sentence to be true. Finnish Impulsatives introduce a modal world wherein the experiencer is also the external argument of the internal predicate.  

Finnish impulsatives share properties of impulsatives in Quechua, Albanian and Bulgarian.  I show that they can be definitionally classified as impulsatives. However, impulsatives in Finnish do not have a dedicated morpheme to contribute the impulsative meaning and intensionality.  Therefore, like Albanian and Bulgarian impulsatives, I conclude that there is a null impulsative element.  Previous analyses of impulsatives in Finnish have analyzed them as causative \citep*{Pylkkanen:1999, Nelson:2004}.  However, I show that impulsatives in Finnish do not involve a causative event.  Instead I argue that the morpheme {\it tta} is rather a reflection of unaccusative syntax.  Thus the null impulse head in Finnish, like Albanian and Bulgarian, selects for an unsaturated predicate.  Furthermore, like Albanian impulsatives, Finnish impulsatives ban unaccusatives from forming impulsatives.  More specifically, the Finnish impulse head selects for Process V' \citep*{Ramchand:2011}.  Nevertheless, there is one major difference between Finnish impulsatives and its Balkan counterparts.  Finnish impulsatives do not allow for the logical objects of the internal predicate.   In Albanian and Bulgarian impulsatives, logical objects of the internal predicate are allowed to receive nominative case; in Finnish, logical objects are blocked from receiving nominative case.  Despite these differences, the semantic denotation of the Finnish impulse head is the same as its cross-linguistic counterparts.  The denotation is repeated below.

\ex. \doublebr{Impulse} = $\lambda$P$_{<e,vt>}$$\lambda$x.$\lambda$e.$\lambda$w.$\forall$w'[w' is compatible with what x feels like in e in w] $\rightarrow$[$\exists$e' in w'.P(x)(e')] 

Semantically, the null impulse head will do several things. First, it will provide intensionality by quantifying over possible worlds. Secondly, it has an event argument. Finally, it introduces an experiencer argument
and links it with the agent of the internal predicate in the modal world. Syntactically, the impulse head will license and case-mark an experiencer argument and select for an unsaturated Voice projection. The
structure for example $\ref{phenom}$ is shown in the tree below.

 \ex. \Tree      [.IP NP$_{Experiencer}$ [.ImpulseP t$_{Experiencer}$ [.Impulse' Impulse\\-tta [.{Process V'} V  ] ] ] ] 

The chapter is organized as follows.  In the second section, I establish that Finnish impulsatives are not desideratives and that they lack a dedicated intensional element.   In section three, I investigate the role of the causative morphology in Finnish impulsatives. I review previous analyses of impulsatives in Finnish that focus on the fact that impulsatives carry morphology associated with causatives in Finnish.  Then, I demonstrate that contrary to both \citet*{Pylkkanen:1999} and \citet*{Nelson:2004}, impulsatives in Finnish have no causing event.  Then, I show that Finnish impulsatives are parallel to their Balkan counterparts by elaborating on \citet*{Nelson:2004}'s initial intuition and argue that {\it tta} is a reflection of unaccusative syntax.   In section four, I show that the null impulse head is not a modal, like its cross-linguistic counterparts. Moreover, Finnish impulsatives are like their Albanian counterparts in that unaccusatives cannot be used impulsatives. Thus, I extend the proposal for Albanian, and suggest the Finnish impulsatives also select for Process V'.  Finally, in section six I provide a complete analysis and derivation. 

\section{Preliminaries}

\subsection{Impulsatives vs. Volitional Desideratives}
\label{impulsative}

Impulsatives in Finnish differ from volitional desiderative predicates in Finnish.   Semantically impulsatives differ from volitional desideratives in that they are not volitional and are often translated as 'feel like'. 
Finnish impulsatives are very salient with verbs involving bodily functions, such as `sneezing' and `coughing'.

\exg. Minu-a aivastu-tta-a. \\
I-PART sneeze-CAUS-3SG\\
`I feel like sneezing.'
\label{sneezing}

\exg. Minu-a yski-tt\"a-\"a \\
I-PART cough-CAUS-3SG\\
`I feel like coughing.' 
\label{coughing}

Examples $\ref{sneezing}$ and $\ref{coughing}$ are impulsatives formed with verbs that are commonly associated with impulses rather than desires.  One generally does not desire to sneeze or to cough unless one has an uncomfortable bodily sensation.  In addition, impulse and desire can be two separate feelings.  

\exg. Maija-a nuku-tta-a.\\
Maija-PART sleep-CAUS-3SG.\\
`Maija feels like sleeping.'
\label{sleep}

Example $\ref{sleep}$, is felicitous in a context where its New Year's Eve and Maija wants to stay up until midnight, but is tired.  However, $\ref{sleep}$ cannot be used in the context where Maija is unable to fall asleep despite wanting to get a good night's rest for the busy day ahead of her.   For the latter context, the volitional desiderative predicate is more appropriate.  


\exg. Maija halua-a nukku-a. \\
Maija.NOM want-3SG sleep-INF\\
`Maija wants to sleep.'
\label{want sleep}

Example $\ref{want sleep}$ describes Maija's desire to sleep, whether or not her body is tired.  In contrast to $\ref{sleep}$, $\ref{want sleep}$ cannot be used in the New Year's Eve context.  Finally, impulsatives in  Finnish cannot occur with a  purpose adverb.

\exg.  *Maija-a nuku-tta-a tahallaan\\
Maija-PART sleep-CAUS-3SG on.purpose\\
`Maija feels like sleeping on purpose.�
\label{purpose}

In example $\ref{purpose}$, the sentence is ungrammatical because of the addition of the purpose adverb {\it tahallaan} `on purpose' \citep*{Pylkkanen:1999}.  In contrast, volitional desiderative predicates can appear with a purpose adverb.

\singlespace

\exg. Maija halua-a nukku-a tahallaan. \\
Maija.NOM want-3SG sleep-INF on.purpose \\
`Maija wants to sleep on purpose.'\\
\citep*[Ex 35]{Pylkkanen:1999}


\doublespace
Impulsatives are not volitional because of the following properties: a) they are used saliently with bodily functions b) can be different than one's volition c) cannot be used with purpose adverbs.  Because of these properties, I conclude that semantically impulsatives are not volitional desideratives.
 
 
 Syntactically, subjects in impulsatives carry experiencer case.  In Finnish, partitive case can be assigned to experiencer arguments.  
 
 \exg. Maija-a palel-s-i.\\
 Maija-PART cold-CAUS-.PST.3SG \\
 `Maija feels cold.'
\label{cold}

\exg. Maija-a h\"ave-tt-i. \\
maija-PART be.ashamed-CAUS-PST.3SG \\
`Maija felt ashamed.'
\label{ashamed}

In examples $\ref{cold}$ and $\ref{ashamed}$, the subject { \it Maija}, as the experiencer of cold or shame,  is marked with partitive case.  This contrasts with the volitional desiderative predicates in Finnish, as in $\ref{want sleep}$, where {\it Maija} is marked with nominative case.  Furthermore, verbs in impulsatives do not agree with their subject while volitional desiderative verbs do.

\exg. Jussia ja Maijaa laula-tta-a/*laula-tta-vat \\
Jussi-PART and Maija-PART sing-CAUS-3SG/sing-CAUS-3PL \\
`Jussi and Maija feel like singing.'
\label{fin imp}


\exg. Maija ja Jussi halua-vat/*halua-a laula-a. \\
Maija.NOM and Jussi.NOM want-3PL/want-SG sing-INF \\
`Maija and Jussi want to sing.'  
\label{fin des}

In example $\ref{fin imp}$, the verb must carry third person singular agreement despite the fact that the subject is plural.  In example $\ref{fin des}$, on the other hand, the verb {\it halua} `want' does carry third person plural agreement and is ungrammatical with singular agreement. Therefore, case and agreement are different for impulsatives than for desideratives both semantically and syntactically. 




\subsection{Morphological Make-up}


Like Albanian and Bulgarian impulsatives, Finnish impulsatives pose a problem for the mapping between syntax and semantics.  There is a discrepancy because Finnish impulsatives, like Albanian and Bulgarian, lack a dedicated morpheme to provide the intensionality.   Finnish impulsatives  have two obligatory morphological elements, namely partitive case and the causative morpheme. Neither of these elements is responsible for the introduction of the intensionality in impulsatives.  Furthermore, Finnish impulsatives, like Albanian impulsatives, are limited to unergative verbs.


%Finnish impulsatives are intensional. In intensional sentences, the presence of a non-existent place does not yield a false sentence\citep*{Murasic:2006}. 

%\exg.  Maija-a laulatti lintukodossa.\\
%Maija-PART sing-CAUS-3SG.PST lintukoto.INESS\\
%`Maija felt like singing in Lintukoto.'
%\label{intensional}

%Lintukoto is a mythical place in Finnish mythology.  Example $\ref{intensional}$ is still true even though Lintukoto is a mythical place.  


Partitive case in Finnish is used on both themes and experiencers.  It is a structural case for objects \citep*{Vainikka:1989, Vainikka:1996}. 


\exg. H\"an luk-i kirja-a.\\
he read-PST.3SG book-PART\\
`He was reading a/the book.'
\label{object}


In example $\ref{object}$, the object {\it kirjaa} `book' receives partitive case.  Partitive case is also used for experiencer arguments as shown in section $\ref{impulsative}$, examples are repeated below.


 \exg. Maija-a palel-s-i.\\
 Maija-PART cold-CAUS-3SG.PST \\
 `Maija feels cold'
\label{cold1}

\exg. Maija-a h\"ave-tt-i. \\
maija-PART be.ashamed-CAUS-PST.3SG \\
`Maija felt ashamed.'
\label{ashamed1}


In examples $\ref{cold1}$ and $\ref{ashamed1}$, the subject { \it Maija}, as the experiencer of cold or shame,  is marked with partitive case.   In neither of the instances where partitive case is used is there intensionality.  Partitive case on its own does not introduce intensionality into the semantics.


Experiencer case in Finnish is typically not nominative; the type of case used for experiencers, however, is not uniform.  In addition to partitive, Finnish also uses the elative and adessive as experiencer cases. Notably, these cases are inherent or `quirky' cases rather than structural case.

\exg. Minu-sta tuntu-u ett\"a nyt alka-a sata-a. \\
I-ELA feel-3SG that now begin-3SG rain-INF \\
`I feel that it is  beginning to rain now.'

\exg. Minu-lla on n\"alk\"a.\\
I-ADE 3SG hunger\\
`I'm hungry.'

Partitive case appears on experiencers when they occur with the morpheme {\it -tta}.  The causative morpheme {\it -tta}  has several uses. In addition to impulsatives, it is used for causatives and on many psych predicates.  Canonical causatives are formed with an unergative verb and a nominative and a partitive argument.



\exg. Jussi laula-tt-i Maija-a.  \\
Jussi-NOM sing-CAUS-3SG.PST Maija-PART   \\
`Jussi made Maija sing.'
\label{causative1}

In example $\ref{causative1}$,  the causative morpheme {\it -tta} is affixed to the verb {\it laula} `sing', the external argument of {\it laulaa} `sing' receives partitive case and a nominative argument {\it Jussi} is introduced as the causer of the event.  Despite appearing very similar to impulsatives, causatives have no intensionality.  In order for example $\ref{causative1}$ to be true, Maija had to have actually sung (and not in some possible world). Additionally, the causative morpheme may appear on psych predicates.

\singlespace

\exg. Minu-a sure-tt-i.\\
I-PART grieve-CAUS-PST.3SG.\\
`I felt grief./saddened.'\\
\citep*[ex 11]{Nelson:2004} \\
\label{grief}

\exg. Pekka-a raivo-stu-tta-a. \\
Pekka-PART fury-INCH-CAUS-3SG \\
'Pekka feels infuriated.' \\
\citep*[ex 12]{Nelson:2004} \\
\label{infuriated}

\doublespace

In examples $\ref{grief}$ and $\ref{infuriated}$ the causative morpheme {\it tta} is affixed to psychological states such as {\it suru} `grief' and {\it raivo} `fury' and an experiencer argument is introduced with partitive case.   Even though this usage is not obviously causative, Finnish literature still calls this morpheme causative.  I will adopt this nomenclature, even for non-causative uses.  However, despite these two very different uses of the causative morpheme, neither introduce intensionality into the semantics.  A diagnostic for intensionality is the truth condition when there is a non-existent term in the sentence.  For example, the following sentence must be false \citep*{Larson:2002}.

\ex. \#I saw a unicorn.
\label{unicorn}

Example $\ref{unicorn}$ must be false because it is common knowledge that unicorns do not exist.  In contrast when a sentence is intensional, the sentence is not necessarily false.

\ex. I want to see a unicorn.
\label{unicorn1}


Irrespective of whether unicorns exist, example $\ref{unicorn1}$ can still be true.  Therefore, the occurrence of a non-existent term can diagnose an intensional sentence.  The morpheme {\it tta} does not introduce intensionality to its sentences.


\exg. \#Jussi    laulatti    Maijaa Valhallassa.\\
Jussi-NOM sing-CAUS-3SG.PST Maija-PART Valhalla.INESS  \\
`Jussi made Maija sing in Valhalla.'
\label{caus int}

\exg. \#Minu-a sure-tt-i Valhallassa.\\
I-PART grieve-CAUS-PST.3SG Valhalla.INESS\\
`I felt grief./saddened in Valhalla.'
\label{psych int}


Examples $\ref{caus int}$ and $\ref{psych int}$ are necessarily false because Valhalla is a fictional place in Scandanavian folklore.  Therefore, these sentences are not intensional. Using this test, it can be shown that the {\it -tta} morpheme does not add intensionality in either canonical causatives or psychological causatives.
 

Finally,  Finnish impulsatives are restricted to unergative verbs such as {\it laulaa} `'sing' in  $\ref{phenom}$, {\it aivastaa} `sneeze' $\ref{sneezing}$  and { \it nukkua} `sleep' $\ref{sleep}$. None of these verbs are intensional. Aditionally, the impulsative reading is unavailable with unaccusative verbs.\footnote{These verbs can be used with the causative morpheme to elicit idiosyncratic meanings, such as { \it kuolettaa} means `to cancel' and {\it saavuttaa} means `to reach'.  }

\exg. *Maija-a kuole-tta-a.\\
Maija-PART die-CAUS-3SG \\
`Maija feels like dying.' 
\label{die}

\exg. *Maija-a saavu-tta-a. \\
Maija-PART arrive-CAUS-3SG \\
`Maija feels like arriving.' 
\label{arrive}


\exg. *Maija-a pudo-tta-a. \\
Maija.PART fall-CAUS-3SG \\
`Maija feels like falling.' 
\label{fall}


Examples $\ref{die}$- $\ref{fall}$ are ungrammatical with unaccusative verb {\it kuolla} `die', {\it saapua} `arrive', and {\it pudota} `fall' respectively, despite having a partitive argument and the causative morpheme {\it -tta}.  Furthermore, while Finnish impulsatives can appear with transitive verbs, their objects are not allowed.


\exg. *Maija-a laula-tta-a laulu. \\
Maija-PART sing-CAUS-3SG song. \\
`Maija feels like singing a song.' \footnote{Possibly grammatical with causative reading `A song makes Maija feel like singing.'}
\label{song}


\exg. H\"an-t\"a kirjoitu-tt-i (*kirje).\\
s/he-PART write-CAUS-PST.3SG (letter)\\
`She felt like writing (letter).'
\label{letter}

Examples $\ref{song}$ and $\ref{letter}$ are ungrammatical when they appear with the objects {\it laulu} `song' and {\it kirjee} `letter'.  Finnish impulsatives have some restriction on allowing the internal predicates to occur with their logical objects.  

Thus, while Finnish impulsatives do require certain morphological and syntactic components, none of these components individually contribute intensionality to the impulsative.   Since there is no apparent source for the intensionality of impulsatives in Finnish, I conclude that like Albanian and Bulgarian, there must be a null element.

 

\section{The Role of Causative Morphology}
\subsection{Impulsatives as Causatives}



\citet*{Pylkkanen:1999} and \citet*{Pylkkanen:1999b} develop an analysis of Finnish impulsatives, which she calls causative desideratives.  Pylkk\"anen's analysis centers around motivating the separation between the causative head and the external argument that it is assumed to introduce.  She analyzes the impulsatives as an instance of a causative that lacks an external argument.  \citet*{Pylkkanen:1999}'s account assumes a null modal element to derive the impulsative meaning.  She analyzes {\it-tta} as the homophonous causative morpheme and claims that the causative morphology introduces a causing event.   This is supported by the fact that causing event can be sluiced. 

\singlespace

\exg. Minu-a laula-tta-a mutt-en tied\"a mik\"a. \\
I-PAR sing-CAUSE-3SG but-not.1SG know what.NOM \\
`Something makes me feel like singing but I don�t know what. (makes me feel like singing)'\footnote{The Finnish speakers I have consulted have given mixed judgements on this data point.}
\citep*[Example 21a]{Pylkkanen:1999b}
\label{sluice}

\doublespace

In example $\ref{sluice}$, the sluiced interpretation is `what makes me feel like singing' as opposed to `what to sing'.  This indicates there is an underlying causing event in Finnish impulsatives.  Under Pykk\"anen's analysis,  impulsatives are causatives that have no external argument and select for a null desiderative head.  She provides this partial structure.


 \ex. \Tree [.{$\lambda$e.[ $\exists$(e')CausedEvent(e') \&CAUSE(e,e')] } CAUSE\\{$\lambda$f$<$s,t$>$[$\exists$e'f(e') \& CAUSE(e,e')]} [.{$\lambda$e.[ CausedEvent(e)} [.{} {}  ] ] ]



Finally, Pylkk\"anen analyzes the partitive argument as an affected argument of CAUSE which is also thematically related to the caused event.  She claims that it receives partitive case as the object of CAUSE.  This is because partitive case is the objective case when the verb is atelic or stative, like with impulsatives.   

 \subsection{Psych Causatives}

  In addition to its canonical causative use, the affix  {\it -tta} is also used to derive psych predicates.  Some Finnish grammarians \citet*{Hakulinen:1979} and \citet*{Sulkala:1992} have analyzed {\it -tta} as having two homophonous morphological processes.  \citet*{Nelson:2004}, however, provides a unified analysis of all uses of the causative morpheme {\it -tta}. She concentrates on psychological predicates that are affixed with the causative {\it -tta}, dubbed Psych Causatives.  Additionally, she differentiates what she calls stative psych causatives and inchoative psych causatives.  She claims that despite their surface similarities, causative morphology affixing to stative and inchoative bases results in different syntactic structures. 
  
  
  
  When causative morphology is attached to inchoative psych bases, the nominative argument receives objective (partitive or accusative) case instead.

\singlespace
\exg. Koira raivo-stu-i (minu-Ile). \\
dog.NOM fury-INCH-PST.3SG me-ALL \\
`The dog became infuriated (because of me).' \\
   \citep*[Ex 14]{Nelson:2004}
\label{inchoative base}


\exg.Asia raivo-stu-tt-i minu-a / minu-t. \\
matter.NOM fury-INCH-CAUS-PST.3S me-PART / me-ACC \\
`The matter was infuriating / infuriated me.' \\
  \citep*[Ex 15]{Nelson:2004}
 \label{inchoative PC}
 
 \doublespace

   Example $\ref{inchoative base}$ is the inchoative base and example $\ref{inchoative PC}$ is an inchoative psych causative, the base with causative morphology.  The experiencer argument in the base is nominative, however it can be either partitive or accusative in the causative.  In addition, what appears to be the theme appears as a nominative argument in the inchoative psych causative example $\ref{inchoative PC}$.  Nelson argues that in contrast to stative psych causatives, the nominative argument in inchoative psych causatives is projected above the experiencer argument.   This means that the nominative argument is introduced by the causative head above the internal inchoative psych predicate.  
  
 Superficially, stative psych predicates look like inchoative psych causatives.   When causative morphology affixes to a stative base, the nominative argument receives partitive case.
    \singlespace
 
 \exg. Mina sur-i-n h\"an-t\"a. \\
I-NOM grieve-PST-lSG him/her-PART \\
`I grieved for him/her.' \\
  \citep*[Ex  9]{Nelson:2004}
\label{stative base}
 
 
 \exg. Minu-a sure-tt-i koira-ni kuolema. \\
I-PART grieve-CAUS-PST.3SG dog-lsPx death.NOM \\
`I felt grief about my dog's death.' \\
 \citep*[Ex 11]{Nelson:2004}
 \label{stative PC}
 
 \doublespace
 Example $\ref{stative base}$ is the stative base and example $\ref{stative PC}$ is the base augmented with the causative morphology producing a stative psych causative. The nominative argument in $\ref{stative base}$ appears with partitive case in example $\ref{stative PC}$.  In addition, the theme of {\it surra} `grieve' appears with nominative case in the psych causative $\ref{stative PC}$ instead of partitive as in $\ref{stative base}$.   Nelson argues that in stative psych causatives the partitive experiencer argument is projected above the nominative theme. This means that the nominative argument is base generated as the theme of the internal predicate and moves to subject position to receive case; thus stative psych causatives behave like an unaccusative predicate. Nelson supports this with data from binding, passivization case, and agreement.  


  In both inchoative psych causatives and stative psych causatives, the causative morphology takes the experiencer argument and assigns it objective case rather than the nominative they receive in the base.  Under Nelson's analysis, causative morphology suppresses the external argument of the internal predicate.  Then it can add an external argument as with canonical causatives and inchoative causatives.   However, it can also internalize the input external argument to derive a predicate with no external argument, as with impulsatives and stative psych predicates.  She further asserts that if the partitive argument is in surface subject position, the event is interpreted as being internally caused.   This is how she unifies all the uses of the causative morphology in Finnish.
   
   \subsection{Impulsatives are Not Causatives}

In this section, I will demonstrate that there is no causative event in Finnish impulsatives. Both Pylkkanen's and Nelson's analyses portray impulsatives as causatives without an external causing argument.  Nelson's analysis takes it a step further and says that it is internally caused and therefore no causing event is interpreted.  However, if a causative head is projected with a causative denotation, it appears incongruous to assert that no causative event be interpreted.  


 Although Nelson never addresses the source of the intensionality in impulsatives, Pylkk\"anen suggests there is a null desiderative element.   While Pylkk\"anen's analysis explicitly states that impulsatives do not have an external `causer' argument, it is unclear what exactly would prevent one from projecting. If CAUSE can select for the null impulsative, then it should be able to notwithstanding an external argument.  Therefore, Pylkk\"anen's analysis predicts that all morphologically causative sentences would be ambiguous between the causative and the impulsative meanings.

\singlespace

\exg. Jussi laula-tt-i Maija-a. \\ 
Jussi.NOM sing-CAUS-PST.3SG Maija-PART  \\
`Jussi made Maija sing.' \\
$*$`Jussi caused Maija to feel like singing.'
\label{causative}

\doublespace

Example $\ref{causative}$, only has the direct causative meaning.  It is only true if Maija sang.  It cannot mean that Jussi caused Maija to feel like singing. Thus this indicates that the CAUSE head cannot always select for the impulse head.


Under Pylkk\"anen's analysis, the structure of impulsatives and causatives would be the same except for the lack an external argument and the addition of a null head that provides the intensionality.  Causatives can have resultative structures as in $\ref{resultative}$. Thus a causative analysis would predict that impulsatives would also be able to form resultatives.  However, this prediction is not borne out as  impulsatives cannot have resultative structures.


\exg. Jussi laula-tt-i Maija-a py\"orryksiin. \\
Jussi.NOM sing-CAUS-3SG.PST Maija.PART dizzy.\\ 
`Jussi made Maija sing (until she's) dizzy.'
\label{resultative}

\exg. *Maija-a laula-tt-i py\"orryksiin.\\
Maija-PART sing-CAUS.3SG.PST dizzy.\\
`Maija is caused to feel like singing dizzy.'
\label{imp res}

In example $\ref{resultative}$, Jussi makes Maija sing and this results in her being dizzy.  However, example $\ref{imp res}$ is completely ungrammatical.  It cannot have a resultative interpretation or structure.  This fact is unexplained if there indeed is an  underlying causing event.


 Lastly, if a causative event is present, it should be able to be negated.  While, this is the case with canonical causatives, impulsatives do not behave in a parallel fashion.

\singlespace

\exg. Jussi e-i laula-tta-nut Maija-a mutta Maija laulo-i muuten vain.\\
Jussi.NOM not-3SG.PST sing-CAUS-INF Maija-PART but Maija.NOM sing-3SG.PST anyway.\\
`Jussi didn�t make Maija sing but she sang anyway.� \\
\label{negation}

\exg.Maija-a e-i laula-tta-nut mutta....\# Maija halus-i laula-a muuten vain. \\
Maija-PART not-3SG.PST sing-CAUS-INF but Maija.NOM want-3SG.PST sing-INF anyway\\
$*$`Nothing made Maija feel like singing, but she wanted to sing anyway.�\\
`Maija didn't feel like singing, but she wanted to sing anyway.'
\label{imp neg}

\doublespace
In example $\ref{negation}$, the negation targets the causing event.  Thus, even though Maija was not caused to sing, she sang. Example $\ref{imp neg}$ on the other hand means that `Maija didn't feel like singing'  as opposed to `Nothing caused Maija to feel like singing.'  The negation targets the impulse to sing rather than an event that causes an impulse, resulting in a contradiction with the continuation `Maija wants to sing anyways.'\footnote{The continuation is not quite sufficient because it is a desiderative as explain in section 2, and would not be a contradiction in contexts described in that section.}  This is unexplained by Pylk\"anen's analysis because the negation should target the causative event as it does in $\ref{negation}$.


Another problem with Pykk\"anen's analysis is the mechanism in which case is assigned to the experiencer argument.  Under Pylkk\"anen's analysis, the experiencer argument receives structural partitive case.  If this were correct, then when it undergoes raising the case should become nominative.  However, this predictions is not borne out.

\exg. Maija-a n\"aytt\"a-\"a laula-tta-va-n.\\
Maija-PART seem-3SG sing-CAUS-INF-ACC\\
`Maija seems to feel like singing.'
\label{seem}

In example $\ref{seem}$ the argument {\it Maija} receives partitive case rather than nominative.  This indicates that the case marking is inherent rather than structural. 

I conclude based upon the interpretation of canonical causatives, resultatives,  and negation that there is no causative event in impulsatives in Finnish.  Furthermore, the experiencer argument receives inherent rather than structural case as object of CAUSE.  In these two regards, Pylk\"anen's analysis fails to capture the syntactic structure of impulsatives in Finnish. In addition, Nelson fails to unite all the possible uses of Finnish causative morphology.




   
  \section{{\it -tta} Attaches to Unaccusative Structures}
  
  In this section, I present a new way to analyze the morpheme {\it -tta}.  I have shown, contrary to Pylkk\"anen and Nelson, that not all uses of Finnish causative morphology is truly causative; more specifically,  impulsatives are not causatives. Consequently, the unified analysis of the uses of {\it -tta} is undermined.  However, I argue that a unified analysis of {\it -tta} can be attained if we abandon the notion that {\it -tta} is causative.   I provide an alternative analysis that analyzes {\it -tta} as the reflection of unaccusative syntax in Finnish, similar to non-active morphology in Bulgarian and Albanian \citep*{Embick:2004}.  Both the causative and impulse heads are null, but select for unaccusative complements that trigger the insertion of {\it -tta}.

  
  
%  
%   There are two prominent patterns with all constructions involving {\it tta}.  First, is that constructions with {\it tta} are unaccusative.  Secondly, an experiencer argument is introduced.  Based on these two properties, I propose that {\it -tta} is a head in essence converts external arguments into experiencer arguments.  In turn, impulsatives and causatives are null heads that select for the {\it tta} head and use the experiencer argument for their denotations.
  
 
  
%  
%   In this section, I provide evidence that not only are impulsatives not causatives but neither are her internally caused stative psych causatives. 
%  
%  
%  
%   I provide an alternative analysis that analyzes the {\it tta} as the reflection of unaccusative syntax in Finnish, similar to non-active morphology in Bulgarian and Albanian \citep*{Embick:2004}.  Both the causative and impulsative  heads are null, but select for unaccusative complements that trigger the insertion of {\it -tta}.
  
  Thus far four different instances of {\it- tta} affixation have been identified: canonical causatives, stative psych causatives, inchoative psych causatives and impulsatives.  In each of these constructions, I argue that {\it -tta} is affixing to an unaccusative predicate,  a predicate that cannot assign objective\footnote{Generally unaccusative predicates are those that cannot assign accusative case.  However it is argued that Finnish verbs can assign either partitive or accusative case to its internal argument depending on factors such as telicity\citep*{Vainikka:1996, Kiparsky:2001}. } case. This is because either the verb is unaccusative already as with inchoative psych predicates, or the addition of the {\it -tta} morpheme renders the verb unable to assign objective case.  
  
  
   First, I will review psych causatives.  Inchoative bases are by definition unaccusative verbs.\footnote{I do not know any unaccusative diagnostics for Finnish intransitives, but the class of verbs Nelson identifies as inchoatives have the meaning `become X', similar to the class of verbs that generally participate in inchoative alternations in other languages.}   These are verbs that cannot assign case to their internal arguments.  The internal argument instead raises to spec IP to receive nominative case.  When the verb is augmented by the causative morpheme {\it -tta}, the internal argument then becomes the object of the causing event and receives objective case from the cause head. 
   
   
   
     In the case of stative psych causatives, Nelson argues that once the {\it -tta} is affixed the predicate, it can no longer assign objective case to its theme.  Consequently, the theme moves to spec of IP to receive nominative case.  Nelson's structure for stative psych causatives is shown below.
     
     
   \ex. \Tree [.IP Theme$_i$ [.vP $<$Eventx,y$>$ [.v' tta [.VP Experiencer [.V'  V t$_i$ ] ] ] ] ]
     
     
     
    
     
        In addition, canonical causatives and impulsatives cannot have thematic arguments with objective case.  
  
  \exg. *Maija-a laula-tta-a laulu-t.\\
Maija-PART sing-CAUS-3SG song-PL.ACC\\
`Maija feels like singing songs.�
\label{imp unacc}

\exg. *Jussi laula-tt-i Maija-a laulu-t\\
Jussi.NOM sing-CAUS-PST.3SG Maija-PART song-PL.ACC\\
`Jussi made Maija sing songs.'
\label{caus unacc}
  
  
  Examples $\ref{imp unacc}$ and $\ref{caus unacc}$ are ungrammatical because they occur with the object {\it laulu}.  In simple sentences, the verb {\it laula} licenses objective case to its object.  When it occurs in either a causative or impulsative construction, however, it loses its ability to license this case.  Therefore causatives and impulsatives in Finnish are unaccusative in that they do not allow their internal predicate to assign accusative case.

 Thus it appears that in all constructions, the complement of {\it -tta} cannot assign objective case, whether the predicate starts out as unaccusative like the inchoative causatives or becomes unaccusative as a result of the {\it -tta} affixation as in stative causatives, canonical causatives and impulsatives. 
 
 
 
  Based on this data,  I propose that {\it -tta} is not a causative morpheme at all. I propose an account in the Distributed Morphology framework \citep*{Halle:1993b}.  I  suggest that {\it- tta} is a reflection of unaccusative syntax, parallel to the non-active morphology in Bulgarian and Albanian \citep*{Embick:2004}. The unaccusative syntax is created by the selection of an unsaturated predicate. The major difference is that, while in Albanian and Bulgarian non-active morphology can be inserted directly under tense and aspect, Finnish only allows certain functional heads to select for unsaturated predicates, such as causative and impulse heads.  Under this analysis, the causative head would be a null functional head.  Thus when the null causative or impulse heads select for an unsaturated predicate, this triggers the insertion of {\it -tta}.  The trees are provided below.\footnote{I am not sure what type of functional head the subjectless stative psych causatives would involve, possibly a bleached out experiencer head as in \citet*{Brisson:1998}. } 
   


\ex.      \Tree [.IP NP$_{causer}$ [.CauseP t$_{causer}$ [.Cause' Cause\\-tta [.Cause' NP$_{causee}$ [.v' v VP ] ] ] ] ]
\label{caus tree}
      
 \ex. \Tree       [.ImpulseP NP$_{Experiencer}$ [.Impulse' Impulse\\-tta [.v' v VP ] ] ]
 \label{impulse tree0}
 
 In the structures in $\ref{caus tree}$ and $\ref{impulse tree0}$ the null heads CAUSE and IMPULSE select for either v'.  This blocks the external argument from projecting and thus v' remains unsaturated.  This syntactic structure triggers the insertion of {\it -tta}.
 
 
  However, unlike Albanian and Bulgarian, Finnish does not insert {\it -tta} in traditionally unaccusative structures such as passives.  This is because Finnish passives are not unaccusative at all, and are fully capable of assigning objective case.


\exg. H\"anet esitel-tiin Juka-lle.\\
He.ACC introduce-PASS Jukka-ALL\\
`He was introduced to Jukka.'
\label{passive}

In example $\ref{passive}$, even though the logical object appears preverbally, it still retains accusative case as the object of the verb {\it asiteltiin} `introduce'.  This demonstrates that the Finnish passive is not a reflection of unaccusative syntax because it can still assign accusative case.  

In this section I have provided a novel way to look at the instances of the morpheme {\it -tta}.  I demonstrated that in each of the cases shown to use the morpheme {\it -tta}, unaccusative syntax was involved.  Thus I suggested a DM account of {\it -tta} as a reflection of unaccusative syntax.  In addition, this unifies impulsatives in Finnish with impulsatives in Albanian and Bulgarian which also select for an unsaturated predicate. 

 
% While not all constructions involving {\it -tta} have a causative event, they all have unaccusative syntax (an inability to assign objective case) and an experiencer argument with partitive case.  Therefore, I propose that instead of representing causative morphology {\it -tta} is a semantically bleached out event property cf. \citep*{Brisson:1998}.  It introduces an experiencer argument, assigns it partitive case and selects for an unsaturated voice projection.  This would mean that the causative is actually a null morpheme as well.  Since both causatives and impulsatives involve experiencer arguments they select for {\it -tta} which supplies them with the needed argument.  When {\it tta} attaches to a stative psych predicate, the {\it tta} merely converts the external argument from a more agentive one to an experiencer. 

%

%In this section I have provided a novel way to look at the instances of the morpheme {\it tta}.  I demonstrated that in each of the cases shown to use the morpheme {\it tta} unaccusative syntax  and a partitive experiencer argument was involved.  Thus I suggested that {\it tta} is a semantically bleached out event property cf. (Brisson, 1998).  This unifies all the constructions involving {\it tta} in Finnish.  
% 
 
 \subsection{Partitive Arguments in Impulsatives are Subjects}
 
 
 
 
 
 
 Stative causatives differ from canonical causatives and impulsatives in that it allows its theme to move out and receive nominative case.  However, in canonical causatives and impulsatives the theme does not have this option and is ungrammatical with nominative case as well.
 
 
 \exg. *Jussi laula-tt-i Maija-a laulu\\
Jussi.NOM sing-CAUS-PST.3SG Maija-PART song.NOM\\
`Jussi made Maija sing a song.'
\label{caus unnom}
 
   \exg. *Maija-a laula-tta-a laulu.\\
Maija-PART sing-CAUS-3SG song.NOM\\
`Maija feels like singing a song.'
\label{imp unnom}


 
 
  Examples $\ref{imp unnom}$ and $\ref{caus unnom}$ are ungrammatical despite the nominative case on {\it laulu} `song'.  Example $\ref{caus unnom}$ already has a nominative argument {\it Jussi}, hence spec IP is already filled and not available to the object.  Even though, example $\ref{imp unnom}$ does not have any other nominative arguments, I propose that the partitive argument moves to spec IP and consequently blocks the theme from moving to that position.   \citet*{Koskinen:1999} argues that all subjects in Finnish, whether nominative or quirky, covertly raise to subject position (Topic/AgrP in his framework).  The partitive argument in canonical causatives, however, stays {\it in situ} and does not move.  This is supported by the following facts about extraction and control.  
  
%  The indication that the partitive argument in impulsatives moves to spec IP is word order. While Finnish has fairly free word order, it avoids verb initial declarative sentences \citep*{Vainikka:1989, Vilkuna:1989}.  It is also argued that negation is situated above TP \citep*{Brattico:2006, Holmberg:2001, Holmberg:2002a}.  Therefore, if the partitive argument appears before negation, then it can be said that the argument moves to a position above.  
%  
%  \exg. Maijaa ei laulata. \\
%  Maija.PART not.3SG sing.CAUS \\
%  `Maija doesn't feel like singing.'
%  
  
The first diagnostic for subjectivity is that only complement objects can be extracted while complement subjects cannot \citep*{Huhmarniemi:2011}.


  
  \exg. Mink\"a Jussi ajattel-i ett\"a Maija oli osta-nut?\\
  what.ACC Jussi.NOM think-3SG.PST that Maija PST buy-INF\\
  `What did Jussi think Maija bought?'
  \label{obj ex}
  
  \exg. *Kuka Jussi ajattel-i ett\"a oli ostanut talo-n? \\
  Who.NOM Jussi think-3SG.PST that PST buy-INF house-ACC? \\
  `Who does Jussi think bought a house?'
\label{subj ex}


In example $\ref{obj ex}$ the object is extracted from the embedded clause to form a grammatical question.  However, example $\ref{subj ex}$ where the complement subject is extracted is ungrammatical.  Thus, if an element cannot be extracted it patterns like a subject.  The prediction is that the partitive argument in impulsatives can not be extracted like a complement subject.  This prediction is borne out.


  \exg. *Ket\"a Jussi ajattel-i ett\"a laula-tt-i?\\
  Who.PART Jussi.NOM think-PST.3SG that laugh-CAUS-PST.3SG\\
  `Who did Jussi think felt like laughing?'
  \label{imp ex}
  
  Example $\ref{imp ex}$ is ungrammatical. This indicates that the partitive argument is a subject in an impulsative.  Conversely, the partitive argument in a canonical causative can be extracted in an embedded clause.
  
  
  \exg. Keta Maija ajattel-i ett\"a Jussi laula-tt-i? \\
    Who.PART Maija.NOM think.PST.3SG that Jussi.NOM laugh-CAUS-PST.3SG\\
    `Who does Maija think Jussi made laugh?'
    \label{caus ex}
    
    In example $\ref{caus ex}$ the partitive argument can be extracted in contrast to example $\ref{imp ex}$.  This is because, unlike the argument in $\ref{imp ex}$, the partitive argument does not move to Spec IP.  
    

  
  
  
A second test for subjectivity is control in adjunct clauses.   Subjects can control adjunct clauses while objects cannot.

\exg. He nauro-ivat laula-essa-nsa\\
3pl.NOM laugh-3PL.PST sing-while-3PX\\
`They laughed while singing.'
\label{adj subj}


\exg. *Jussi n\"ak-i heid\"at laula-essa-nsa.\\
Jussi.NOM see-3SG.PST 3PL.ACC sing-while-3PX\\
`Jussi saw us while singing.'
\label{adj obj}


  In example $\ref{adj subj}$ the subject of the matrix clause {\it he} `they' controls the subject in the adjunct clause.  However in example $\ref{adj obj}$, the object of the matrix clause cannot control the subject in the adjunct clause.   In impulsatives the partitive argument can control the subject of an adjunct clause, acting like a subject.
  
  


  
  \exg. Lapsi-a naura-tt-i katselle-ssa-nsa oma-a laulamista-an. \\
  children-PART laugh-CAUS-3SG.PST watching-WHILE.3PX own singing.3PX \\
  `The children felt like laughing while watching themselves sing.'
 \label{adjunct}
 
 
 In example $\ref{adjunct}$ the subject of the adjunct must be {\it meita} `we'.  This is also evident in the agreement on the verb in the adjunct  clause which carries third person plural agreement.    This indicates that the partitive argument moves to the spec of IP.  In contrast, in a canonical causative, the partitive argument cannot control the subject of an adjunct clause, patterning more like an object.
 
 \exg. *Jussi naura-tt-i heita laula-ssa-nsa\\
 Jussi.NOM laugh-CAUS-3SG 3PL.PART sing.while.3PX \\
 `Jussi made them laugh while singing.'
 \label{caus adj}
  
  
  In example $\ref{caus adj}$ the partitive argument {\it heita} `them' cannot control the subject of the adjunct clause.  Instead the nominative argument {\it Jussi} controls the subject of the adjunct clause.  The interpretation is that Jussi made them laugh while Jussi was singing.  This indicates that the nominative argument is the true subject of the sentence while the partitive argument is an object. 

  

  
  
 
The structure for canonical causatives therefore is still the same as in the structure given in $\ref{caus tree}$.  However, structure for impulsatives has changed in that now, the experiencer argument moves to spec of IP.  The proposed structure is provided below

%\ex.      \Tree [.IP NP$_{causer}$ [.CauseP t$_{causer}$ [.Cause' Cause\\-tta [.Cause' NP$_{causee}$ [.v' v VP ] ] ] ] ]
%\label{caus tree}
      
 \ex. \Tree      [.IP NP$_{Experiencer}$ [.ImpulseP t$_{Experiencer}$ [.Impulse' Impulse\\-tta [.v' v VP ] ] ] ] 
 \label{impulse tree1}

%
%\ex.      \Tree [.IP NP$_{causer}$ [.I' I [.CauseP t$_{causer}$ [.cause' cause [.-ttaP' NP$_{experiencer}$ [.-tta' tta [.v' v VP ] ] ] ] ] ] ]
%\label{caus tree}
%      
% \ex. \Tree      [.IP NP$_{Experiencer}$ [.I' I [.ImpulseP Impulse [.ttaP t$_{Experiencer}$ [.tta' tta [.v' v VP ] ] ] ] ] ]
% \label{impulse tree}
 
 
 
 
 
  
 






\subsection{Conclusion}


In this section, I reviewed previous accounts that relied heavily on the assumption that {\it -tta} was a causativizing morpheme.    Then I demonstrated that impulsatives do not involve causation.  This not only created problems for a causative analysis of impulsatives, but also for the assumption that {\it -tta} was always associated with a causative construction.  Therefore, I suggested that {\it -tta} is not causative but rather a reflection of unaccusative syntax.



%  Nelson notes that impulsatives share many properties with stative psych causatives; they are both stative, have partitive subjects and describe mental states.  The difference, she claims, is that stative causatives license two internal arguments while impulsatives appears to only have one.  

\section{Properties of the Impulse Head}

%
%Impulsatives introduce a modal world. Generally, functional heads responsible for introducing modals worlds are modals.  However, syntactically they differ from other elements that introduce modals worlds in Finnish. 
%Impulsatives in Finnish introduce their own arguments and have selectional restrictions whereas modals do not.\footnote{This can be interpreted two ways, either impulsatives are not modals or that some modals have more syntactic structure than others.}   


In the previous section, I posited a null impulse head that selects for an unsaturated Voice projection.  However, in order to establish a semantic denotation of the null impulse head, we must uncover other properties of the impulse head.  In this section, I will demonstrate that the null impulse head is a predicate over events, introduces an argument and has selectional restrictions.
 

Finnish impulsatives introduce an event.  Evidence of the presupposition generated by the adverb `again'.  Modals are not predicates over events, instead they have an event variable in the accessibility relation \citep*{Hacquard:2006}.  As a result, modals in Finnish do not generate a presupposition when they occur with the adverb `again';  only the embedded predicate generates a presupposition.  Impulsatives in Finnish, on the other hand, do generate a presupposition with the adverb `again'.   
\singlespace

\exg. Maija-n tayty-y opiskella uudestaan.\\
Maija-GEN must-3SG study.inf again.\\
`Maija must study again.'\\
Presupposition: Maija studied before. \\
 \# Maija had to study before.
\label{mod again}

\exg. Maija-a laula-tt-i uudestaan (taas).\\
Maija-PART sing-CAUS.PST-3SG again\\
`Maija felt like singing again.' \\
Presupposition: Maija felt before. \\
\# Maija sung before.
\label{again}


\doublespace
In example $\ref{mod again}$, only the embedded verb {\it opiskella} `study' generates a presupposition, while the modal {\it tayty} `must' does not.  Thus the sentence presupposes that Maija studied before, but she may not have had to.  However, in example $\ref{again}$, only the impulsative generates a presupposition.  Thus the sentence presupposes that Maija had an urge to sing previously but may have not sung.\footnote{Due to independent factors, Finnish does not allow the internal predicate to generate presuppositions.  In fact the internal predicate cannot be modified in any way.  Canonical causatives pattern the same way.}  This indicates that the impulse head is a predicate over events.



 
A second property of Finnish impulsatives is that they introduce an argument.  Consequently, it is incompatible with weather predicates because weather predicates do not have any arguments.  In contrast, modals in Finnish do not introduce arguments and are therefore compatible with weather predicates.  



\singlespace

\exg. Huomenna tayty-y sata-a / ol-la kaunis-ta. \\
tomorrow must-3SG rain-TA / be-TA beautiful-PART \\
`It has to rain / be beautiful tomorrow.'  \\
\citep*{Laitinen:1993}
\label{rain modal}

\exg.*Pian sada-tta-a. \\
Soon rain-CAUS-3SG \\
`It's about to rain.' 
\label{rain}

\doublespace
Example $\ref{rain modal}$ is grammatical because the modal {\it tayty} `must' does not introduce an external argument.  On the other hand, example $\ref{rain}$,  is ungrammatical because there is no argument even though the impulse head necessarily introduces an argument.  Since weather predicates do not have an argument this renders the impulsative ungrammatical.  


 Furthermore, the impulse head assigns its argument inherent partitive case.  This is supported by evidence from raising.   If the impulse head assigned structural case, the partitive argument should not retain its case when it raises to the subject position of a raising predicate \citep*{Sigurdsson:2002}.  
 
 
\exg. Maija-a n\"aytt\"a-\"a laula-tta-van\\
Maija-PART seem-3SG sing-CAUS-INF\\
`Maija seems to feel like singing.�
\label{raising}

In example $\ref{raising}$, the partitive case on {\it Maija} is retained.  Therefore, the case assignment is not structural but rather `quirky'.

%In addition,  impulsatives have selectional restrictions.  

%\exg. Maijan taytyy saapua. \\
%Maija-GEN must-3SG arrive.INF \\
%`Maija must arrive.' 

%\exg. Maijan taytyy pudota. \\
%Maija-GEN must-3SGt fall.INF \\
%`Maija must fall.' 


 In addition, as shown in section $\ref{impulsative}$, impulsatives are limited to unergatives and attempting to form impulsatives with unaccusatives results in ungrammaticality.  Examples are repeated below.

\exg. *Maija-a sapua-tta-a. \\
Maija.PART arrive.CAUS.3SG \\
`Maija feels like arriving.' 


\exg. *Maija-a pudo-tta-a. \\
Maija-PART fall.CAUS.3SG \\
`Maija feels like falling.' 




To account for selectional restrictions, I propose that the Impulse head in Finnish selects for \citet*{Ramchand:2011}'s process v', parallel to Albanian impulsatives.  



In this section, I have illustrated various properties of Finnish impulsatives.  Impulsatives in Finnish are predicates of events. Additionally, impulsatives in Finnish introduce their own argument and are responsible for assigning it inherent or quirky case.  Lastly, impulsatives select a very specific complement,  Process  V'.  This limits the verbs that can form impulsatives in Finnish to unergatives and some transitive verbs.\footnote{Although they are detransitivized.}









\section{Conclusion}


 I posit a null impulsative modal with the following denotation.

\ex. \doublebr{Impulse} = $\lambda$P$_{<e,vt>}$$\lambda$x.$\lambda$e.$\lambda$w.$\forall$w'[w' is compatible with what x has an impulse to do in e in w ] $\rightarrow$[$\exists$e' in w'.P(x)(e')] 

 The null impulsative modal does the following things.  It provides intensionality, it introduces an event, namely the `feeling like' event, and it introduces an experiencer argument.  In addition, it adds this to the assertion of the sentence.  The following is a sample derivation.

\singlespace


      
 \ex. \Tree      [.IP NP$_{Experiencer}$ [.ImpulseP t$_{Experiencer}$ [.Impulse' Impulse\\-tta [.{Process V'} {Process  V} ] ] ] ] 
 \label{impulse tree}


\ex. \a.  \doublebr{Process V'} = $\lambda$x.$\lambda$e. sing(e) \& Agt(e,x) \\
 \doublebr{Impulse}  = $\lambda$P$_{<e,vt>}$$\lambda$x$\lambda$e.$\lambda$w.$\forall$w' [w' is compatible with what x has an impulse to do in e in w] $\rightarrow$ [$\exists$e' in w'.P(x)(e')]\\
 Functional Application \\
\b. \doublebr{Impulse'} = $\lambda$x$\lambda$e.$\lambda$w.$\forall$w' [w' is compatible with what x has an impulse to do in e in w ] $\rightarrow$ [$\exists$e' in w'. sing(e') \& Agent(e',x))] \\
 \doublebr{Maija} = Maija \\
 Function Application \\
\b. \doublebr{ImpulseP} = $\lambda$e.$\lambda$w.$\forall$w' [w' is compatible with what Maija has an impulse to do in e in w ] $\rightarrow$ [$\exists$e' in w'. sing(e')  \& Agent(e',Maija))]\\


\doublespace


First, the predicate combines with its bar projection.  Then, the Impulse head selects for Process V'. This triggers the insertion of the ``causative" morphology because the Process V'
has an unsaturated argument. Next, it combines with its argument. Here it assigns its argument the experiencer theta role and dative case. In addition, the experiencer receives the
agent theta role in the possible world.


In conclusion, Finnish impulsatives are best analyzed by positing a null impulse head. This null head assigns partitive case to its argument and selects for Process V'. By selecting a category of Process V, this limits the construction to unergative and transitive verbs that are activities. By selecting a bar projection, this triggers the insertion of {\it -tta} following \citep*{Embick:2004} Distributed Morphology analysis. This analysis explains the ``causative" morphology, the source of the intensionality,  and the selectional restrictions of Finnish impulsatives. It also unifies the analysis with the analyses of other impulsatives in other languages, such as Bulgarian, Albanian, and Quechua.



%

%\bibliographystyle{plainnat}
%\bibliography{/Users/mec/Documents/Resources/bibtex/mec}

%

%\end{document}