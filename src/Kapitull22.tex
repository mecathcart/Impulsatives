%\documentclass{udthesis}
%\usepackage{covington}
%\usepackage{fullpage}
%%\usepackage{xyling}
%\usepackage{qtree}
%\usepackage{amsfonts}
%\usepackage{marvosym}
%\usepackage{lingtrees}

%\usepackage{hyperref}
%\usepackage{linguex}
%\usepackage{setspace}
% \usepackage{natbib}
% \bibpunct{(}{)}{;}{a}{,}{,}
%%\qtreecenterfalse

%\newcommand{\doublebr}[1]{[\hspace{-.02in}[{\bf#1}]\hspace{-.02in}]}
%\newcommand{\doublebrexpand}[1]{$\left[\hspace{-.06in}\left[#1\hspace{-.5in}\right]\hspace{-.06in}\right]$}

%

%\author{Albanian Chapter}
%\title {Kapitull}
%\begin{document}
%\pagenumbering{arabic}
%\maketitle

%\doublespace


\chapter{Albanian}

\section{Introduction} 

Albanian is spoken by almost six million people, primarily in the country of Albania but also other areas of the Balkans \citep*{Ethnologue:2005}. Albanian is an Indo-European language and constitutes its own branch.  Consequently, Albanian is genetically distant from other Indo-European languages.  Its default word order is SVO and has a morphological fusional nature.  Albanian nouns are inflected by case (nom-acc system), gender and number and Albanian verbs are inflected by person, number, tense and voice \citep*{Hubbard:1985}.


In this chapter, I discuss impulsatives in Albanian. The goal of this chapter is to provide a syntactic structure and semantic derivation for impulsatives in Albanian.   Like Bulgarian impulsatives, Albanian impulsatives lack a dedicated morpheme to indicate the intensionality in the construction.    In Albanian, impulsatives are composed of the non-active form of the verb and a dative argument, as shown in example $\ref{albanian}$.\footnote{Unless otherwise stated, the examples are Albanian elicited from two speakers from Tirana, Albania.}
  
  \singlespace

\exg.  Agim-it  i k\"erce-het n\"e zyr\"e.\\
Agim.DAT DAT.3SG. dance.3SG.NACT.PR in office\\
`Agim feels like dancing in the office.' \\
`There was dancing in the office and it affected Agim (i.e. it was Agim's office).'
\label{albanian}


\doublespace


In example $\ref{albanian}$ the argument {\it Agim} is marked with dative case {\it -it} and the verb is in the non-active form.  Example $\ref{albanian}$ has two readings.  One reading is the impersonal passive reading where there is dancing happening and Agim is somehow affected.  I will refer to this type of reading as the affected argument reading \citep*{Pylkkanen:1999, Bosse:2008}.\footnote{English translations for affected readings are not exact, for simplicity I will use `on/for Agim'.}  The second reading is that Agim has an urge or impulse to dance. This is the impulsative reading.  Impulsatives introduce modal semantics or intensionality because in the modal world of the impulsative, the experiencer is the external argument of the verb.  All potential impulsatives are ambiguous between the affected argument and the impulsative readings in Albanian. 





 In this chapter, I extend the analysis of Bulgarian impulsatives to Albanian impulsatives.  I propose that Albanian impulsatives have a covert impulse head. The null impulse head contributes the intentionality in Albanian impulsatives.  I reject analyses that treat impulsatives without a dedicated null element; specifically \citet*{Rivero:2009}, whose analysis uses an imperfective operator as the source of the intensionality and \citet*{Kallulli:2006}, whose analysis involves non-active morphology to derive the impulsative reading.  Albanian impulsatives have two properties that will play important roles in the analysis: they are bi-eventive and, unlike Bulgarian, require selectional restrictions.  These two properties motivate the positing of a phonologically null (covert) impulse head in Albanian.  Having established a null impulsative element, my next step is to determine its semantic and syntactic properties.  First, I show that Albanian impulsatives are not a modal.  Then, I demonstrate Albanian impulsatives abide by the definition for impulsatives I developed in chapter 1.  Next, I provide a monoclausal analysis that explains both the selectional restrictions and non-active morphology.  Following \citet*{Ramchand:2011}, I propose that the null head selects for a projection of process v.  Then, I adopt Embick's \citeyearpar{Embick:2004} analysis of non-active voice in Albanian and further propose that the null head selects process v', when the external argument of the internal predicate has not yet been projected.  With the syntactic structure in place, I elaborate its semantic composition.  I claim that it has the following denotation.
 
\ex. \doublebr{Impulse} = $\lambda$P$_{<e,vt>}$$\lambda$x.$\lambda$e.$\lambda$w.$\forall$w'[w' is compatible with what x has an impulse to do in e in w ] $\rightarrow$[$\exists$e' in w'.P(x)(e')]


Semantically, the null impulse head will do several things.  First, it will provide intensionality by quantifying over possible worlds.  Secondly, it has an event argument. Finally, it introduces an experiencer argument and links it with the agent of the internal predicate in the modal world. Syntactically, the impulse head will license and case-mark an experiencer argument and select for an unsaturated process V projection.    The structure for example $\ref{imperfect0}$ is shown in the tree below.   

\exg. Nj\"e moll\"e (Agimit) i hahej\\
A apple.INDEF  Agim.DAT 3SG.DAT.cl eat.NACT.IMP.3SG \\
`I was feeling eating an apple.'
\label{imperfect0}


 \ex. \Tree [.TP apple$_i$ [.T' T [.ImpulseP Agim.DAT [.ImpulseP' {Impulse\\ NACT}  [.{Process V'} eat {\it t$_i$}  ] ] ] ] ]


This chapter is organized as follows.  The second section is dedicated to finding the source of the intensionality. First, I show that unlike Bulgarian, aspect plays no role in Albanian impulsatives.  Therefore, I can reject Rivero's \citeyearpar{Rivero:2009} analysis. Like Rivero, \citet*{Kallulli:2006} presents an analysis with no null element involving non-active morphology which is present in both Bulgarian and Albanian impulsatives.  However, Kallulli's treats impulsatives as undecomposible states: states that cannot be broken up into sub-events.  This runs contrary to complex predicate analyses proposed \citep*{Massey:1991} and the biclausal analysis for the Slavic languages proposed by  \citet*{Murasic:2006}.  I argue in favor of the latter treatment of Albanian impulsatives because of their bi-eventive properties.   I argue that the source of the intensionality must come from a null element.  In the third section, I argue that the null impulsative is not desiderative predicate, but rather an impulsative, as defined in chapter 1. Furthermore, the null impulsative is responsible to introducing and assigning dative case to its experiencer argument.  In the fourth section, I determine the syntactic structure of Albanian impulsatives.  I show that they are not biclausal and that a monoclausal analysis accounts for both selectional restrictions and non-active morphology.   With regards to selectional restrictions, I suggest that the null impulse head selects process V following \citep*{Ramchand:2011}.  Then I  present a monoclausal structure that derives the non-active morphology by using Distributive Morphology \citep*{Halle:1993b}.  Finally, section five provides a full derivation of the analysis and concludes the paper.






\doublespace

\section{The Source of the Intensionality}


Like Bulgarian impulsatives, Albanian impulsatives lack a dedicated lexical or morphological element denoting the impulsative meaning.  This is problematic because the source of the intensionality is mysterious.  In the first section, I discuss the morphological make-up of Albanian impulsatives and show that the intensionality cannot come from any over morphological component.  However, Albanian impulsatives differ from Bulgarian impulsatives in two ways.  First, Albanian impulsatives do not have any aspectual restrictions. Consequently, Rivero's \citeyearpar{Rivero:2009} analysis cannot apply to Albanian.  Second, Albanian impulsatives have selectional restrictions.   \citet*{Kallulli:1999b} claims that the impulsative meaning is derived when non-active morphology is applied to process verbs.  She argues that the non-active morphology reduces the event structure of a process verb to that of a stative. Kallulli's analysis ultimately fails because, like Bulgarian impulsatives, Albanian impulsatives are bi-eventive. I argue that, in fact, selectional restrictions together with bi-eventivity indicate that Albanian impulsatives have a null impulsative element responsible for the intensionality. 




     \subsection{The Composition of  Albanian Impulsatives}
     \label{sec:Morphological Make-Up A}
     
     Albanian impulsatives present the same puzzle as Bulgarian impulsatives; while the impulsative construction as a whole introduces modal semantics, none of its lexical, morphological or syntactic components singularly do. The morphological makeup of Albanian impulsatives appears to only have two obligatory elements: non-active morphology and a dative argument.  Non-active morphology and dative arguments are individually used elsewhere in the grammar, in non-intensional contexts.  In addition, the predicate in Albanian impulsatives is restricted to a certain class of verbs; namely unergatives and some transitives.   None of these elements specifically contribute the intensionality to Albanian impulsatives.
     
     
   As mentioned in the introduction,  Albanian impulsatives are composed of the non-active form of the verb and a dative argument as shown in example $\ref{impulsative1}$.
      
      \singlespace

\exg.  Agimit k\"ercehet n\"e zyr\"e.\\
Agim.DAT DAT.3SG.cl dance.3SG.NACT.PR in office\\
`Agim feels like dancing in the office.' \\
\label{impulsative1}


\doublespace


Non-active voice morphology in Albanian also appears on passives$\ref{passive alb}$, unaccusatives$\ref{unaccusative alb}$ and reflexives$\ref{reflexive alb}$.
%explain that non-active in Albanian is a conjugated thing.

\exg.  Molla hahej.\\
apple.NOM eat.3SG.NACT.PST \\
`The apple was eaten.'
\label{passive alb}

  
\exg. Papritmas, u duk dielli. \\
  suddenly NACT.cl appear.3SG.NACT.PST sun.NOM \\
  `Suddenly, the sun appeared.'
  \label{unaccusative alb}
  
  \exg. Un\"{e} ushqehem.  \\
1SG.NOM feed.1SG.NACT  \\
  `I feed myself.'
  \label{reflexive alb}


In example $\ref{passive alb}$, the non-active morphology makes the sentence a passive. Example $\ref{unaccusative alb}$ is an example of an unaccusative verb in Albanian. Example $\ref{reflexive alb}$ is a verb that can be made reflexive by the addition of the non-active morphology. None of these examples $\ref{passive alb}$-$\ref{reflexive alb}$ are intensional.
     
  
  Dative arguments in Albanian can also occur in multiple constructions. In addition to occurring in ditransitive constructions, dative arguments can also appear as subjects of psych predicates and as affected
arguments in applicative constructions \citep*{Pylkkanen:2002a}.
  
  \singlespace
  \exg. Agim-i i-a dha Drit\"es libr-in. \\
Agim-NOM  DAT.3SG.-ACC.3SG give.3SG.PST Drita.DAT book.ACC \\
 `Agim gave Drita the book.'\\
\label{ditransitive a}

\exg. Kamerierit nuk i u durua m\"e vetja. \\
waiter.DAT NEG 3SG.DAT  NACT endure.3S.PST.NACT more itself.NOM \\
`The waiter could not endure himself any longer.' \\
\citep*[p 69]{Massey:1991} \\
\label{psych pred a}

\exg. Agim-i	i-a	theu	vazon	Dritan-it. \\
Agim-NOM	3SG.DAT-3SG.ACC 	break.3SG.PST 	vase.ACC		Dritan.DAT \\
`Agim broke the vase on Dritan.' \\
= `Agim broke the vase, and this matters to Dritan (negatively or positively).' 
\label{aff arg a}

 \doublespace
     
     In example $\ref{ditransitive a}$, the dative argument functions as the second argument in a ditransitive, while it serves as the subject of a psych predicate in example $\ref{psych pred a}$. Finally, in example $\ref{aff arg a}$ the dative argument can take a number of affected roles in applicative constructions.  While $\ref{aff arg a}$ can be interpreted many ways, none of these interpretations
are intensional. Furthermore, neither $\ref{ditransitive a}$ and $\ref{psych pred a}$ are intensional.
     
However, unlike Bulgarian, impulsatives in Albanian cannot be used with any predicate.  Impulsatives appear to be restricted to some transitive verbs and unergatives.


\exg. M\"e shtypeshin fara p\"er tre or\"e.\\
1SG.DAT.cl  pound.3PL.NACT.IMP seed.PL.INDEF.NOM for three hour.PL\\
`I felt pounding seeds for three hours.' 

\exg. M\"e punohet shum\"e. \\
I.cl.DAT work.3SG many\\
`I feel very much like working.' 


 Furthermore, impulsatives in Albanian exclude certain verbs, such as unaccusatives \citep*{Hubbard:1985}\footnote{Rivero cites Kallulli (pc) that unaccusatives are acceptable for some speakers}, and causative verbs.\footnote{\citet*{Kallulli:2006b} mentions that certain verbs  such as causatives cannot appear in an impulsative with perfect aspect.  The speakers I consulted did not find the impulsative reading unavailable based on aspect but ruled out the impulsatives with those predicates entirely.  Thus, they also judged the imperfect form to also lack the impulsative reading 
 
\exg.Ben-it thy-hej nje vazo.\\
Ben-theDAT himcL break-NACT,P,IMP3S a vase.NOM \\
(i) `$*$Ben felt like breaking a vase.' \\
(ii)`$*$Ben unintentionally broke a vase.' 

.}   
     
     
    
\singlespace

\exg.Ben-it i vdis-et. \\
Ben.DAT DAT.3SG die.3SG.NACT.PR \\
`Someone died on Ben'\\
$*$`Ben feels like dying.' \\
\label{die alb}

%\exg. Nj� mal m�u ngjit.\\
%a mountain.s.def.NOM me.DAT.cl�nonact climb.3.s.nonact\\
%`To me a mountain was climbed.' \\
%$*$`I felt like climbing a mountain.' \\


\exg. Benit iu thye nj\"e      vazo. \\
Ben.DAT DAT.3SG-NACT.cl break.NACT.3SG.PST a vase.INDEF \\
`To Ben, a vase was broken.' \\
$*$`Ben felt like breaking a vase.' \\
\label{causatives0}
%actually i should give the imperfect example too



\exg. Ben-it i nderto-hej nje shtepi. \\
Ben.DAT DAT.3SG build-NACT.3SG.IMP. a house.NOM \\
`For Ben, a house was built.' \\
 $*$`Ben felt like building a house.' \\
 \label{build0}

%
%\exg. Ben-it i-u ndertua nje shtepi. \\
%Ben-theDAT himcL-NACT build.AOR,3S a houseNoM \\
%(i) '$*$Ben felt like building a house' \\
%(ii)'Ben unintentionally built a house' \\





\doublespace


In example $\ref{die alb}$, the verb {\it vdes} `die' is conjugated in non-active third person singular form and the argument {\it Ben} carries dative case {\it -it}, however, in this sentence the impulsative reading is unavailable, only the affected argument reading is available.  Similarly, examples $\ref{causatives0}$ and $\ref{build0}$, despite having both non-active morphology on the predicate and dative case marking on the animate argument,  does not yield the impulsative reading.  


Thus the composition of an Albanian impulsative consists of a dative argument and a particular predicate with non-active morphology.  None of these elements is intensional.  Therefore, the intensionality of Albanian impulsatives is not readily apparent given its composition.   In the following we will see different ways of providing the intensionality, and ultimately conclude that it comes from a dedicated covert intensional impulse head.


\subsection{The Role of an imperfective Operator in Albanian Impulsatives }
\label{Rivero}


In the last chapter, I reviewed Rivero's \citeyearpar{Rivero:2009} analysis and argued that it did not work for Bulgarian impulsatives because an imperfective operator IMP$^{OP}$ was neither necessary nor sufficient to provide the source of intensionality in Bulgarian impulsatives.   While the role of aspect in Bulgarian impulsatives is complicated and contradictory, it is clear that in Albanian, aspect plays no role in Albanian impulsatives.\footnote{I base my analysis on the judgements received from my informants.  It does appear that aspect plays a role for Kallulli.}  First, imperfect morphology is not obligatory on Albanian impulsatives.  Furthermore, Albanian impulsatives are not limited to imperfect readings.  Finally, unlike Bulgarian imperfects, Albanian imperfects do not introduce intensionality. Therefore, even if it were present in Albanian impulsatives, it cannot be the source of the intensionality.  In addition, Rivero's use of a high applicative head to introduce the experiencer argument is problematic for Albanian impulsatives.  Finally, Rivero's analysis does not explain the role of non-active morphology in Albanian impulsatives.  Thus, Rivero's analysis cannot account for either Bulgarian or Albanian impulsatives.  


While imperfect morphology is sometimes obligatory in Bulgarian impulsatives, it is not necessary in Albanian impulsatives.  Albanian impulsatives can appear in both the aorist past (perfective aspect) and the imperfect past (imperfective aspect). \footnote{\citet*{Kallulli:2006b} mentions that certain verbs cannot appear in an impulsative with perfect aspect.  The speakers I consulted did not find the sentences ungrammatical based on aspect but ruled out the impulsatives with those predicates entirely.}

\exg.Nj\"e moll\"e m'u h\"eng\\
A apple.INDEF  1SG.DAT.CL eat.NACT.PST.3SG \\
`I felt like eating an apple.'
\label{perfect}

\exg.Nj\"e moll\"e m\"e hahej\\
A apple.INDEF  1SG.DAT.cl eat.NACT.IMP.3SG \\
`I was feeling eating an apple.'
\label{imperfect1}

In examples $\ref{perfect}$ and $\ref{imperfect1}$ the tense and aspect morphology modify the `feeling like' event and not the `eating' event.  Thus it appears that the IMP$^{OP}$ necessary for Rivero's analysis is not obligatorily present in Albanian impulsatives.


However, it could be the case that there is a covert imperfect operator that would apply to the internal `eating' event.   If the internal event were imperfect, it would be atelic and unbounded.  The imperfect is incompatible with a telic event as illustrated in example $\ref{telic}$.
 
 \exg. *Haja nj\"e moll\"e br\"enda nj\"e minut\"e. \\
 eat.IMP.1SG a apple.INDEF in a minute. \\
`I  was eating an apple in one minute.'
\label{telic}
 
  In example $\ref{telic}$, the verb is inflected for the imperfect past and the event is bounded by the adjunct {\it br\"enda nj\"e minut\"e} `in one minute'.  Since the imperfect morphology indicates that the event is not completed, the adjunct that refers to the time of completion creates a conflict.  However, Albanian impulsatives can describe a telic event, as in example $\ref{alb telic}$.  

 \exg.Nj\"e moll\"e m\"e hahej br\"enda nj\"e minute\\
a apple.INDEF me.DAT eat.NACT.PST.3SG. in a minute\\
`I felt like eating an apple in a minute.'
\label{alb telic}


In example $\ref{alb telic}$, the internal event is an apple being eaten within a minute.  Thus the internal event must be a completed action.  The adverb `in a minute'  makes the `eating' event bounded and therefore incompatible with a progressive reading.  Therefore, the internal event in Albanian impulsative constructions is not obligatorily imperfective.  Rivero's analysis for impulsatives does not account for the Albanian impulsatives because there is no evidence that the Albanian impulsatives has the IMP$^{OP}$.  


Furthermore, the Albanian IMP$^{OP}$ appears not to be a source of intensionality.  If the imperfect is a possible source of intensionality in Albanian, the futurate should be available when there is the imperfect form and a nominative argument, as it does in Bulgarian.  


 \exg. *Per dy jave skuadra (po) luante neser. \\
For two weeks team.NOM (PROG) play.IMP.3SG tomorrow. \\
`For two weeks, the team was playing tomorrow.'
\label{Alb Progressive}


The imperfect past in Albanian does not yield a futurate reading.  Thus, when there are two time adverbs, as in example $\ref{Alb Progressive}$, a conflict arises because they are both attempting to modify the same event, in this case the `playing' event.   Moreover, the futurate reading is still unavailable even when the progressive {\it po} is added.  Thus neither the imperfect or the progressive forms in Albanian can introduce an intensional reading.  Even if there was a covert imperfective operator in Albanian impulsatives, it would not introduce any intensionality to Albanian impulsatives. 


As I mentioned in the previous chapter, Rivero's analysis also includes a high applicative.  Including a high applicative in an analysis of Albanian impulsatives is also problematic. In her analysis, the dative is introduced by a high applicative phrase that is above the TP.  This predicts that the nominative argument in spec of TP should not be able to c-command the dative argument projected above spec TP.  However, this prediction is not borne out, and the nominative argument can c-command the dative argument.

\exg.Peshkaqen\"et$_1$ i hahen nj\"eitjetrit$_1$.\\
Shark.pl.NOM CL.pl eat.NACT.3PL each.other.DAT\\
`The sharks feel like eating each other.'
\label{c-command0}

In example $\ref{c-command0}$, the nominative argument {\it peshkaqen\"et} `the sharks' c-commands the dative argument {\it nj\"eitjetrit}  `each other'.  {\it Nj\"eitjetrit} `each other'  must be co-referential with the nominative argument {\it Peshkaqen\"et} `the sharks'.  Therefore, the nominative argument must be above the dative argument, contrary to Rivero's account.


Finally, as with Bulgarian impulsatives, Rivero's analysis does not explain the role of non-active morphology in Albanian impulsatives.  Rivero's analysis does not predict that impulsatives in Albanian would be limited to non-active sentences. However, as mentioned in section 4.2.1, without the non-active morphology, the active sentence only has an affected argument reading.



\singlespace
\exg. I-a lexoi librin Bes\"es. \\
DAT.cl. ACC.cl read.3SG.PST book.ACC Besa.DAT\\
`S/he read Besa's book and it matters to her.'\\ 
$*$`Besa felt like reading a book'

\doublespace




Moreover, just like in Bulgarian, non-active morphology applies differently to applicatives than impulsatives in Albanian. A predicate with non-active morphology  associated with an affected argument still has a passive meaning; as in $\ref{apple}$.  In example $\ref{apple}$  the agent of the overt predicate `eat' is existentially bound. In contrast, impulsatives do not involve existential binding of the agent of the overt predicate because the dative argument necessarily receives the agent theta role. The dative argument in impulsatives doubles as the experiencer of the �feeling� event and the agent of the internal predicate. In contrast, applied arguments do not carry any other theta role.

\singlespace

\exg. Dritanit i'u h\"eng nj\"e moll\"e \\
Dritan.DAT 3SG.DAT.cl-NACT.cl eat.3SG.PST.NACT a apple.INDEF  \\
`Dritan felt like eating an apple.' \\
`An apple was eaten on Dritan'
\label{apple}





\doublespace




In the impulsative example in $\ref{apple}$, { \it Dritan} is both the feeler and the eater. It cannot be the case that Dritan desires for anyone else to do any eating.. In the affected argument reading however, Dritan cannot have eaten an apple. The agent of eating must be someone else. Similarly to Bulgarian, Rivero's analysis fails to capture the role of non-active morphology in impulsatives in Albanian.

In this section, I have shown that Rivero's analysis cannot extend to Albanian impulsatives: there are problems with adopting both the IMP$^{OP}$ and the high applicative, which she argues are crucial for her analysis.  Furthermore, her analysis does not explain the role of non-active morphology in Albanian impulsatives.


     
     \subsection{\citet*{Kallulli:1999b}'s Non-Active Analysis}
     
     While one of Rivero's \citeyearpar{Rivero:2009} major shortcomings was the sidelined role of non-active morphology, in  \citep*{Kallulli:1999b} its role is central.  \citet*{Kallulli:1999b} claims that non-active morphology can affect the lexical meaning of a predicate, depending on the predicate type.   Specifically, predicates that denote activities such as unergatives and many transitives when affixed with non-active morphology and associated with a dative argument can yield an impulsative reading.  
     
     
	According to Kallulli, non-active morphology in Albanian is a morphological operation which operates on event structures. There are three primitive event types \citep*{Pustejovsky:1991}.  The most primitive type are states: a single event which is evaluated relative to no other event.  These include such verbs as `know' and `believe'.  The second event type are processes: a sequence of identical events identifying the same semantic expression.  These include verbs like `run' and `work'.  Lastly there are transitions: a single event identifying a semantic expression which is evaluated relative to another single event, namely, its opposition.  These verbs include `break' and `wet'.  
		

Kallulli argues that non-active morphology in Albanian is an event type-shifting device, applying to higher event types to yield lower event types.  The highest event types are transitions while the lowest event types are states.   When non-active morphology is affixed to a predicate, it shifts the event type associated with the predicate into a lower event type.  When non-active morphology applies to a process predicate it becomes a stative predicate.


Essentially, Kallulli argues that in the impulsative construction, the process event is type-shifted into a stative event.  The dative argument cannot be the agent because the stative has no agent.  Therefore, according to Kallulli, it must be the experiencer.  The impulsative reading is attained pragmatically to link the experiencer with the stative event.  


     \subsection{ There Must be a Null Element}
\label{There must be a null element}

This section begins by demonstrating that \citet*{Kallulli:1999b}'s approach is misguided precisely because the event structure of impulsatives is not that of a simple stative. On the contrary, impulsatives contain two sub-events.  I agree with both \citet*{Massey:1991} and \citet*{Murasic:2006} that impulsatives have a null element.  However, Kallulli's \citeyearpar{Kallulli:1999b} observation that only process verbs can yield impulsative readings is valid.  I adopt  Massey's \citeyearpar{Massey:1991} suggestion that the verbal restrictions are due the null element selecting process verbs.  Thus, by positing a null element, selectional restrictions and the bi-eventivity of impulsatives in Albanian are explained.


Kallulli's \citeyearpar{Kallulli:1999b} approach is attractive because it explains the obligatory presence of non-active morphology and availability of impulsative readings to only activity verbs.  However, it does so at the cost of the event structure of impulsatives.  Like Bulgarian impulsatives, Albanian impulsatives are bi-eventive.  Maru\v{s}i\v{c} and \v{Z}aucer's (2006) diagnostics determine that Albanian impulsatives are bi-eventive contrary to Kallulli's event reduction analysis.


   Under Kallulli's \citeyearpar{Kallulli:1999b} analysis, the event is the lowest event type there is and cannot be decomposed farther.  This means that impulsative constructions should not be able to be decomposed into two sub-events.  However, Albanian impulsatives  can support two conflicting time adverbs.
   
    \exg. Dje m\"e k\"ercehej sot. \\
Yesterday, 1SG.DAT.cl dance.NACT.PST today. \\
`Yesterday I felt like dancing today.'
\label{bieventive}

  Example $\ref{bieventive}$ could be used in a context where tonight there is a party, and yesterday, I felt like dancing at the party tomorrow.  Example $\ref{bieventive}$ has the two sub-events.  Each event, the impulse and the `dancing', can occur at two separate times.  In this case, the impulse happened yesterday, but the dancing was intended to occur today, not at the time of the impulse. This is unexpected if there is only event.  Non-derived statives in Albanian cannot support two conflicting time adverbs as in $\ref{love}$.
   

\exg. *Sot, Besa e dashuron (at\"e) dje.\\
Today Besa.NOM ACC.cl love.3SG.PST him yesterday.\\
`Today, Besa loved him yesterday.'
\label{love}
   
 Kallulli (2006) claims that some non-derived statives can support two conflicting time adverbs as in example $\ref{recall}$.

\exg. Sot m\"e kujtohen fjal\"et/rrobat e An\"es dje. \\
Today, me.DAT recall.NACT.PR.1SG  words/clothes.NOM AGR Ann.DAT yesterday.\\
`Today, I recall Anna's words/clothes of yesterday.'
\label{recall}

However, this does not invalidate Maru\v{s}i\v{c} and \v{Z}aucer's (2006) diagnostic for bi-eventivity.  In order to have any sort of interpretation, $\ref{recall}$ must have two event variables.  In example $\ref{recall}$ the time adverb {\it Sot} `Today' modifies the time I was recalling while the second time adverb {\it dje} `yesterday' modifies the time Anna's words were uttered or her clothes were worn.   If both adverbs were to modify the same event, the one denoted by the predicate {\it kujtohen} `recall', the sentence would lead to a contradiction.  Thus, the predicate introduces two event variables: the event in which the person is doing the recalling and the event that is recalled.  Hence, the verb  {\it kujtohen} `recall' in Albanian is different in this respect from canonical statives like {\it dashuron} `love' in example $\ref{love}$.

Nevertheless, there is a crucial difference between a non-derived stative including {\it kujtohen} `recall' and an impulsative.  An impulsative introduces an event with a proposition, whereas a non-derived state does not.  Adverbs such as `again' generate a presupposition for every proposition in the sentence \citep*{Stechow:1984, Bale:2006}. 

  \exg. Agimit k\"ercehet s\"erish. \\
   Agim.DAT dance.NACT.PST.3SG again. \\
   `Agim feels like dancing again.' 
   \label{again a}
   
   The example in $\ref{again a}$ has two possible presuppositions.   The first scopes over the impulse, generating the presupposition that Agim felt like dancing before but had never danced before.  The second presupposition is where Agim may have danced before, albeit begrudgingly, however, now that he knows how to, he feels the urge to dance again.  In contrast, the psychological predicate {\it kujtohen} `recall' only generates one presupposition despite being bi-eventive.

\exg.  M\"e kujtohen fjal\"et/rrobat e An\"es s\"erish. \\
me.DAT recall.NACT.PR.1SG words/clothes.NOM AGR Ann.DAT again.\\
`I recall Anna's words/clothes again.'
\label{recall again}

Example $\ref{recall again}$ only has one presupposition, that is that the recalling event took place before.  It does not generate the presupposition that Anna's words were spoken or that Anna's clothes were worn again. Therefore,  there is only one proposition for a stative predicate such as  {\it kujtohen} `recall'.  Kallulli's analogy to non-derived states disintegrates when comparing examples $\ref{again a}$ and $\ref{recall again}$.  This suggests that not only are impulsatives bi-eventive but that they introduce the impulsative event via a proposition, unlike the non-derived stative {\it kujtohen} `recall'.


The fact that impulsatives are bi-eventive suggests that there is a null element.  There needs to be an element that introduces the impulsative event into the semantics.  \citet*{Massey:1991} suggests that impulsative constructions have a complex predicate.  Under her analysis, non-active morphology is realized in Infl.  Non-active Infl takes as its complement a VP headed by an empty verb. Then, the empty verb selects an activity verb, as shown in the structure $\ref{Massey}$ below.  By positing an empty verb, Massey simultaneously accounts for both the bi-eventive nature and the verbal restrictions.  While Massey did not consider event structure when she proposed this structure, the data on adverbs provide additional evidence.   Abverbs are provided another node to attach to, either modifying the empty verb or the activity verb.  In addition, the idiosyncratic class of verbs allowed in impulsative constructions is explained as a by-product of selectional restrictions.
       
\Tree [.IP [.NP -theta ] [.I' [.I -act ] [.VP [.NP exp ] [.V' [.V e ]  [.VP [.V' [.V DO ] PC ] ] ] ] ] ]
\label{Massey}

Both bi-eventivity and selectional restrictions provide motivation for positing a null element in Albanian impulsatives. Despite an analysis involving the event structure of non-active constructions, Kallulli's \citeyearpar{Kallulli:2006} analysis ultimately fails to account for the event structure of impulsatives in Albanian.  As with Bulgarian impulsatives, Albanian impulsatives are shown to be bi-eventive by adopting Maru\v{s}i{c} and \v{Z}aucer's \citeyearpar{Murasic:2006} approach. Finally, by viewing the verbal category that yields impulsatives as a result of selectional restrictions as suggested by \citet*{Massey:1991}, a null impulse head is further warranted.    Thus, I conclude that impulsatives have a null element.


\subsection{Conclusion}

This section began by establishing there was no overt intensional element in Albanian impulsatives.  Next, I reviewed two accounts that attempt to derive the impulsative reading without a null intensional element.  First, was Rivero's \citeyearpar{Rivero:2009} analysis of Slavic impulsatives involving aspect.  However, it was shown that Albanian impulsatives do not depend on imperfective aspect.  Second was Kallulli's \citeyearpar{Kallulli:2006} account that derived the meaning by claiming that the non-active morphology reduced the event structure of a process verb to a state.  However, following \citet*{Murasic:2006}, I demonstrated that Albanian impulsatives are more than a single state and in fact are bi-eventive.  This suggests that Albanian impulsatives involve a null element.   Finally, positing a null element in Albanian impulsatives is further supported because it allows for a straightforward way to capture the class of verbs that yield impulsatives as a result of selectional restrictions as suggested by  \citet*{Massey:1991}.  In the following sections, I will elaborate what the null element looks like and provide a different account of non-active morphology from \citet*{Murasic:2006} and \citet*{Massey:1991}.
 



\section{ What is the Null Element?}

In this section, I discuss the nature of the null element.  First, I show that it is not a desiderative/volitional verb.  Because of its particular syntax and semantics, I argue that the null element is an impulsative as defined in Chapter 1. Then, I demonstrate that the impulse head introduces a dative experiencer argument.

  \subsection{The Element is an Impulsative}
     
     
     As mentioned in the Bulgarian chapter, Maru\v{s}i\v{c} and \v{Z}aucer consider the null verb a desire/volitional predicate, class 3 under Belletti and Rizzi \citeyearpar{Belletti:1988a}'s classification of psych predicates. However, as I have outlined before, there are several systematic differences between impulsatives and traditional desideratives.  I will compare the Albanian impulsative to the Albanian volitional verb {\it dua} `want'.  The first is a lexically semantic one; impulsatives are not volitional.  Secondly, there are syntactic differences having to do with case and agreement. 
     
     
     Impulsatives in Albanian are not volitional.  They are often translated as `feel like' or `in the mood to' as opposed to being translated as `will' or `want'.  In addition, impulsative constructions are most salient with verbs that involve bodily functions such as `cough', `sneeze', and `vomit' as shown in the example below.
     
     \singlespace
     
     \exg. M'u voll. \\
     1SG.cl.DAT-NACT  vomit.PST.\\
     `I felt like vomiting.' \\
      `A vomit came over me.'
     
     \doublespace
     
     It is semantically salient to have an urge or impulse to vomit; however, vomiting is something few people desire to do in the volitional sense.   Furthermore, it is possible for the desire and impulse to be two separate things.  Consider the following example.
     
     \singlespace
     
     \exg. Agimit i fli-het.\\
    Agim.DAT DAT SIeep.3SG.NACT\\
     `Agim feels like sleeping.'\\
     `Agim is sleepy.'
     \label{sleep a}
     
     \doublespace
     
     Example $\ref{sleep a}$ is salient in a context where Agim is tired even though he may not want to sleep. For instance, it may be New Year's Eve and Agim wants to be awake at midnight but is very tired. However, this sentence cannot be used when what Agim wants is not what he is feeling. The sentence cannot be used in a context where Agim has a busy day the next day and wants to get a good night's rest but cannot fall asleep.
     
     \doublespace
     
     Moreover, syntactic properties of Albanian impulsatives are different from those of volitional desideratives.  The volitional verb {\it dua} `want' in Albanian takes nominative case, the canonical subject case in Albanian.  
     
     \exg. Jani do t\"e blej\"e nj\"e makin\"e.  \\
     Jan.NOM want.3SG COMP buy.3S.SUBJ a car  \\
       `Jan wants to buy a car/will buy a car.'
  \label{Jan wants}
     
     In addition, the verb {\it dua} `want' also agrees with its  subject.
     
 \exg.    Dua t\"e blej nj\"e makin\"e.\\
 want.1SG COMP buy.1SG.SUBJ a car\\
 `I want to buy a car.'
 \label{I want}
 
 In example $\ref{Jan wants}$, the subject {\it Jan} takes the nominative marking {\it i}.   In addition, the verb {\it do} is conjugated in the third person singular form in contrast to example $\ref{I want}$ where the verb is conjugated with the first person singular form {\it dua}\footnote{Citation form in Albanian is the first person singular form.}.
 
 In contrast, subjects in impulsatives carry dative case and do not trigger agreement with the predicate.  
     
     
     
 \exg.   Agimit i hahet nj\"e moll\"e. \\
    Agim.DAT DAT.cl eat.3SG.NACT a apple. \\
     `Agim feels like eating an apple.'
     \label{sing}
     
     \exg. Agimit dhe Dritanit u hahet/*hahen nj\"e moll\"e.\\
     Agim.DAT and Dritan.DAT 3PL.DAT eat.NACT.3SG/eat.NACT.3PL a apple.\\
     `Agim and Dritan feel like eating an apple.'
 \label{plural}    
     
  Despite having different subjects, both examples $\ref{sing}$ and $\ref{plural}$ have third person singular agreement on the verb.  Furthermore, in example $\ref{plural}$, where third person plural is expected, this form is ungrammatical.  Hence, the plural subject is not triggering plural agreement, unlike a canonical desiderative.
  
  Consequently, from both the semantic and syntactic perspective, impulsatives are distinct from volitional desideratives  or `want' type verbs.  Therefore, I argue that impulsatives are a grammatical category of their merit.     
     

  
  
  
     
%     The null impulse head is not a modal. Modals in Albanian do not produce presuppositional ambiguities with the adverb �again�.
%     
%     \exg. Agim Mund t\"e marr\"e k\"et\"e lib\"er s\"erish.\\
%    Agim.NOM may.3SG COMP take.3SG.SUBJ this book.INDEF again \\
%     `Agim may take this book again.'
%     \label{mund}
%     
%     Example $\ref{mund}$ only generates on presupposition; that Agim had taken the book before, but he may take it again.  This sentence cannot be used in a context where  Yesterday, Agim went to the library and thought about taking out the book, but didn't.  Today, he is at the library again, and may take it out (again).  In other words,  it does presuppose that he had the possibility of taking the book before but didn't, and now may take it again.  This is because modals in Albanian do not introduce eventualities.  This contrasts with impulsatives.  When impulsatives occur with adverbs such as `again' generate two possible presuppositions.
%   
%   \exg. Agimit k\"ercehet s\"erish. \\
%   Agim.DAT dance.NACT again. \\
%   `Agim feels like dancing again.' 
%   \label{again}
%   
%   The example in $\ref{again}$ has two possible presuppositions.   The first scopes over the impulsative predicate, generating the presupposition that Agim felt like dancing before but had never danced before.  The second presupposition scopes over the dancing predicate generating the presupposition that Agim danced before although he may not have felt like it.  
%   
%   
%   In addition \citet*{Murasic:2006} claim another difference between modals and impulsatives is that modals cannot admit three adverbials, one frame adverbial \citep*{Parsons:1990} and two temporal adverbs, while impulsatives can.
%   

%

%      \exg.  *Gjat\"e luft\"s, \c{c}do dit\"e mund t\"e k\"ercej ne\"ser. \\
%       During war, every day may MOOD dance.SUBJ tomorrow \\
%   `During the war, every day I may have danced tomorrow.' \\
%\label{three modal}
%   %double check data
%   
%      
%   \exg.  Gjat\"e luft\"s, \c{c}do dit\"e m\"e k\"ercehej nes\"er. \\
%   During war, every day 1SG.DAT dance.3SG.NACT.IMPF tomorrow \\
%   `During the war, every day I felt like dancing tomorrow.' 
%\label{three imp}

%Example $\ref{three modal}$ cannot admit the addition of a frame adverbial to two temporal adverbs.  Contrastingly, example $\ref{three imp}$ can.  This contrasts demonstrates that impulsatives are syntactically more substantiative than modals.  
%   
   
   
        \subsection{Introducing an Experiencer}

   
   
The impulse head is responsible for introducing an argument and assigning that argument dative case.  The dative argument in impulsatives is the experiencer of the feeling event introduced by the impulse head. The impulse head introduces this argument so that it can assign it the experiencer theta role of the feeling event.  Furthermore, the impulse head must assign its argument case because the case on the argument is not a structural one.  If it were structural, it would be affected by a raising verb, as subject of a raising verb it would receive nominative case.  This prediction is not borne out.
   
  \exg.  M\"e fillohet t\"e k\"ercej. \\
   me.DAT begin.NACT COMP dance.INF \\
`I feel like beginning to dance.'
\label{raising}
%is begin a raising verb in Albanian?
%you should try this in 3rd person

In example $\ref{raising}$, the  argument {\it m\"e} is dative not nominative.  This suggests that the case is marked lexically as opposed to structurally.  Consequently, it must be the null element assigning the lexical case.  If the null element were modal, it would be unable to assign case, therefore the null element is not a modal.
   
  In addition it assigns it dative case.  Dative case is the case for experiencers in Albanian.
   
   \singlespace
   
   \exg. Kamerierit nuk i u durua m\"e vetja. \\
waiter.DAT NEG him.DAT NACT endure.3SG.PST more itself.NOM \\
`The waiter could not endure himself any longer.' \\
\citep*[p 69]{Massey:1991}
\label{dative}


\exg. Sot m\"e kujtohen fjal\"et/rrobat e An\"es dje. \\
Today, me.DAT recall.NACT.1SG words/clothes.NOM AGR Ann.DAT yesterday\\
`Today, I recall Anna's words/clothes of yesterday.'
\label{dative2}

   \doublespace
   
   
   In example $\ref{dative}$ and $\ref{dative2}$ the verbs {\it durua} `endure' and {\it kujtohen}, `recall', take a dative subject because that argument receives an experiencer theta role.   Therefore, the impulse head introduces an experiencer argument and assigns it inherent dative case.
      
       
     \section{What is the Syntactic Structure?}
     
     Thus far, I have established that there is a null impulse head.  However, I must develop an analysis for the syntactic of the impulsative construction as a whole.  First, I show that impulsatives in Albanian, unlike Maru\v{s}i\v{c} and \v{Z}aucer's \citeyearpar{Murasic:2006} analysis for Slovenian are not biclausal but are monoclausal.\footnote{It should be noted that \citet*{Murasic:2006} do present a tentative typology, suggesting that Albanian impulsatives select vP.}   Furthermore, selectional restrictions indicate that a monoclausal analysis would be best suited for impulsatives as mentioned in section $\ref{There must be a null element}$.  Specifically, I suggest that the impulse head selects for \citeyearpar{Ramchand:2011}'s Process v Projection.  Then, I propose a monoclausal analysis that account for the non-active morphology.  Finally, I show that this analysis makes the correct predictions regarding c-command relations mentioned in section $\ref{Rivero}$.
   
     
     \subsection{Impulsatives are Not Biclausal}
     
     
          
      Albanian impulsatives are not biclausal.  Albanian impulsatives only have one tense and aspect.  Furthermore, the tense and aspect modify the impulsative event and not the internal predicate.

\singlespace

	\exg. Nj\"e moll\"e m\"e hahet nj\"e moll\"e. \\
	a apple 1SG.DAT eat.3SG.NACT  \\
	`I feel like eating an apple.' \\
	\label{present}


	\exg. Nj\"e moll\"e m'u h\"eng.\\
	A apple 1SG.DAT eat.3SG.NACT.PST\\
	`I felt like eating an apple.' \\
	$*$`I feel like I ate an apple.'\\
	\label{aorist}
	
	\exg.Nj\"e moll\"e m\"e hahej\\
	A apple 1SG.DAT eat.3SG.NACT.PST.IMP.\\
	`I was feeling like eating an apple.'\\
	$*$`I feel like I was eating an apple.'
	\label{imperfect}
	


\doublespace

Examples $\ref{present}$ and $\ref{aorist}$ differ in tense, present and past respectively.  The tense modifies the impulsative event, as opposed to the eating event.  Example $\ref{aorist}$ cannot mean `I felt like I ate an apple.'  In fact, the sentence gives no indication whether an apple was eaten or not.  Similarly, aspect also modifies the impulsative event and not the internal predicate.  Example $\ref{imperfect}$ differs from example $\ref{aorist}$ in aspect and cannot mean `I feel like I was eating an apple.' 


Finally, impulsatives in Albanian cannot have complementizers.


\exg. M\"e vjen (q\"e) t\"e k\"erceja. \\
	me.DAT come.3SG COMP MOOD dance.1SG \\
	`I feel like dancing.'
	\label{comp}


	\exg. *Q\"e m\"e (q\"e) k\"ercehej\\
	COMP me.DAT dance.NACT.PR\\
	`I feel like dancing.'
	\label{no comp}

	
While the periphrastic example in $\ref{comp}$ can have a complementizer, the covert impulsative in $\ref{no comp}$ does not allow the presence of an impulsative.  Based on evidence from complementizers, tense, and aspect I conclude that Albanian impulsatives are not biclausal.

     \subsection{Selectional Restrictions}
     \label{selectional restrictions}

     Recall in section $\ref{sec:Morphological Make-Up A}$, impulsatives in Albanian are restricted to certain verbs.  Examples are repeated below.  
     
     \singlespace
     \exg.Ben-it i vdis-et. \\
Ben.DAT 3S.DAT die.3SG NACT \\
`Someone died on Ben.'\\
$*$`Ben feels like dying.' \\
\label{unaccusative1}

%\exg. Nj� mal m�u ngjit.\\
%a mountain.s.def.NOM me.DAT.cl�nonact climb.3.s.nonact\\
%`To me a mountain was climbed.� \\
%`$*$I felt like climbing a mountain.� \\


\exg. Benit iu thye nj\"e      vazo. \\
Ben.DAT DAT.cl-NACT.cl break.NACT.3SG.PST a vase.INDEF \\
`To Ben, a vase was broken'. \\
$*$`Ben felt like breaking a vase.' \\
\label{causatives}
%actually i should give the imperfect example too



\exg. Ben-it i nderto-hej nje shtepi. \\
Ben.DAT DAT.cl build-NACT.3SG.IMP. a house.NOM \\
`For Ben, a house was built.' \\
 $*$`Ben felt like building a house.' \\

     
     \doublespace
     
     
     This is not due to lack of semantic salience as suggested by some (\citet*{Rivero:2004} cites Kallulli and Arnaudova (p.c.)), as all of the above predicates can be used in the periphrastic impulsative form.




\exg. M\"e vinte t\"e vdes. \\
I.cl.DAT come.3SG.PST MOOD die.SUBJ. \\
`I feel like dying.'
\label{die a}

\exg. M\"e vjen t�\"e thyej nj\"e vazo.\\
I.cl.DAT come.3SG.PR MOOD break.SUBJ a vase \\
`I feel like breaking a vase.'
\label{break}

%\exg. M� vinte t\"e ngjitem nj\"e mal. \\
%I.cl.DAT come.3.s.pst MOOD climb.subj a mountain.acc \\
%`I felt like climbing a mountain.'
%\label{climb}

\exg. M\"e vjen t\"e nd\"ertoj sht\"epi.\\
1SG.DAT.cl come.3SG.PR MOOD build.SUBJ a house. \\
`I feel like building a house.' 
\label{build}

 
 
 
Examples $\ref{die a}$-$\ref{build}$ demonstrate that it possible to have an urge or impulse `die', `break a vase', or `build a house', despite how rarely these feelings may occur.  Thus, it is not an issue of pragmatics.  Instead it appears to be a syntactic constraint.  Even \citep*{Kallulli:2006} treated the class of verbs as process verbs.  However, instead of an indication of the non-active morphology decomposing the event structure, I propose that it is an indication of the impulse head selecting for the process phrase VP in Ramchand's \citeyearpar{Ramchand:2011} First Phase Syntax.

Under Ramchand's \citeyearpar{Ramchand:2011} framework, the verb phrase is made up of three sub-event projections. Ramchand's \citeyearpar{Ramchand:2011} structure is presented below.

\ex. \Tree[.{vP\uput{.01cm}[0](2,.1) {causation projection}} {NP\\ subj of `cause'} [.v' v [.{VP\uput{.01cm}[0](2,.1) {process projection}} {NP\\ subj of `process'} [.V' V [.{RP\uput{.01cm}[0](2,.1) {result projection}} {NP\\ subj of `result'} [.R' R XP ] ]  ] ] ] ]
\label{Ramchand0}

In the structure in $\ref{Ramchand0}$ above there are three sub-events each with their own projection.  The vP introduces the causation event.  The VP introduces the process event, and finally, the RP provides the result state.  

If the impulse head were to only select the VP which correlates to the process event, then it would correctly predict that only process verbs could form impulsatives.  Furthermore, this would prevent causative and unaccusative verbs from being able to form impulsative constructions.  However, for those speakers who do allow unaccusatives, then one could postulate that for those speakers, the impulse head can select for both Process VP and RP.  These selectional restrictions indicate that the impulse head has direct control over the verb class that it selects.  Therefore, a monoclausal analysis is necessary to account for these selectional restrictions.





 
     \subsection{A Monoclausal Analysis}
     
   
     In this section, I provide a monoclausal structure for Albanian impulsatives.  I account for the presence of non-active morphology by extending Embick's \citeyearpar{Embick:2004a} analysis of non-active morphology to impulsative constructions.   Finally, this analysis correctly predicts the c-command relationship between the dative argument and the nominative object.
     
     
     
      
 Embick proposes an account of the morphological syncretism of the non-active voice in Greek and Albanian in the Distributed Morphology framework \citep*{Halle:1993b}. Non-active voice morphology appears on passives, reflexives, and unaccusatives as shown in section  \ref{sec:Morphological Make-Up A} repeated below.   
 
 
 
 
 
 \exg.  Molla hahej.\\
apple.NOM eat.3SG.NACT.PST \\
`The apple was eaten.'
\label{passive}

  
\exg. Papritmas, u duk dielli. \\
  suddenly NACT.cl appear.3SG.NACT.PST sun.NOM \\
  `Suddenly, the sun appeared.'
  \label{unaccusative}
  
  \exg. Un\"{e} ushqehem.  \\
1SG.NOM feed.1SG.NACT  \\
  `I feed myself.'
  \label{reflexive}

 
 
 
 
 
 
  Embick suggests that non-active morphology is a reflection of unaccusative syntax.  This assumes that external arguments are not introduced by the verb itself but by a higher functional head, Voice \citep*{Kratzer:1996}.  Unaccusative syntax is any syntactic structure where the external argument is not projected.  At spell-out, whenever there is a syntactic structure without the projection of the external argument, the non-active morphology is inserted.   In the tree below, v' is projected but not the full vP or spec vP where the external argument would be placed.

\ex. \Tree [.XP [.v' v  [.VP $\sqrt{ROOT}$ ] ] ]

These are all cases in which the external argument is not projected in Albanian.  Passives are a clear case for unaccusative syntax because the external argument that occurs in active versions of the sentence is missing.     Unaccusative verbs are verbs where the subject has been shown to be an underlying object or internal object of the verb.  And lastly, Embick argues that reflexives also have unaccusative syntax in that the underlying object raises to the specifier of v, after the cliticization of an anaphoric external element.    

However, in Ramchand's \citeyearpar{Ramchand:2011} First Phase Syntax there are several projections that introduce external arguments.  Nevertheless, the analysis can still be carried over. In Ramchand's \citeyearpar{Ramchand:2011} framework, non-active morphology is inserted whenever either v or V fail to project their argument.  Thus an XP with a process V' as a complement would also represent unaccusative syntax.  



I propose that the null impulse head selects for V'.  Since the external argument  of the overt verb does not project, the non-active morphology is inserted.  

\ex. \Tree  [.ImpuP' {nonact impu}  [.V' dance  ] ] 

In addition, the null impulse head passes on to its experiencer argument the agenthood of the internal predicate.  This is shown in the semantic denotation I have proposed below.

\ex. \doublebr{Impulse} = $\lambda$P$_{<e,vt>}$$\lambda$x.$\lambda$e.$\lambda$w.$\forall$w'[w' is compatible with what x feels like in e in w] $\rightarrow$[$\exists$e' in w'.P(x)(e')] 
\label{denotation}

This structure captures the monoclausal nature of Albanian impulsatives in addition to providing a template for which non-active morphology in Albanian is inserted uniformly.  Moreover, it also explains why impulsatives cannot form with an unaccusative verb.

   
 Finally, this monoclausal structure correctly predicts the c-command relationship between the dative argument and the nominative object.  Under this analysis, a transitive impulsative would be unaccusative, and thus not be able to assign accusative case to an object.  Thus, the object must move to spec of TP to receive nominative case as shown in the tree below.
 
 \ex. \Tree [.TP NP$_i$ [.T' T [.ImpulseP' {NACT impulse}  [.V' eat {\it t$_i$}  ] ] ] ]

 
 
  In Albanian impulsatives, the nominative argument can bind the dative argument.
  
\exg.Peshkaqen\"et$_1$ i hahen nj\"eritjetrit$_1$.\\
Shark.pl.NOM CL.pl eat.NACT.3PL each.other.DAT\\
`The sharks feel like eating each other.'
\label{c-command}

In $\ref{c-command}$, the dative argument {\it nj\"eitjetrit} `each other' is c-commanded by the nominative argument {\it peshkaqen\"et } `the sharks'.  Thus this analysis can explain the correct c-command relationship. Furthermore, this analysis explains why the object receives nominative case.
  
  
  


Thus, the selection of V' grants us four benefits.  First, it prevents both causatives and unaccusatives from forming impulsatives.  Secondly, it accounts for the non-active morphology.  Thirdly, it derives a monoclausal structure.  Finally, it accounts for the correct c-command relations between the experiencer subject and the nominative logical object.



     \subsection{Conclusion}
     
     
     In this section, I have detailed the syntactic structure of Albanian impulsatives.  First, I showed the Albanian impulsatives are not biclausal.  Next, I demonstrated that selectional restrictions motivated a monoclausal analysis.  And finally, I provided a monoclausal analysis that incorporated both non-active morphology and selectional restrictions.  With the exception of the selectional restrictions, this analysis parallels the analysis for Bulgarian impulsatives.
     
     
     
     
     
\section{Full Derivation and Conclusion}



 I posit a null impulsative modal with the following denotation

\ex. \doublebr{Impulse} = $\lambda$P$_{<e,vt>}$$\lambda$x.$\lambda$e.$\lambda$w.$\forall$w'[w' is compatible with what x has an impulse to do in e in w ] $\rightarrow$[$\exists$e' in w'.P(x)(e')] 

 The null impulsative modal does the following things. First, It provides modality.  In addition, it introduces another event-- namely the `feeling like' event and an experiencer argument.  In addition, it adds this to the assertion of the sentence.  The following is a sample derivation.

\singlespace

\ex. \Tree [.TP [.NP {an apple.NOM}$_1$ ] [.ImpulseP Agim.DAT [.Impulse' {Impulse\\NACT} [.{Process V'} [.{Process V} eat ] [.NP {\it t$_1$} ] ] ] ] ] 


\ex. \a. \doublebr{V} =$\lambda$y.$\lambda$x.$\lambda$e. eat(e) \& Agt(e,x)  \& Thm(e,y)\\
\doublebr{NP} =  an apple\\
Functional Application\\
\b. \doublebr{V'} = $\lambda$x.$\lambda$e. eat(e) \& Thm(e, an apple) \& Agt(e,x) \\
 \doublebr{Impulse}  = $\lambda$P$_{<e,vt>}$$\lambda$x$\lambda$e.$\lambda$w.$\forall$w' [w' is compatible with what x has an impulse to do in e in w ] $\rightarrow$ [$\exists$e' in w'.P(x)(e')]\\
 Functional Application \\
\b. \doublebr{Impulse'} = $\lambda$x$\lambda$e.$\lambda$w.$\forall$w' [w' is compatible with what x has an impulse to do in e in w ] $\rightarrow$ [$\exists$e' in w'.eat(e')  \& Thm(e', an apple)\& Agent(e',x))] \\
 \doublebr{Agim} = Agim \\
 Function Application \\
\c. \doublebr{ImpulseP} = $\lambda$e.$\lambda$w.$\forall$w' [w' is compatible with what Agim has an impulse to do in e in w ] $\rightarrow$ [$\exists$e' in w' .eat(e') \& Thm(e', an apple) \& Agent(e',Agim))]\\


\doublespace

First, the predicate combines with its bar projection.  If the verb is a transitive, such as `eat', here is where it semantically combines with its object. Then, the Impulse head selects with V'. This triggers the insertion of non-active morphology because the process V has an unsaturated argument.  Next, it combines with its argument.  Here, it assigns it the experiencer theta role and assigns it dative case. In addition, the experiencer receives the agent theta role in the possible world.  Finally, the object moves to spec of TP to receive nominative case.



In conclusion, Albanian impulsatives are best analyzed by positing a null impulse head.  This null head assigns dative case to its argument and selects for process V'.  By selecting a category of process V, this limits the construction to unergative and transitive verbs that are activities.  By selecting a bar projection, this triggers the non-active morphology to be inserted following Embick's \citeyearpar{Embick:2004} Distributed Morphology analysis. This analysis explains the non-active voice, the source of the intensionality, the bi-eventivity and selectional restrictions of Albanian Impulsatives. It also unifies the analysis with the analyses of other impulsatives in other languages, such as Bulgarian, Finnish, and Quechua.




