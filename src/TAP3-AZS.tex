%\documentclass{udthesis}
%\usepackage{covington}
%\usepackage{fullpage}
%%\usepackage{xyling}
%\usepackage{qtree}
%\usepackage{amsfonts}
%\usepackage{marvosym}
%\usepackage{lingtrees}
%\usepackage{tipa}
%\usepackage{hyperref}
%\usepackage{linguex}
%\usepackage{setspace}
% \usepackage{natbib}
% \bibpunct{(}{)}{;}{a}{,}{,}
%
%
%\newcommand{\doublebr}[1]{[\hspace{-.02in}[{\bf#1}]\hspace{-.02in}]}
%\newcommand{\doublebrexpand}[1]{$\left[\hspace{-.06in}\left[#1\hspace{-.5in}\right]\hspace{-.06in}\right]$}
%
%
%
%
%\begin{document}




\title{Impulsatives \protect \linebreak The Syntax and Semantics of Involuntary Desire}
% To indicate a new line in the title, use \linebreak
% at the end of each line but the last. The title is converted to
% upper case automatically.
\author{MaryEllen Cathcart}
\type{dissertation} % e.g., thesis, dissertation or
% executive position paper
\degree{Doctor of Philosophy} % e.g., Master of ..., Doctor of Philosophy or
% Doctor of Education
\educationtrue %or \educationfalse
%To indicate whether or not the document is an executive
%position paper.
%Default is \educationfalse.
\majorfieldtrue  %or \majorfieldfalse
% To indicate whether or not a major is to be included.
% Default is \majorfieldfalse.
\majorfield{Linguistics} % e.g., Physics
% Include only if you used \majorfieldtrue.
\degreedate{Spring 2011} % e.g., Fall 1993
% Spring 1993
% Summer 1993
\maketitlepage % make title page


\begin{approvalpage} % begin approval page

%\prof{Benjamin Bruening, PhD}
%Do not use for dissertation or executive position paper.
%\prof{name, highest degree of your second thesis adviser}
%Do not use for dissertation or executive position paper.
\chair{Frederic Adams Ph.D.} {Chairperson of the Department of Linguistics and Cognitive Science}
%\auxchair{name, highest degree of AuxChair}{position (title) of AuxChair}(2)
\dean{George H. Watson Ph.D.}{Dean of College of Arts and Sciences}


\end{approvalpage} % end approval page


\begin{signedpage} % begin signature page (Dissertation only)

\profmember{Benjamin Bruening, Ph.D.}
\member{Satoshi Tomioka, Ph.D.}
\member{Gabriella Hermon, Ph.D.}
\member{Peter Cole, Ph.D.}
\member{Anne Vainikka, Ph.D.}


\end{signedpage} % end signature page







\begin{front} % begin front material


\figurespagefalse % Remove if you want a List of Figures.
                   \tablespagefalse  %will not produce a 
                  % List of Tables.

\maketocloflot
%\tableofcontents


\prefacesection{Acknowledgments}

I think it goes without saying that completing a dissertation is a monumental experience and that's why before everything else is the acknowledgments.   First and foremost, thanks goes to my committee members, who have each played a crucial role.  In particular, I would like to thank my advisor Benjamin Bruening, who always held the bar high.  He ensured that the the level of quality of my work was the best, that my arguments were solid and that my writing was clear.  In addition, he provided a tremendous amount of support.  He was always available to answer questions and he was always positive and encouraging.  Secondly, I'd like to thank Satoshi Tomioka, Gabriella Hermon and Peter Cole for all the insight and support over the years.  And special thanks goes to Anne Vainikka for all her dedication and help on the Finnish chapter.  

In linguistics, everything relies on data.  Research cannot be done without good data.  Therefore, I would like to extend my gratitude to all my informants. As mentioned, Anne Vainikka provided much of the Finnish data.  In addition, Hannu Reime was also a great help.  For Bulgarian,  I would like to thank Slaven Gospodinov and Todor Koev.  For Albanian, I would like to thank Eni and Fioralba Cakoni.   And for Cusco Quechua, I would like to thank Magda Yabar, Milagros Sua\~na Cusi and Lucia Limo Llanos.   In addition, I'd like to thank John Miller and MariaElena Mendoza for helping me while i was in Cusco, Peru.


My colleagues in the linguistics department also contributed a lot to my linguistic thinking.  The members of FYD (Finish your Dissertation) Solveig Bosse, Anne Peng and Nadya Pincus were always there to discuss, debate, explain and decipher any problem that came my way.  They helped to solve not only linguistic problems but provided motivation and camaraderie.   Special thanks goes to Gina Cook who fundamentally changed my work habits and taught me to problem solve and pay attention to details.  

Last but not least, I'd like to thank my husband, Gilberto Carbajal and my family. They have been there by my side throughout these years, through the ups and the downs.   Their constant support and love through thick and thin has pushed me through these years.   









% list of figures, and list of tables
\prefacesectiontoc{Abstract}



Many languages utilize a desiderative affix to express desire. However, some languages, particularly the Quechua languages, have desiderative affixes that express involuntary desire \citep*{Hermon:1985, Cole:1985, Jake:1978}.  Non-volitional desideratives differ systematically from volitional desideratives.  In addition to semantic differences, desiderative constructions in Quechua have syntactic differences such as   oblique case marking and lack of verbal agreement.  Therefore, it is necessary that they be clearly differentiated.  I propose the term { \bf Impulsative} for the non-volitional type.  This dissertation investigates the syntax and semantics of impulsatives cross-linguistically.  


Impulsatives in the Quechua languages share properties with other constructions in the literature, such as the Involuntary State Construction \citep*{Rivero:2009} present in Albanian\citep*{Kallulli:2006},  and the South Slavic languages \citep*{Murasic:2006, Rivero:2004, Franks:1995}, and causative desideratives in Finnish  \citep*{Pylkkanen:1999}. Like Quechua impulsatives, these constructions also have subjects that lack volition, have oblique case-marking on the subject and do not inflect agreement on the verb.   However, there is one striking difference.  These languages lack dedicated verbal morphology that indicate `feel like/be in the mood to V'. 

  There is a debate in the literature as to whether the impulsative meaning comes from a covert impulsative element \citep*{Murasic:2006} or from other syntactic properties such as being imperfect or non-active \citep*{Rivero:2009, Kallulli:1999b}.     I argue that these languages have a covert instantiation of the overt impulsative in Quechua based upon the event and argument structure of impulsatves in each language.    I provide a unified semantic denotation of the impulse head.  Furthermore, I provide a syntactic structure for impulsatives in each language.  
















\prefacesectiontoc{Abbreviations}

\noindent   ACC = Accusative \\
    AOR = Aorist \\ 
    AUX = Auxiliary  \\
    CAUS = Causative \\
    COMP = Complementizer \\
     DAT = Dative \\
	DEF = Definite \\
	DES = Desiderative \\
    EV = Evidential \\
    F = Feminine \\ 
    IMPF = Imperfect \\
      INDEF = Indefinite \\
      IMPU = Impulsative \\ 
M = Masculine \\
NEG = Negation \\
  NOM = Nominative \\
    N = Neuter \\
    NACT = Non-active \\ 
          OM = Object Marker \\
     PART = Partitive \\ 
     P = Preposition \\
     PL = Plural\\
     PR = Present \\
     PROG = Progressive \\
     SG = Singular \\
     SUBJ = Subjunctive
    



\end{front} % end front material


%\end{document}