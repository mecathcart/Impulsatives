\qtreecenterfalse

\chapter{Introduction}

\section{Two Types of Desideratives}

Every language has a way to express a desire that is not volitional.  This is often times described as an uncontrollable urge, a yearning, a craving or even something one is in the mood to do.  English uses a periphrastic expression `feel like', such as `I feel like sleeping', `I feel like watching TV',  or `I feel like dancing.'   Intuitively, English speakers know that there is a difference between these sentences and sentences with the volitional verb `want'.  While both express desire, the former expresses an involuntary desire while the latter expresses a willing desire.  Often, these two desires can compete.  For example, Chuck is on a diet and he `wants' to eat more healthily, however he may `feel like' eating junk food. Despite the semantic difference, these two types of desires have been traditionally grouped together under the category of desideratives.  

The linguistic category ``desiderative" is a well-recognized and well-documented one; the World Atlas of Linguistic Structures (WALS) \citep*{Haspelmath:2008}, for instance, lists 45 languages distributed across
the globe as having desiderative affixes.  The WALS lists affixes under the section on `want complements', assuming desiderative affixes are all volitonal. Nevertheless, I argue that  we need to distinguish two different categories \footnote{While canonical categories are parts of speech such as noun and verbs, linguistic categories also include anything from tense to causatives \citep*{Crystal:1985, Hopper:1992, Bybee:1985} } that have been labelled desideratives: volitional and non-volitional.  Many of the affixes listed are either volitional or their volitionality is not discussed.  

However, there is one notable exception.  It has been observed that desiderative affixes in the Quechua languages are involuntary in nature primarily by  \citet*{Jake:1978, Cole:1985, Hermon:1985} for Imbabura Quechua.  \citet*{Hermon:1985} additionally discusses involuntary desideratives in Huanca and Ancash Quechua.  In addition to the semantic difference, they observe that desideratives in the Quechua languages also have non-canonically case-marked subjects.  Involuntary desideratives in the Quechua languages differ both semantically and syntactically from the voluntary desideratives in other languages.  

Furthermore, I argue that there are a number of systematic differences between them that requires us to clearly differentiate them, including the oblique case marking mentioned for Imbabura Quechua.    I will call the category of involuntary desire an impulsative, from the noun { \it impulse}, which is defined by the Oxford English Dictionary as a ``Sudden or involuntary inclination or tendency to act,'' (definition 3c).  I will simply call desideratives that are not impulsatives, volitional desideratives. 



This dissertation will investigate the systematic differences between impulsatives and volitional desideratives. As mentioned, semantically, impulsatives are non-volitional, in the sense that the subject has no control over the desire.   Impulsatives are always translated with non-volitional meanings such as `feel like' or `have an urge to'.  The feeling is often described as a yearning, an urge or an impulse as in the following example from Cusco Quechua. \footnote{Research on Cusco Quechua in
this dissertation is supported by NSF Doctoral Dissertation Research Improvment Grant \#0518308.  }


\exg. Noqa-ta tusu-{\bf naya}-wa-n.\\
I-ACC dance-IMPU-1OM-3SG \\
`I feel like dancing.'\label{quechua}

In example $\ref{quechua}$, the suffix \emph{-naya} added to a verbal stem V gives the meaning `feel like/be in the mood to V'.  In addition, speakers say the most salient context for impulsatives are verbs that describe bodily functions, such as `pee', `vomit', `cough,' `yawn', and `sleep'. 

  \exg. Noqa-ta hanllari-naya-wan.\\
  I-ACC yawn-IMPU-1OM-3SG\\
  `I feel like yawning.'
  $\label{bodily function}$
  
 
\exg. Noqa-ta aqtu-naya-wan. \\
I-ACC sleep-IMPU-1OM-3SG\\
`I feel like vomiting.'

 It is semantically salient to have an urge or impulse to vomit, however, vomiting is something few people desire to do in the volitional sense.   Furthermore, it is possible for the desire and impulse to be two separate things.  Consider the following example.
     

\exg. Noqa-ta pu\~nu-naya-wa-n. \\
I-ACC sleep-IMPU-1OM-3SG\\
`I feel like sleeping.'
 \label{punu}
 
 
\doublespace


Example $\ref{punu}$ is salient in a context where I am tired, even though I may not want to sleep.  For instance, it may be New Year's Eve and I want to be awake at midnight but am very tired.  This sentence cannot be used when what I want is not what I am feeling.  The sentence cannot be used in a context where I have a busy day the next day and want to get a good night's rest, but cannot fall asleep. 


In contrast, typical volitional desiderative affixes can be translated as `want' or `will' (often being ambiguous between the two); the WALS has desiderative affixes as a subcase of the feature `` `want' complement subjects.''   Moreover, in many languages, desideratives can also mean `will' and have a future interpretation.  This is true of volitional desideratives in Chukchi \citep*{Skorik:1961}, for instance. 

\singlespace
Chukchi

\exg. \~neekkeqej toryky {\bf re}-winren-{\bf \~n}-yrkyn \\
little.girl you re-help-�-PRES.I \\
`The little girl wants to help you' \\
\label{chukchi1}

\exg. qutti tyletumgyt petle {\bf re}-jen-{\bf \~n}yt (re-jet-\~nyt). \\
other fellow-travellers soon will-arrive \\
`The other fellow travellers will soon arrive'
\label{chukchi2}

\doublespace

In example $\ref{chukchi1}$ the circumfix {\it re....\~n} provides the sentence with a volitional desiderative meaning and is translated as `want'.  However this circumfix can also provide a future reading, as in example $\ref{chukchi2}$, where it is translated as `will'.  Furthermore, volitional desideratives can be used in contexts where impulsatives can not, as in the following Japanese example.

\singlespace

Japanese
 
  \exg. watashi-ga  ne-{\bf ta}-i  \\
speaker.NOM sleep-DES-PR \\
`I want to sleep.'
\label{japanese1}

\doublespace

Example $\ref{japanese1}$ cannot be used in the same contexts as $\ref{punu}$.  It infelicitous to say $\ref{japanese1}$ when one is tired on New Year's Eve.  On the other hand, it is natural to say this sentence when one cannot fall asleep but wants to be rested for the next day.  Whereas example $\ref{punu}$ referred to the uncontrollable urge to sleep despite ones's desires, $\ref{japanese1}$ refers to one's desire to sleep despite one's ability to sleep. 


 
In addition to semantic differences, impulsatives also diverge from volitional desideratives syntactically.  The subject is marked with an oblique case, not the nominative that characterizes subjects generally, and it also fails to agree on the verb (this pattern of case and agreement probably reflects the lack of volition). 


\exg. Noqa-ta pu\~nu-naya-wa-n.\\
I-ACC sleep-IMPU-1OM-3SG \\
`I feel like sleeping.' 
\label{impulse}

%\exg. Noqa pu\~nuyta munani.\\
%I.NOM sleep-INF-ACC sleep-1SG\\
%`I want to sleep.' 
%\label{desiderative}

% The construction in $\ref{impulse}$ is distinct from a construction that has an overt �want� verb, which behaves like a typical desiderative construction, as in $\ref{desiderative}$.  

In example $\ref{impulse}$, the impulsative affix is {\it naya}.  The subject {\it noqa} `I' is marked with accusative case {\it ta} as opposed to nominative case, which subjects in Cusco Quechua ordinarily receive.  Furthermore, the verb receives third person singular subject agreement despite the fact that the logical subject is first person.  Desideratives in other languages, on the other hand, usually assign the typical subject case, as in the Japanese example $\ref{japanese1}$, and do not affect agreement, as shown in the following examples from Passamaquoddy.

\singlespace
Passamaquoddy

\exg. Msi=te keq {\bf 't}-olluk-{\bf hoti}-ni-{\bf ya} ewapoli-ko-k \\
all=Emph what 3-do-Plural-N-3P IC.wrong-be-IIConj \\
`They do everything that is wrong.' \\
 \citep*[Line 5]{Mitchell:1921a} \\
 \label{passa1}


\exg. Aqami=te=hc {\bf 't}-oli=koti=olluk-{\bf hot}i-ni-{\bf ya}. \\
more=Emph=Fut 3-thus=DES=do-Plural-N-3P \\
`They will want to do it even more.' \\
\citep*[Line 99]{Mitchell:1921}
\label{passa2}

\doublespace

\noindent In example $\ref{passa1}$, the verb {\it ollok} is inflected for the third person plural agreement, which is distributed across three morphemes: a prefix {\it 't}, a suffix {\it hoti} and another suffix {\it ya}.  When the verb {\it 'ollok} has the desiderative preverb a {\it koti} , as in example $\ref{passa2}$, it still agrees with the third person plural subject.



Cross-linguistically, it is common for languages to not assign nominative case to arguments of experiencer predicates \citep*{McCawley:1976a}.   \citet*{Dixon:2001} list various categories of predicates that can take non-canonically\footnote{Dixon et al use `non-canonical' to refer to any marking which is not typically used for subjects of basic active verbs. Their discussion includes nominative-accusative languages where subjects are generally marked with nominative with the exception of some of these types of predicates.} marked subjects, such as verbs with affected subjects,, verbs of possession, verbs expressing `happenings'  and verbs with secondary modal meanings.   They discuss various semantic parameters such as volitionality, stativity and modality. For example, volitionality plays a role in the marking of non-canonical subjects in South Asian languages \citep*{Masica:1976, Klaiman:1980} where non-canonically marked subjects lack volition.   Cusco Quechua Impulsatives pattern much like these experiencer predicates falling in the class of verbs with secondary modal meanings.  Accordingly, the non-canonical subject and verb marking in impulsatives reflects a lack of volition.  To reiterate, the construction in Cusco Quechua differs from volitional desideratives both semantically and syntactically.   Desideratives express a willful desire and exhibit canonical subject marking. Impulsatives, however, pattern like experiencer predicates in that they are non-volitional and do not exhibit canonical case marking. Given the robust cross-linguistic trend which distinguishes typical predicates and experiencer predicates, it is not surprising that similar differences arise between desideratives and impulsatives as well.





\section{Overt vs. Covert Impulsatives}


While the only attested case of an impulsative affix exists in the Quechua languages, many languages have constructions that are very similar to impulsatives in Quechua, but lack dedicated morphology.  The constructions in these languages have meanings akin to ``feel like/ be in the mood to V'.  Furthermore, the logical subject takes an oblique case, and the verb no longer agrees with the logical subject.  For example, in Albanian, the impulsative is marked by the use of dative case and the non-active verbal form.

\exg.  Agimit k\"ercehet n\"e zyr\"e. (Albanian)\\
Agim.DAT dance.3SG.NACT.PR in office.DEF \\
`Agim feels like dancing in the office.'  \footnote{ Can also be interpreted as `There was dancing in the office and it affected Agim (e.g. it was Agim's office.)'  This is the passive of the affected argument construction, which looks identical on the surface. }
\label{covert imp}

In example $\ref{covert imp}$, the impulsative in Albanian is formed by using the non-active form of the verb {\it k\"ercen} `dance' and by marking the experiencer argument {\it Agim} with dative case.   In addition, the experiencer argument must also be the subject of the internal verb.  It cannot be a distinct subject.  {\it Agim} must desire that {\it Agim} dance, not anyone else.   I will call impulsative constructions which lack a dedicated morpheme, {\it covert impulsatives}.  In addition, Albanian also has a periphrastic impulsative.  


\exg. Shpesh m\"e vinte t\"e k\"erceja n\"e zyr\"e. \\
often 1sg.DAT come.3SG.IMP MOOD dance.1SG.IMP.SUBJ in office \\
`Often I have felt like dancing in my office.�
\label{peri}


Example $\ref{peri}$ is analogous to example $\ref{covert imp}$ both semantically and syntactically.   The subject is marked with dative case and the verb is conjugated in third person.  Many grammatical categories, like the causative and the desiderative, have both morphological and periphrastic counterparts.   Albanian does not have a morphological volitional desiderative but does have a verb `want' which I will call a periphrastic volitional desiderative. Periphrastic volitional desideratives  in Albanian are distinct from both the covert and periphrastic impulsative.



\exg.(Un\"e) dua (q\"e) t\"e shkoje n\"e Shqip\"eri. \\
I.NOM want.PR.1SG (that) to go.SUBJ.3SG to Albania \\
`I want that he/she to go to Albania.'


This sentence differs from $\ref{covert imp}$ and $\ref{peri}$ in several ways.  First, the subject is volitional.  Secondly, the subject receives canonical nominative case marking and the verb agrees with the subject.  Furthermore, the `goer' can be distinct from the `wanter'.  


  While impulsatives in Albanian like $\ref{covert imp}$ and Quechua appear different morphologically, they share a number of elements.  Impulsatives can be defined as having the following properties. First, the verb is morphologically complex.  Quechua has a dedicated morpheme; while Albanian uses the non-active voice.  Secondly, the associated NP non-volitionally feels like doing V. The subject is an experiencer, and the content of the experience an impulse to do V.   The experiencer is usually marked with a non-subject case, and the verb does not agree with the experiencer.   In addition, the construction is mono-clausal in that there is only one set of tense and aspect.   Lastly, impulsatives also introduce modal semantics. In the modal world of impulse,  the experiencer is the external argument of V.   I will use the term, covert impulsative, to refer to constructions like those in Albanian and, overt impulsatives, to refer to constructions like those in Quechua. 
  
Covert impulsatives of this sort have been observed  in many languages including Albanian \citep*{Kallulli:2006}, Finnish \citep*{Pylkkanen:1999} and the South Slavic languages \citep*{Murasic:2006, Rivero:2004, Franks:1995}.  The literature is divided as to whether the source of the intensionality comes from a covert impulsative element \citep*{Murasic:2006}, or from other syntactic properties such as being imperfect or non-active \citep*{Rivero:2009, Kallulli:1999b}.   Furthermore, there has been no attempt to build a unified analysis for covert impulsatives across languages.  In addition, covert impulsatives have never been compared to languages that have overt impulsatives, such as Quechua \citep*{Hermon:1985}. 

This dissertation will look at impulsative constructions in Albanian, Bulgarian, Finnish and Cusco Quechua and provide a unified analysis by defining an impulsative head.  All data from these languages come from my own fieldwork and very gracious informants.   Before I can introduce my denotation for the impulsative head, I must acknowledge some background assumptions.



\section{Background Assumptions}

This thesis approaches the syntactic structure of impulsatives with semantic compositionality in mind.  At the end of each chapter, I will provide a semantic derivation for the impulsatives in that language.  Syntactic trees are generated by mechanisms from \citet*{Heim:1998} and the semantic values of each element.  Each element $\alpha$ is a syntactic object with a semantic value written as \doublebr{$\alpha$} and a semantic type T written as $\alpha_{T}$.  Semantic types are governed by the following rules in $\ref{types}$. 

\singlespace
\ex.Semantic Types\\
a. e, t, s, and v are the semantic types of individuals, truth-values, situations and events,
respectively. \\
b. If $\alpha$ and $\beta$ are semantic types, then $<$$\alpha$,$\beta$ $>$ is a semantic type.  \\
c. If $\alpha$ is a type than $<$s,$\alpha$ $>$\\
d. Nothing else is a semantic type. \\
\citep*[Chapter 2.3 Ex 5 and Chapter12.3 Ex 1 ]{Heim:1998}

\label{types}

\doublespace

Type theory in conjunction to the functional rule below, determine the semantic value of each syntactic node.

\singlespace
\ex. Function Application \\
If $\alpha$ is a non-branching node, \{ $\beta$, $\gamma$\} is the set of $\alpha$'s daughters, and \doublebr{$\beta$} is a functions whose domain contains \doublebr{$\gamma$} then  \doublebr{$\alpha$} = \doublebr{$\beta$}(\doublebr{$\gamma$}).\\
\citep*[Chapter 3.1 Ex 3]{Heim:1998}

\doublespace


Furthermore, I will assume event semantics as in \citep*{Kratzer:1996, Kratzer:2003, Pylkkanen:2002a}.  In the event semantics, verbs are understood as properties of events, and they may take an internal argument.  Following \citep*{Kratzer:1996}, we assume that external arguments are not arguments of the verb, but are introduced by a higher functional head, Voice.  Additionally, I use neo-davidsonian logical forms, assuming theta roles such as Agent, Experiencer and Theme. Thus the denotations of Voice and a ordinary verb, such as {\it hit}, are as follows.

\singlespace

\ex. \doublebr{Voice}= $\lambda$x.$\lambda$e. Agt(e,x)

 \ex. \doublebr{hit} = $\lambda$x.$\lambda$e. hit(e) \& Thm(e,x) 


 VP and Voice combine via Event Identification, as follows:
 
 \exg.  f$_{<s,t>}$ g$_{<e,st>}$ $\rightarrow$ h$_{<e,st>}$ \\
$\lambda$e.f(e) $\lambda$x.$\lambda$e.g(x)(e) {} $\lambda$x.$\lambda$e.g(x)(e) \& f(e) \\
 
 
 Sample Derivation


\ex. Agim hit Dritan. \\
	
\small

\Tree [ Agim$_e$ [.{Voice'$_{<e, vt>}$\uput{.4cm}[0](.1,.1){=$\lambda$x. $\lambda$e. hit(e) \& Agt(e,x) \& Thm(e,Dritan)}} {Voice$_{<e, vt>}$\\$\lambda$x.$\lambda$e. Agt(e,x)} [.{VP$_{<v,t>}$\uput{.05cm}[0](.1,.1){=$\lambda$e. hit(e) \& Agt(e,Agim) \& Thm(e,Dritan)}} {hit$_{<e, vt>}$ \\ $\lambda$x.$\lambda$e. hit(e) \& Thm(e,x)} Dritan$_e$ ] ] ].{VoiceP$_{<v,t>}$\uput{.4cm}[0](.1,.1){= $\lambda$e. hit(e) \& Agt(e,Agim) \& Thm(e,Dritan)}} \\

\doublespace

In addition, I will be assuming \citet*{Chomsky:1995, Chomsky:2000, Chomsky:2001}'s case and agreement framework.  In this framework, the heads T and {\it v} are probes that look for goals, which are arguments which to assign structural case via Agree. Subsequently, the arguments check agreement features on {v} and T.  I also assume that certain verbs have the ability to assign inherent case that is dependent on the theta role of the argument\citep*{Chomsky:1986}.






Lastly, since impulsatives are by nature intensional, therefore it is necessary to introduce possible worlds \citep*{Fintel:2002}.   I will follow \citep*{Cresswell:1990} in treating possible worlds as a variable.  Furthermore, I have applied \citet*{Hintikka:1969}'s conception of attitude verbs to that of the impulsative.  Therefore it is a function which maps the proposition to a set of possible worlds which are compatible with the impulse.  


\section{Analysis}

In this dissertation, I provide a unified account of impulsatives in Cusco Quechua, Bulgarian, Albanian and Finnish. I propose a semantic denotation for the impulsative head as shown below.

\ex. \doublebr{Impulse} = $\lambda$P$_{<e,vt>}$$\lambda$x.$\lambda$e.$\lambda$w.$\forall$w'[w' is compatible with what x has an impulse to do in e in w] $\rightarrow$[$\exists$e' in w'.P(x)(e')]\footnote{While it is true that English has several periphrastic ways to denote involuntary desire, none are quite exact.  English `feel like' generally is not used for bodily functions.  I use less common `have an impulse to' as the closest translation and something that can be used for both bodily functions and more deliberate events such as `reading a book.' } 
\label{Impulse}

The denotation reads: For all possible worlds w' such that w' is compatible with what x has an impulse to do in e in w then there exists an event e' such that e' is x doing the P in w'.   

Semantically, the null impulsative head will do several things.  First, it will provide intensionality by quantifying over possible worlds.  Secondly, it has an event argument. Finally, it introduces an experiencer argument and links it with the agent of the internal predicate in the modal world. Syntactically, the impulsative head will license and case-mark an experiencer argument and select for a voice projection. The overt impulsative head in Cusco Quechua will select for a saturated voice projection.  The covert impulsative head in Albanian, Bulgarian and Finnish will select for an unsaturated impulsative head.

\pagebreak

\ex. Overt Impulsative	\tab \tab \tab \tab \tab \tab 				 Covert Impulsative \\ 
\Tree [.ImpulseP {NP-CASE}  [.ImpulseP' {Impulse}  [.vP PRO [.v' v  [.VP [.V ] ] ] ] ] ] 
 \Tree  [.ImpulseP {NP-CASE}  [.ImpulseP' {Impulse}  [.v' v  [.VP [.V ] ] ] ] ] 

I begin my investigation with Cusco Quechua which has a dedicated morpheme to indicate the impulsative in Chapter 1.  I discuss both the event and argument structure of impulsatives in Cusco Quechua.  Next, in chapter 2,  I independently motivate a null impulsative head in Bulgarian by showing no other element can be responsible for the intensionality in impulsatives.   Furthermore, impulsatives in Bulgarian are bi-eventive.\footnote{The second event of the overt predicate is included in the possible world introduced by the impulsative head.  Truth-conditionally, impulsatives are not bi-eventive like causatives, but they are bi-eventive in that the event of feeling and the event of the overt predicate in the possible world are separate events.}  In addition to further motivating the positing of a null morpheme, the bi-eventive nature of impulsatives in Bulgarian parallels the event structure in Cusco Quechua impulsatives. Because of the similarities of event and argument structure I argue that Bulgarian impulsatives are the covert version of the impulsative morpheme in Cusco Quechua.   Next, I cover Albanian impulsatives in Chapter 4, which morphologically corresponds to the impulsative in Bulgarian with one difference; namely that impulsatives in Albanian are subject to selectional restrictions.  This adds to the mounting evidence that there is an impulsative head.  Lastly, I look into impulsatives in Finnish in Chapter 5, which superficially appear to be the most different. Nevertheless, I show that impulsatives in Finnish can be accounted for by the same impulsative head used in all the other languages.  Moreover, while impulsatives in Finnish appeared to be causative, I argue that the morphology involved in Finnish impulsatives has similar properties to the morphology in Albanian and Bulgarian impulsatives.   Thus, impulsatives in Cusco Quechua, Bulgarian, Albanian and Finnish can all be accounted for by the denotation in $\ref{Impulse}$.  This denotation provides an argument and event structure that is identical across impulsatives.  In addition, the impulsative head explains the intensional behavior of the seemingly non-intensional constructions in Bulgarian, Albanian and Finnish.   





































