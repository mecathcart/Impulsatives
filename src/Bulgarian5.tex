%\documentclass{article}
%\usepackage{covington}
%\usepackage{fullpage}
%%\usepackage{xyling}
%\usepackage{qtree}
%\usepackage{amsfonts}
%\usepackage{marvosym}

%\usepackage{hyperref}
%\usepackage{linguex}
%\usepackage{setspace}
% \usepackage{natbib}
% \bibpunct{(}{)}{;}{a}{,}{,}

%\newcommand{\doublebr}[1]{[\hspace{-.02in}[{\bf#1}]\hspace{-.02in}]}
%\newcommand{\doublebrexpand}[1]{$\left[\hspace{-.06in}\left[#1\hspace{-.5in}\right]\hspace{-.06in}\right]$}

%

%\author{}
%\title {Impulsatives in Bulgarian:\\ A Covert Version of a New Grammatical Category\footnote{Thanks goes to Ben Bruening, Satoshi Tomioka, Gaby Hermon and the members of Sysel.  Research on Cusco Quechua in this paper is supported by NSF Dissertation Grant 0518308. }}
%\begin{document}
%\pagenumbering{arabic}
%\maketitle

%\doublespace


\chapter{Bulgarian}



\section{Introduction}

Bulgarian is  spoken by over 9 million speakers primarily in Bulgaria but also throughout the Balkan peninsula \citep*{Ethnologue:2005}.  Bulgarian is a South Slavic language and part of the larger Indo-European language family.  Bulgarian has SVO word order and is morphologically fusional.  Nouns are inflected for gender and number.  Verbs are inflected for person, number tense and aspect.  The orthographic system of Bulgarian uses cyrillic characters.  For purposes of transparency, I am following \citet*{Rivero:2009}, using Latin characters with diacritics like other South Slavic languages. 








 In this chapter, I will discuss a construction in Bulgarian which I view as a covert instantiation of the impulsatives in Cusco Quechua.  The goal of this chapter is investigate the syntax and semantics of Bulgarian impulsatives.  Similar constructions also exist in other South Slavic languages.  Bulgarian impulsatives and their South Slavic counterparts present interesting questions about the mapping between syntax and semantics because the source of the impulsative meaning in these constructions is not obvious from the morphology and syntax.  
 


 In Bulgarian, impulsatives are composed of a dative argument, the non-active clitic {\it se} and the imperfective aspect.\footnote{All examples, unless otherwise stated are Bulgarian elicited from two speakers from Sofia, Bulgaria.}

\singlespace
\exg. Na Ivan mu se pi\v{s}e\v{s}e kniga.\\
P Ivan DAT.M.SG NACT write.3SG.IMPF book  \\
`Ivan felt like writing a book.'\\
`A book was being written for/on Ivan.'
\label{phenom}

\doublespace

In example $\ref{phenom}$, the argument {\it Ivan} is marked with the dative preposition {\it na} and the verb is preceded by the non-active clitic {\it se}.  In addition, $\ref{phenom}$ is ambiguous.  One interpretation of this sentence is where a book was being written and Ivan is somehow affected.  I will refer to this type of reading as the affected argument reading \citep*{Pylkkanen:2002a, Bosse:2008}.\footnote{English translations for affected readings are not exact, for simplicity I will use `on/for Ivan'.}  The second reading is that Ivan had the desire or impulse to write a book. This is the impulsative reading.  Impulsatives introduce modal semantics or intensionality because in the modal world of the impulsative, the experiencer is the external argument of the verb.  All potential impulsatives are ambiguous between the affected argument and the impulsative readings.  This chapter focuses on the impulsative readings.  I will only address the affected argument readings when relevant.





Impulsatives in Bulgarian raise many questions. The central question is the source of the intensionality and the impulsative reading.  Since there is an absence of dedicated morphology, it is unclear where the intensionality is coming from.  A corollary question is: what is the structure of these constructions?  How many events, arguments, functional heads like tense and aspect are in this construction?  Furthermore, what licenses the experiencer argument? Finally, what role does the non-active morphology play in this construction?  


	While South Slavic impulsatives have long been recognized \citep*{Benedicto:1995, Franks:1995, Dimitrova-Vulchanova:1999, Rivero:2003, Rivero:2004},  they have recently garnered attention from two opposing accounts. The two accounts differ in two aspects in particular: the source of the intensionality and the size of the structure. The first is a biclausal account with a null psych-predicate, put forth by  \citet*{Murasic:2006}, who call this construction the ``feel like construction.''   The second is   Rivero's (2009) monoclausal account involving an imperfective operator as the source of the intensionality.  \citet*{Rivero:2009} calls this construction the `involuntary state construction.''   However, neither account fully captures the nature of impulsatives in Bulgarian.  I propose a monoclausal analysis with a null impulsative element with the following semantic denotation.

\ex. \doublebr{Impulse} = $\lambda$P$_{<e,vt>}$$\lambda$x.$\lambda$e.$\lambda$w.$\forall$w'[w' is compatible with what x has an impulse to do in e in w ] $\rightarrow$[$\exists$e' in w'.P(x)(e')]

Semantically, the null impulse head will do several things.  First, it will provide intensionality by quantifying over possible worlds.  Secondly, it has an event argument. Finally, it introduces an experiencer argument and links it with the agent of the internal predicate in the modal world. Syntactically, the impulse head will license and case-mark an experiencer argument and select for an unsaturated voice projection.    The structure for example $\ref{phenom}$ is shown in the tree below.   

 \singlespace

\ex. \Tree [.TP {kniga$_1$} [.AspP {} [.ImpulseP {Na Ivan}  [.ImpulseP' { Impulse}  [.v' v  [.VP [.V pi\v{s}e ] [.N t$_1$ ] ] ] ] ] ] ]


\doublespace



The rest of this chapter is organized as follows.  The second section is dedicated to finding the source of the intensionality.  I present both Maru\v{s}i\v{c} and \v{Z}aucer (2006)'s account with a null source and  Rivero's (2009) analysis with an imperfective operator as the source of the intensionality.   I conclude in favor of  \citet*{Murasic:2006} that the source of the intensionality comes from a null element.  In the third section, I depart from Maru\v{s}i\v{c} and \v{Z}aucer (2006)'s use of a psych predicate and argue that this null element is not a desiderative/volitional verb but rather an impulsative, parallel to the construction in Cusco Quechua.   I then characterize the null impulse head as introducing an argument and an event.  In the fourth section, I further depart from \citet*{Murasic:2006}'s analysis and argue that the impulsative construction is not biclausal but monoclausal.  Then I  present a monoclausal structure that derives the non-active morphology by using Distributive Morphology \citep*{Halle:1993b}.  Finally, section five provides a full derivation of the analysis and concludes the chapter.


\section{The Source of the Intensionality}

This section explains the paradox that impulsatives present.  I discuss the morphological makeup of Bulgarian impulsatives and demonstrate why none of the overt components of the sentence directly or singularly contribute to the intensional meaning.  There are two ways to solve this issue.  \citet*{Murasic:2006} motivate a null verb with a clausal complement.  On the other hand, \citet*{Rivero:2009} argues that there is no need to posit a null element.  She argues that impulsatives are analogous to English Futurates \citep*{Copley:2002} and Spanish Imperfects \citep*{Cipria:2000} in that an imperfective operator provides intensionality.  Nevertheless, I argue that Maru\v{s}i\v{c} and \v{Z}aucer (2006)'s position must be upheld in that the intensionality of impulsatives in Bulgarian cannot be derived from the components of the construction and that there must be a null element contributing the intensionality and impulsative meaning.

\subsection{Morphological Makeup of Impulsatives}

 The mapping of the syntax to the semantics is problematic in impulsatives because the morphological and syntactic components do not singularly introduce modal semantics, while the impulsative construction as a whole does. The morphological makeup of impulsatives appears to have three obligatory elements: non-active morphology, a dative argument, and imperfective aspect.  All elements can be used in constructions distinct from that of the impulsative. Moreover,  all of these constructions are non-intensional.  However, if a sentence is missing any of the three components the impulsative reading cannot be achieved.   
  
  
 Impulsatives in Bulgarian create intensional contexts.  One test for intensionality is ability to be true with a non-referring term \citep*{Larson:2002}.  If there is no intensional context a sentence is false if a term is non-referring, i.e. doesn't exist.  However, if there is an intensional context, then the sentence is possible true.

\exg. Pro sreshtnah ednorog.\\
    1SG meet.PAST unicorn \\
    `I met a unicorn'
    \label{extensional}

 \exg. Na mene mi se pokanva ednorog. \\
 P me.DAT DAT SE invite.3SG unicorn\\
 `I felt like inviting a unicorn.'
 \label{intensional}
  
  
  Example $\ref{extensional}$ is false because unicorns do not exist.  However, $\ref{intensional}$ could be true, despite the fact that unicorns do not exist.  In intensional context one could posit a possible world where unicorns do exist.  Therefore, example $\ref{intensional}$ creates an intensional context.  This is surprising since none of the elements contributes this intensionality. 
  

The first component of Bulgarian impulsatives is the clitic {\it se}.  This is traditionally known as a reflexive clitic \citep*{Franks:2000} however it has other uses as well, including passive and unaccusative.  Because of its use with arguments that lack an external argument I will follow the use in the Albanian and Greek grammars of the term non-active morphology.



\exg.  Knigata se pisa.\\
book.the NACT write.AOR.3SG\\
`The book was written.'
\label{non-active}

\exg. Ivan se poyavi.\\
Ivan NACT appear.AOR.3SG\\
`Ivan appeared.'
  \label{unaccusative0}

\exg.  Ivan se mie.\\
Ivan NACT wash.3SG.PR\\
`Ivan washes himself.'
  \label{reflexive1}

In example $\ref{non-active}$, the non-active clitic {\it se} makes the sentence a passive.  Example $\ref{unaccusative0}$ is an example of an unaccusative verb in Bulgarian.  Example $\ref{reflexive1}$ is a verb that can be made reflexive by the addition of the non-active clitic.  None of these examples $\ref{non-active}$-$\ref{reflexive1}$ are intensional.  



Dative arguments in Bulgarian can also occur in multiple constructions.   In addition to occurring in ditransitive constructions, dative arguments can also appear as subjects of psych predicates and as affected arguments in applicative constructions \citep*{Pylkkanen:2002a}.


\exg. Petar dade knigata na Ivan.\\
Petar give.3SG.AOR book.the P Ivan\\
`Peter gave the book to Ivan.'
\label{ditransitive}

\exg. Na Ivan mu se privi\v{z}hdat tezi momi\v{c}eta. \\
P Ivan DAT.M.SG NACT imagine.3PL these girls.the \\
`Ivan has a vision of these girls.'
\label{dative4}

\exg.  Joana mu pi\v{s}e\v{s}e mnogo statii na Ivan.\\
Joana DAT.M.SG write.3SG.IMPF many articles P Ivan \\
`Joana was writing many articles on/for Ivan.'
\label{aff arg1}


  In example $\ref{ditransitive}$, the dative argument functions as the second argument in a ditransitive, while it serves as the subject of a psych predicate in example $\ref{dative4}$.  Finally, in example $\ref{aff arg1}$ the dative argument can take a number of affected roles in applicative constructions.  Affected roles include benefactive: `Joana dedicated many articles to Ivan'; malefactive: `Joana was stealing Ivan's ideas'; or by proxy: `Joana was writing the articles so that Ivan didn't have to.'  While $\ref{aff arg1}$ can be interpreted many ways, none of these interpretations are intensional.  Furthermore, neither $\ref{ditransitive}$ and $\ref{dative4}$ are intensional. 

The imperfective morphology in Bulgarian is fused with past tense morphology. This is often referred to as the imperfect form.  The imperfect form in Bulgarian can be habitual, progressive or iterative.

\exg. Joana pi\v{s}e\v{s}e (vseki den).\\
Joana write.3SG.IMPF every day\\
`Joana was writing (every day).'
\label{imperfective}

  In $\ref{imperfective}$, the imperfective morphology allows the sentence to have various imperfective readings.  It can be used as the past progressive in the context where Joana was writing when the phone rang.  It can also have habitual or iterative readings that are more salient when adverbial phrases such  as {\it vseki den} `every day' is added or understood.  These readings are not intensional, however there is one possible intensional reading of the imperfect in Bulgarian, this is the futurate reading \citep*{Rivero:2009}.
  
 
 \exg. Dnes (po plan) izbuxva\v{s}e sta\v{c}kata.\\
today (per plan) start.Imp.3Sg strike.the \\
`According to plans, the strike was breaking out today.'
\label{plan}


Example $\ref{plan}$ can have  the reading that a strike will break out, this is the futurate reading.  In this sense, the imperfect in Bulgarian can create intensional contexts, however I will show that the imperfect in Bulgarian impulsatives is ultimately unnecessarily and that the imperfect cannot be the source of intensionality in impulsatives.

Thus, dative arguments, non-active and imperfective morphology in non-impulsative contexts perform specific functions in Bulgarian.  In their primary roles, they do not introduce any intensionality.   Examples $\ref{non-active}$-$\ref{imperfective}$ do not have any intensionality.  Since these morphological components do not inherently supply intensionality, the source of the intensionality in impulsatives is a mystery.  While none of the morphological components alone can create an impulsative, the impulsative reading cannot be attained without these crucial ingredients. It appears that if any one of these components is absent, the impulsative reading will not be achieved.  Later, I will argue that while non-active morphology and dative arguments are indeed obligatory, imperfective morphology is not.


Non-active morphology is obligatory in impulsatives.  If a sentence is active, the impulsative reading is not available.

\singlespace

\exg.  Na Ivan {} mu pi\v{s}e\v{s}e mnogo statii.\\
P Ivan pro DAT.M.SG write.3SG.IMPF many articles \\
`He/She was writing many articles on/for Ivan.' 
\label{no nonact}

\doublespace

Example $\ref{no nonact}$  is an active sentence with a dative argument and imperfective aspect.  It does not have the impulsative meaning, but rather a distinct affected argument reading.

Additionally, dative arguments are required in impulsatives.  Without the dative argument, the sentence is interpreted as a passive.


\exg.  Knigata se pi\v{s}e\v{s}e.\\
book.the NACT write.AOR.3SG\\
`The book was being written.'
\label{no dative}


 Example $\ref{no dative}$ is a non-active sentence with imperfective aspect, but lacking a dative argument.   This sentence also does not receive an impulsative reading, but rather is interpreted like a passive. 

Lastly, there appears to be an aspectual restriction. In the past tense, impulsatives must be imperfect.  In the aorist tense, the impulsative reading disappears:
\singlespace

\exg.  Na Ivan mu se pisa pismoto. \\
P Ivan DAT.M.SG NACT write.AOR.3SG letter.the \\
`They/ people wrote the letter for Ivan.'\\
$*$`Ivan felt like writing the letter.'
\label{aorist1}

\doublespace


  Example $\ref{aorist1}$ is in the simple aorist tense, the other morphological past tense in Bulgarian.  Though there is both non-active morphology and a dative argument, there is no impulsative meaning.   Examples $\ref{no nonact}$-$\ref{aorist1}$ indicate that all three ingredients are necessary to create the impulsative construction.

While the impulsative meaning cannot be attributed to any of its parts, the impulsative construction is contingent on their presence.  The apparent lack of compositionality drives at the central questions concerning impulsative constructions.  As initially discussed, the source of the intensionality remains to be explained.  Additionally, the requirement of each morphological component in impulsatives, must also be explained.  






%!!
\subsection{An Account with a Null Verb}

 \citet*{Murasic:2006} propose a biclausal analysis of impulsatives in the South Slavic Languages that involves positing a phonologically null psychological predicate.  They compare impulsatives in Slovenian to psych predicates that are nearly synonymous and argue that the structures are the same.  


\singlespace
Slovenian

\exg. Gabru se ple\v{s}e.   \\
Gaber.DAT SE dance3P.SG \\
`Gaber feels like dancing.'  \\
\citep*[Ex 2]{Murasic:2006}\\
\label{slovenian impulsative}

\exg. Gabru se lu\v{s}ta plesati. \\
Gaber.DAT SE desire3P.SG dance.INF \\
`Gaber feels like dancing.' \\
\citep*[Ex 3]{Murasic:2006}
\label{slovenian periphrastic}

\doublespace

 Maru\v{s}i\v{c} and \v{Z}aucer argue that Slovenian impulsatives are biclausal because they are not subject to restrictions that monoclausal constructions are.  One, in particular, is temporal adverbial modification.  Biclausal sentences can have two conflicting time adverbs without bringing about a temporal clash.
\singlespace

\exg. V\v{c}eraj se mi ni \v{s}lo jutri domov. \\
yesterday SE DAT AUX.NEG.PST go tomorrow home \\
`Yesterday, I didn't feel like going home tomorrow.' \\
\citep*[Example 13]{Murasic:2006}
\label{bieventive0}

\doublespace

In example $\ref{bieventive0}$, {\it v\v{c}erai} `yesterday' modifies the `feel like' event while {\it domov} `tomorrow' modifies the `going home' event.   On the other hand, non-agreeing adverbs are impossible in ordinary monoclausal constructions \citep*{McCawley:1979}. 

\singlespace
\ex. $*$Tomorrow, Jim will play basketball in two weeks.\\
\citep*[Example 8]{Murasic:2006}
\label{monoclausal}

\doublespace

However, many  \citep*{Ross:1976, Partee:1974, McCawley:1979, Dowty:1979, Dikken:1996, Larson:1997} have noted that some apparently monoclausal constructions can support two conflicting time adverbs.  \singlespace
\ex. Tomorrow, Jim will want a new bike in two weeks.\\
\citep*[Example 7]{Murasic:2006}
\label{null have}

\doublespace

While it appears that example $\ref{null have}$ only has one predicate, this cannot be true.  If it did only have one predicate, it would behave like the monoclausal example $\ref{monoclausal}$, and not be able to support two conflicting adverbs. \citet*{Larson:1997} argue that this is one indication that intensional verbs like {\it want} always take complement clauses rather than a direct object. This makes the structure biclausal. Because of this and other facts, such as selectional restrictions, binding effects, and ellipsis, \citet*{Larson:1997} propose that intensional verbs actually take a null {\it have} when it appears that they are taking only a direct object.  It is therefore the case that only biclausal constructions allow conflicting time adverbs.  On the basis of this and related arguments such as conflicting manner adverbials and depictive secondary predicates, which are also limited to biclausal structures, Maru\v{s}i\v{c} and \v{Z}aucer conclude that Slovenian impulsatives have a biclausal structure and that they also involve a phonologically null verb.  The following is the structure they propose for Slovenian impulsatives.

\ex. \Tree [.vQP NP$_{Dat}$ [ non-active [.VP {} [ FeelLike [.RModP {} [.AspP {} [.vP PRO [.v' {} [.VP ] ] ] ] ] ] ] ] ]
\label{M and Z}

In the structure in $\ref{M and Z}$, the non-active vP, labelled ``vQP''\citep*{Boeckx:2003b}, selects for the `feel like' predicate, the source of intensionality and the impulsative reading.
It, in turn, selects a clausal complement with both mood and aspect and an internal subject filled by PRO. Note that it is almost a full clause, except for the lack of tense.


This tree is intended to account for impulsatives in Slovenian.   Mura\v{s}i\v{c} and \v{Z}aucer note that certain facts make Bulgarian different from Slovenian, notably with regards to an aspectual restriction.  Mura\v{s}i\v{c} and \v{Z}aucer provide a typology of differences among impulsatives in the South Slavic languages.  They claim that the differences stem from the category that the `feel like' predicate selects.  In Bulgarian, the `feel like' predicate selects for a deficient clause no bigger than vQP rather than R-ModP.  Without an aspect projection to determine aspect, the default option which in Slavic is imperfective \citep*{Oresnik:1994} is chosen.  Aside from this difference, Bulgarian and Slovenian impulsatives are identical, according to Mura\v{s}i\v{c} and \v{Z}aucer.


\subsection{Impulsatives without a Syntactic Head}



 In contrast, \citet*{Rivero:2009} argues for a monoclausal analysis with no covert verb.  Rivero derives the intensionality in impulsatives from an imperfective operator.  She argues that impulsatives in Bulgarian and Slovenian are analogous to English futurates and Spanish Imperfects.  English futurates and Spanish imperfects have a particular intensional reading brought about by the use of imperfective or progressive morphology.  These readings exist in addition to the conventional reading that the morphology provides.  While it is understood that in most contexts, present progressives in English mean that something is happening now, they can also have a futurate meaning.  Contrast the following sentences:
 
 \singlespace
 
 \ex. \a. The movie is playing.
 \label{now}
 \b. The movie is playing at 7.
 \label{future}
 
 \doublespace
 
 
 In example $\ref{now}$, the progressive is used to indicate that the movie has already started and is currently playing.  The time of the utterance is located within the progression of the movie playing event.  This contrasts with example $\ref{future}$, where the playing event cannot happen at the time of the utterance, because there is a later time adverbial {\it at 7}.  Instead, the progressive morphology indicates a planned future event, in which it is understood that the movie is going to start at 7.  This is the futurate reading of the English progressive. 
 
  Under Copley's (2002) analysis, English futurates are understood as plans.  These plans have directors.  The directors are usually the subjects of the clause and are often understood as both the `planners' of the planning event and external argument of the internal predicate.  The planner could also be a third party such as the manager of the movie theatre in $\ref{future}$.  Despite being monoclausal, English futurates \citep*{Copley:2002a} and Spanish imperfects \citep*{Cipria:2000} can support two conflicting time adverbs without leading to a contradiction.


\singlespace
\ex. For two weeks, the Red Sox were playing the Yankees tomorrow. \\
\citep*[Example 1a]{Copley:2002a}\\
\label{imperfect0}

\exg. Durante dos semanas, el equipo jugaba ma\~{n}ana. \\
for two weeks the team play.IMPF.3SG tomorrow \\
`For two weeks, the team was playing tomorrow.' \\
\citep*[Example 5]{Rivero:2009}

\doublespace

In example $\ref{imperfect0}$, `for two weeks' modifies the plan to play and the day of the actual playing is to occur `tomorrow'.  Rivero argues that it is not necessary that impulsatives be biclausal, since English futurates are apparently monoclausal and can support conflicting adverbs.

Moreover, Rivero claims that an imperfective operator is the source of the intensionality in impulsatives in South Slavic, analogous to English futurates.   However, in impulsatives the ``subject" is a dative argument rather than a nominative argument as in English futurates.  Rivero argues that oblique or �quirky� subjects with dative case cannot function as directors, unlike nominatives in English futurates.  Under Rivero's analysis, the dative argument is introduced by a high applicative head \citep*{Pylkkanen:2002a}.   The high applicative phrase is above the TP, which makes it a topic position.  Rivero further argues that an argument in the topic position is in the contextual background and therefore cannot be a director.  Hence, the dative argument would not be understood as a `planner' the way the nominative argument is in English futurates.\footnote{This seems to conflict with Copley's analysis, since directors can be a third party as in $\ref{future}$.  The manager of the movie theatre can only be understood from the contextual background.}  Instead, the dative argument would identify the person with the relevant desire for the main predicate.  Her structure is provided below.


  \ex. \Tree [.ApplP NP$_{DAT}$ [.Appl' Appl [.TP Tense [.AspP IMP$^{OP}$ [.vP v VP ] ] ] ] ]
  
  
  

By deriving the intensionality from the Imperfective Operator (IMP$^{OP}$) in the head of Aspect, Rivero is able to provide a monoclausal structure for South Slavic impulsatives.  By placing the dative argument in the specifier of a high applicative she is able to differentiate between the futurate and impulsative constructions and explain why they receive different interpretations.  In addition, the high applicative head licenses the dative argument.  By these means, Rivero is able to achieve an impulsative construction without having to posit a null source for the intensionality.



\subsection{IMP$^{OP}$ is Not the Source of Intensionality}

In this section, I will argue that an imperfect operator is not the source of intensionality in Bulgarian Impulsatives.  \citet*{Marusic:2010} argue that the parallel between the futurate and the impulsative is problematic.  I provide additional evidence showing that the IMP$^{OP}$ is neither necessary or sufficient to account for the intensionality in impulsatives.  Finally, an argument from a cross-linguistic standpoint, I demonstrate that IMP$^{OP}$ cannot be the source of intensionality in either Slovenian or Albanian.

The first argument that \citet*{Marusic:2010} put forth against Rivero (2009)'s analysis. is to say that oblique arguments do not block futurate readings or necessarily create an impulsative reading.  Rivero's analysis  creates a dichotomy between futurate and impulsative readings.  Futurates are sentences with the imperfective operator and a nominative object which serves as the director of the intended plan.  On the other hand, impulsatives are sentences with imperfective operator and oblique argument.  She argues that the oblique argument  can not be the director of the intended plan because oblique arguments are incompatible with control.   This dichotomy is tenuous firstly because futurates can have directors that are not supplied by the nominative argument.    Additionally, \citep*{Marusic:2010} point out that imperfective sentences with dative subjects do not necessarily yield impulsative readings.

\singlespace
\exg. ? Danes je Petru jutri mraz. (Slovenian) \\
today aux Peter.DAT tomorrow cold \\
`Today it seems that Peter will be cold tomorrow.'\\
(Impossible: 'Today Peter feels like being cold tomorrow.') \\
\label{dat fut}

\exg. ? V\v{c}eraj je bilo Petru jutri \v{s}e mraz. (Slovenian) \\
yesterday aux been Peter.DAT tomorrow still cold \\
`Yesterday it seemed that Peter would still be cold tomorrow.' \\
(Impossible: 'Yesterday Peter felt like still being cold tomorrow.') \\
 \citep[Examples 6 and 7]{Marusic:2010}
\label{dat fut1}


\doublespace

Examples $\ref{dat fut}$ and $\ref{dat fut1}$ are marginally acceptable as futurates but cannot have impulsative readings.  Rivero's analysis predicts not only should the futurate reading be completely blocked but she also predicts that the impulsative reading should available.  Neither of these predictions is correct.  This is also true of Bulgarian.



\exg. Na Ivan mu (se) xaresvat tezi momicheta.  \\
P Ivan DAT.M.SG (NACT) like.3PL these girls.the \\ 
`Ivan likes these girls.' 
\label{dat fut2}

\exg. Na Ivan mu se privizhdat tezi momicheta. \\
P Ivan DAT.M.SG NACT imagine.3PL these girls.the \\
`Ivan has a vision of these girls.'
\label{dat fut3}

Despite having a dative argument and present tense, which Rivero claims involves the IMP$^{OP}$,  examples $\ref{dat fut2}$ and $\ref{dat fut3}$ do not have impulsative readings.  Both involve stative predicates with experiencer dative arguments, but the readings are distinct for that of the impulsative.  

Furthermore, \citet*{Marusic:2010} point out that impulsatives can also receive futurate readings.  

\singlespace

\exg.V\v{c}eraj se mi danes \v{s}e ni \v{s}lo v hribe. (Slovenian) \\
yesterday refl I.dat today still not go to mountains \\
`Yesterday, it did not seem that I would be in the mood today for going to
the mountains.'  \\
 \citep*[Example 8]{Marusic:2010}
\label{fut imp}

\exg. Vchera      na Ivan mu         se        vryshtashe       v  Sofia utre. (Bulgarian) \\
yesterday  to Ivan  he.DAT REFL go-back.PAST in Sofia tomorrow \\
`Yesterday, it did not seem that Ivan would feel like going back to Sofia tomorrow.' \\
\label{fut imp2}



\doublespace
Example $\ref{fut imp}$ and $\ref{fut imp2}$ are futurates of impulsatives.  Under Rivero's analysis this should be impossible since both readings stem from the same IMP$^{OP}$, it can only provide one sense of intensionality, it would be impossible for it to simultaneously supply both.  

Moreover, futurate readings are not uniformly available wherever impulsative readings are.  Under Rivero's analysis, the imperfective operator can yield either the futurate reading or the impulsative reading depending on the case of the noun.  If the subject is nominative, then the sentence can have a futurate reading.

\exg. Dnes (po plan) izbuxva\v{s}e sta\v{c}kata.\\
today (per plan) start.Imp.3Sg strike.the \\
`According to plans, the strike was breaking out today.'


  Rivero argues that the availability of the impulsative reading indicates that the imperfective operator is present. In addition to the imperfect past, Bulgarian also has a number of prefixes and suffixes that can change the aspect or telicity of an event.  In particular, Rivero claims that a verb with the imperfective suffix { \it -va} is an instantiation of the imperfective operator, because the impulsative reading is available even if the verb is in the aorist tense.   If this is the case, a futurate meaning should be available when the subject receives nominative case.   However, this prediction is not borne out.

\singlespace
\exg. Na Ivan mu se na-pis-va-xa mnogo statii. \\
P Ivan 3SG.DAT REFL PF-write-VA-AOR.3Pl many articles \\
`Ivan felt like writing up many articles.'\\ 
 \citep*[Ex 44]{Rivero:2009} \\
 \label{va impulsative}


\exg. *Dnes po plan Ivan napisva mnogo statii.\\
Today per plan, Ivan PF-write.VA.AOR.3SG many articles\\
`According to plan, Ivan is writing many articles.'
\label{va futurate}

\doublespace

While in example $\ref{va impulsative}$ the impulsative reading is available with the imperfective affix { \it -va},  example $\ref{va futurate}$ does not have the intended futurate reading with the same suffix. This suggests that the imperfective suffix {\it -va} isn't an instantiation of the imperfective operator that provides intensionality in futurates\footnote{That is not to say that the suffix is not imperfective}.  Without IMP$_{OP}$  in this construction, the availability of the impulsative reading is unexplained. 
 
 Another indication that the imperfective operator is not involved in Bulgarian impulsatives comes from the periphrastic perfect. The periphrastic perfect entails a telic event \citep*{Anagnostopoulou:1998}, as illustrated in $\ref{perfect0}$.

\exg. Na Maria i se e pisalo.\\
P Maria DAT.F.SG NACT has write.3SG.AOR.EV.N \\
`Maria had felt like writing.'
\label{perfect0}

In example $\ref{perfect0}$, we see that the impulsative reading exists, not only without imperfective morphology but with both the aorist tense and perfect morphology.  Hence, an imperfective operator is not present in the sentence.  This demonstrates that it is not necessary to have an imperfective operator to achieve an intensional impulsative reading, contrary to Rivero's account.

Furthermore, when the perfect is combined with the imperfective, the impulsative reading disappears.

\singlespace

\exg. Na Maria i se e pi\v{s}elo.\\
P Maria DAT.F.SG NACT has write.3SG.EV.N\\
`Something has been being written for Maria.�\\
$*$`Maria had been feeling like writing.'
\label{perfect imperfect}

\doublespace

\noindent Example $\ref{perfect imperfect}$ has imperfective morphology and a dative argument.  Therefore, Rivero's analysis would predict this sentence to have an impulsative interpretation.  However, that meaning is unavailable, and it can only be interpreted as a passive with an affected argument.  Thus, the imperfective operator does not always induce the impulsative meaning when it is present. If the imperfective operator were providing the intensionality in impulsatives, example $\ref{perfect imperfect}$ should have an impulsative meaning.  Nevertheless, it does not.  Therefore, an imperfective operator is not sufficient to capture all the impulsatives in Bulgarian.

To summarize, the facts about the aspectual restriction on impulsatives in Bulgarian are not clear.  On the one hand, an impulsative reading cannot be attained when a verb is in the simple aorist past with no other aspectual prefixes, making it appear that imperfective aspect is necessary.  On the other hand, in the periphrastic perfect form, impulsative readings can only be achieved with aorist tense and disappear with the imperfect form.  Therefore, an imperfective operator cannot be used to supply the intensionality in impulsatives.  The restriction cannot be solely attributed to the Aspect head.  Unfortunately, I do not have an explanation for this contradictory pattern, but it is clear that an imperfective operator cannot characterize it.  



 Finally, cross-linguistically, the imperfective is not a crucial component in impulsatives.  Albanian impulsatives are very similar to Bulgarian impulsatives.  They both have dative arguments and non-active morphology.  However, Albanian impulsatives do not require impulsatives to be imperfective. \footnote{\citet*{Kallulli:2006b} has some examples where the imperfective is necessary to achieve the impulsative meaning, however my informants did not agree with these judgments.}


\singlespace

\exg. Nj\"e moll\"e m'u h\"eng.\\
     A apple.INDEF.NOM DAT.1SG eat.NACT.AOR.3SG\\
     `I felt like eating an apple.' \\
     \label{alb aorist}


\exg.Nj\"e moll\"e m\"e hahej\\
     A apple.INDEF.NOM DAT.1SG eat.NACT.3SG.IMPF.\\
     `I was feeling like eating an apple.'
     \label{alb imperfect}
     
     \doublespace
     
   
Examples $\ref{alb aorist}$ and $\ref{alb imperfect}$ show that in Albanian, impulsatives can appear with both the imperfective and perfective past tenses.  This indicates that, for Albanian, the imperfective is not obligatory in impulsatives.  Even if Bulgarian impulsative relied on an imperfective operator for their intensionality, the intensionality in Albanian would still be unexplained.

In addition, Slovenian impulsatives can also appear in the perfect tense \citep*{Marusic:2010}.  

\exg. Jutr odpotujem v Potsdam. (Slovenian) \\
tomorrow departPF to Potsdam \\
'I leave for Potsdam tomorrow.' 


I conclude that the imperfective operator is not the source of intensionality in Bulgarian impulsatives.  I have shown various instances where the impulsative reading can be obtained where the sentence does not have an imperfect operator that contributes intensionality.  Hence, an imperfective operator is not necessary in Bulgarian impulsatives. Furthermore, I showed an example where there is an imperfective operator, however no impulsative reading exists.  Therefore, an imperfective operator is not sufficient.  Finally, data from Albanian and Slovenian impulsatives indicate that the imperfective is not an obligatory component of impulsatives cross-linguistically, and therefore could not be the source of the intensionality for impulsatives in either language.





\subsection{There Must be a Null Element}

 Bulgarian impulsatives must contain a null element. I have shown that no overt component can be responsible for the intensionality of impulsatives.  I have also shown that Rivero's analysis, which attempted to derive intensionality from the imperfective operator, fell short of predicting when an impulsative reading is available.  Hence, it appears that there is nothing that can explain the intensionality or the experiencer argument in impulsatives.   
 
 This leaves two possibilities; recognizing a construction as a grammatical formative or a positing a null element.  Currently in the generativist framework, constructions are unnecessary.  Null elements, on the other hand,  are common and unavoidable.  There are well established null elements such as little {\it pro} \citep*{Chomsky:1981, Chomsky:1982, Rizzi:1986a}.  Maru\v{s}i\v{c} and \v{Z}aucer were the first to propose the existence of a null element in impulsatives on the basis of, among other things, conflicting adverbs.  Adverbs are event properties of type $<$s,$t>$ and combine with event properties via Predicate Modification. That is, they adjoin to nodes of type $<$s,t$>$ \citep*{Parsons:1990, Landman:2000}.  Consequently, in order for an adverb to modify the `feeling' event, there must be a node that introduces the `feeling' event.  I agree with Maru\v{s}i\v{c} and \v{Z}aucer that there is such a node in impulsatives, and that nothing overt in the sentence could possibly be the node that introduces such an event.  Therefore, the node must be null. 
 
  Furthermore, this null element must be intensional, introduce an event and an experiencer argument and assign the latter case.   Incidentally, these are the properties attributed to the overt Impulse head in Cusco Quechua.   The existence of an overt Impulse head in Cusco Quechua provides motivation to posit it covertly for languages like Bulgarian.  In the following section, I will show that the Bulgarian constructionis an impulsative.
 
 
 
 
 

 

 \section{What is the Null Element?}
 
 In this section, I discuss the nature of the null element.  First, I argue that it is different from a desiderative/volitional verb.  Instead, I demonstrate that the construction in Bulgarian is similar to the impulsatives in Cusco Quechua. Then I demonstrate that the impulse head introduces an experiencer argument and assigns it case an additional event.  By positing a null head these properties of impulsatives fall out straightforwardly.  

\subsection{The element is not a `want' type verb, but an Impulse head}


Maru\v{s}i\v{c} and \v{Z}aucer consider the null verb a desire/volitional predicate, class 3 under Belletti and Rizzi's \citeyearpar{Belletti:1988a} classification of psych predicates. However, as stated in the introduction, impulsatives are not volitional.  In the following I demonstrate why the Bulgarian construction is more like the Quechua impulsative than a volitional predicate.  


\singlespace
\exg. Na Ivan mu se pi\v{s}e\v{s}e kniga.\\
P Ivan DAT.M.SG NACT write.3SG.IMPF book  \\
`Ivan felt like writing a book.'
\label{phenom2}


\doublespace

 First, its subject is non-volitional.   Impulsatives are always translated with non-volitional meanings such as `feel like' or `have an urge to'. The feeling is often described as a yearning, an urge or an impulse.  In addition, speakers say the most salient context for impulsatives are verbs that describe bodily functions, such as `pee', `vomit', `cough,' `yawn', and `sleep'. 

\singlespace

\exg. Na Ivan mu se spi. \\
P Ivan DAT.M.SG NACT sleep.PR.3SG\\
`Ivan feels like sleeping.' \\
`Ivan is sleepy.'
 \label{involuntary}
 
 
\doublespace


Example $\ref{involuntary}$ is salient in a context where Ivan is tired even though he may not want to sleep.  For instance, it may be New Year's Eve and Ivan wants to be awake at midnight but is very tired.  However, this sentence cannot be used when what Ivan wants is not what he is feeling.  The sentence cannot be used in a context where Ivan has a busy day the next day and wants to get a good night's rest but cannot fall asleep.  

 
Subjects of volitional desiderative predicates, on the other hand, are volitional as in $\ref{iska}$.


\exg. Ivan iska da spi.\\
Ivan want.3SG COMP sleep.PR.3SG\\
`Ivan wants to sleep.'
\label{iska}






% Moreover, in many languages, desideratives can also mean `will' and have a future interpretation. 

% This is true of desideratives in Japanese, for instance. 

%Japanese
% 
%  \exg. watashi-ga  ne-{\bf ta}-i  \\
%speaker.NOM sleep-DES-NONPAST \\
%`I want to sleep.'
%\label{japanese1}

Example $\ref{iska}$ cannot be used in the same contexts as $\ref{involuntary}$.  It infelicitous to say $\ref{iska}$ when one is tired on New Year's eve.  On the other hand, it is natural to say this sentence when one cannot fall asleep but wants to be rested for the next day.  Whereas example $\ref{involuntary}$ referred to the uncontrollable urge to sleep despite ones's desires, $\ref{iska}$ refers to one's desire to sleep despite one's ability to sleep.  Consequently there is a semantic difference between impulsatives and desideratives, namely impulsatives are not volitional while desideratives necessarily are.




 
 
  A second difference between impulsatives and volitional desideratives is that subjects of impulsatives are marked with an oblique case, not the nominative that characterizes subjects generally.  In the impulsative in example $\ref{phenom2}$, the subject has dative case.  In contrast, in the desiderative example in $\ref{iska}$ the subject has nominative case.  
  
  Furthermore,  the subject in impulsatives also fails to agree with the verb.  The verb instead agrees with the logical object as in example $\ref{passives}$.
\singlespace

\exg. Na Ivan mu se pi\v{s}exa mnogo statii. \\
P Ivan DAT.M.SG NACT write.3PL many articles. \\
`Ivan feels like writing many articles.' \\
\label{passives}

\doublespace

\noindent In example $\ref{passives}$, the verb {\it pi\v{s}exa} `write' carries third person plural agreement, despite the fact that the dative subject has is singular.  The plural argument in the sentence is the object {\it mnogo stattii} `many articles'.

 In contrast, volitional desideratives do not affect agreement.  The verb agrees with the subject of the volitional desiderative.
 
 
 
 \exg. Ti ishkashe da me pokanyish.\\
 You.nom want to me invite.2sg\\
 `You want to invite me.'\\
 
 

%\singlespace
%Passamaquoddy

%\exg. Msi=te keq {\bf 't}-olluk-{\bf hoti}-ni-{\bf ya} ewapoli-ko-k \\
%all=Emph what 3-do-Plural-N-3P IC.wrong-be-IIConj \\
%`They do everything that is wrong.' \\
% \citep*[Line 5]{Mitchell:1921a} \\
% \label{passa1}

%
%\exg. Aqami=te=hc {\bf 't}-oli=koti=olluk-{\bf hot}i-ni-{\bf ya}. \\
%more=Emph=Fut 3-thus=DES=do-Plural-N-3P \\
%`They will want to do it even more.' \\
%\citep*[Line 99]{Mitchell:1921} 
%\label{passa2}

\doublespace

%\noindent In example $\ref{passa1}$, the verb {\it ollok} is inflected for the third person plural agreement which is distributed across three morphemes a prefix {\it 't}, a suffix {\it hoti} and another suffix {\it ya}.  When the verb {\it 'ollok} has the desiderative preverb a {\it koti} , as in example $\ref{passa2}$, it still agrees with the third person plural subject.


To summarize, the construction in Bulgarian differs from volitional desideratives both semantically and syntactically.  Semantically, Bulgarian impulsatives are not volitional and are frequent with verbs  associated with urges rather than volitional desire.  Syntactically, subjects in Bulgarian impulsatives have oblique case, while volitional desideratives assign nominative case.  Lastly, verbs in impulsatives do not agree with their subjects, while verbs in volitional desideratives do.

  
   \subsection{Properties of Impulsatives}
   
   
   In this section, I describe the event and argument structure of impulsatives in Bulgarian. First, impulsatives introduce an experiencer argument.  The dative argument in impulsatives is the experiencer of the feeling event introduced by the impulse head.  The impulse head introduces this argument so that it  can assign it the experiencer theta role of the feeling event.   In addition, it assigns it dative case.  Dative case is the case for experiencers in Bulgarian.


\exg. Na Ivan mu (se) xaresvat tezi momicheta.  \\
P Ivan DAT.M.SG (NACT) like.3PL these girls.the \\ 
`Ivan likes these girls.' 
\label{dative0}

\exg. Na Ivan mu se privizhdat tezi momicheta. \\
P Ivan DAT.M.SG NACT imagine.3PL these girls.the \\
`Ivan has a vision of these girls.'
\label{dative00}

In example $\ref{dative0}$ and $\ref{dative00}$ the verbs {\it xaresvat}, `like' and {\it privizhdat}, `have a vision', take a dative subject because that argument receives an experiencer theta role.  Furthermore, dative case is assigned inherently.  Evidence of this is base on raising data.

\exg. Na Ivan mu se zapochva da kiha. \\
P Ivan DAT DAT.M.SG NACT begin.3SG SUBJ sneeze.INF\\
`Ivan begins to feel like sneezing.'
\label{raising b}

In example $\ref{raising b}$, the dative case is retained on the subject {\it Ivan} despite the fact that it is now subject of the verb {\it zapochva} `begin'.  This indicates that the case assigned on the subjects of impulsatives is inherent rather than structural.By proposing that the impulse head introduces an external argument and assigns it dative case, we resolve the issue of case and theta role assignment.  

Secondly, Bulgarian impulsatives introduce an event.  Evidence of the presupposition generated by the adverb `again'.  Modals are not predicate over events, instead they have an event variable in the accessibility relation \citep*{Hacquard:2006}.  As a result, modals in Bulgarian do not generate a presupposition when they occur with the adverb `again', only the embedded predicate generates a presupposition.  Impulsatives in Bulgarian, on the other hand, do generate a presupposition with the adverb `again'.   


\exg. As bix rabotil otnovo.\\
I.NOM may.1SG work.EV.M again\\
`I might work again.'
\label{modal again}

The example in $\ref{modal again}$ only has one possible presupposition: that I have worked before. It does not presuppose that there was a possibility of working before (and I didn't work). This is because the modal {\it bix} `might' does not introduce an eventuality.  Only the verb {\it raboti}, `work', does.  This contrasts with the impulsative, which does create ambiguities with the adverb `again', as in example $\ref{feel again}$.  

\exg. Na Ivan mu se pi\v{s}e\v{s}e kniga otnovo.\\
P Ivan DAT.M.SG NACT write.3SG.IMPF book again  \\
`Ivan felt like writing a book again.'
\label{feel again}


The example in $\ref{feel again}$ can have two possible presuppositions.  The first scopes over the impulsative predicate, giving the presupposition that Ivan had the urge to write before, but may not have ever written.  The second presupposition scopes over the internal predicate `writing a book.'  Ivan may have just completed his first book begrudgingly.  However, now that it is done, he feels the urge `write a book again.'  This indicates that the impulse head is not a modal, because unlike modals, impulsatives can have more than one presupposition with the adverb `again'.  Thus, I conclude that the null impulsative element introduces an event.  In addition, impulsatives in Bulgarian introduce and assign case to an external argument.



\subsection{Summary}

  Bulgarian impulsatives are not prototypical volitional/desiderative predicates. In this section, I argue instead that this is the covert instantiation of the Quechua impulsative.    Furthermore, the null impulse head introduces an event and an external argument which it assigns case.  
  
  
\section{Are Impulsatives Biclausal or Monoclausal?}

Thus far, the analysis I have put forth for impulsatives in Bulgarian contains a null impulse head that introduces both an event and an argument.  What remains to be determined is the nature of the structure of impulsatives. Namely, I must weigh in on the debate whether impulsatives are monoclausal or biclausal.  While I have followed \citet*{Murasic:2006} in having a null predicate, I depart from their analysis with regards to biclausality. In particular, I demonstrate that Bulgarian impulsatives are not parallel to their periphrastic biclausal counterparts.   I derive the monoclausal structure by proposing that the impulsative predicate selects for an unsaturated Voice projection. The monoclausal account explains the differences  between periphrastic biclausal impulsatives and covert impulsatives in Bulgarian.   Additionally, this accounts for the non-active morphology that is a characteristic of Bulgarian impulsatives. 



\subsection{Impulsatives are Not Biclausal}

In this section,  I show that impulsatives in Bulgarian are not biclausal.  Maru\v{s}i\v{c} and \v{Z}aucer propose that the structures of Slovenian periphrastic impulsatives and covert impulsatives are parallel.  However the difference between the two constructions would be that the Slovenian covert impulsative involves a near-synonymous phonologically null verb. Bulgarian, like Slovenian has both covert impulsatives and overt periphrastic counterparts.    Examples $\ref{impu}$ and $\ref{periphrastic}$ are Bulgarian counterparts to the Slovenian examples  $\ref{slovenian impulsative}$ and $\ref{slovenian periphrastic}$ introduced in the prior section.

\exg. Na Ivan mu se pi\v{s}e\v{s}e kniga.\\
P Ivan DAT.M.SG NACT write.3SG.IMPF book  \\
`Ivan felt like writing a book.'
\label{impu}


\exg. Na Ivan mu se \v{s}te\v{s}e da pi\v{s}e kniga.\\
P Ivan DAT.M.SG NACT want.IMPF.3SG COMP write.3SG. book  \\
`Ivan felt like writing a book.'
\label{periphrastic}

  I will demonstrate that in Bulgarian, periphrastic impulsatives and covert impulsatives do not have parallel structures.\footnote{However, Slovenian impulsatives may be best analyzed as biclausal.  Mura\v{s}i\v{c} and \v{Z}aucer give an example of modal ambiguity that indicates that the structure has at least two modal projections.  This suggests that the structure for Slovenian impulsatives is biclausal.  My informants were unable to give me clear judgements on comparable examples in Bulgarian.}  Furthermore, some of the differences reveal that covert impulsatives have a smaller complement and others can be explained by the monoclausal account I provide in the following section.



  While periphrastic impulsatives in Bulgarian are fully biclausal, covert impulsatives in Bulgarian are not.   Periphrastic impulsatives have accusative objects, can have reflexive objects, overt subjects, and passivized lower clauses, whereas covert impulsatives cannot.  All of these facts indicate that rather than having the same structure, covert impulsatives instead have a different, smaller structure than periphrastic impulsatives.

 
The first difference between periphrastic impulsatives and covert impulsatives is that covert impulsatives cannot have an overt subject.  Bulgarian does not have control sentences where an overt subject is disallowed: biclausal constructions always allow overt embedded subjects.  


\exg. Na Ivan mu se \v{s}te\v{s}e { \bf toj} da gi preg\textschwa rne.\\
P Ivan DAT.M.SG NACT will.IMP he SUBJ them hug.3SG\\
`Ivan feels like hugging them.'
\label{periphrastic subject}

\exg. Na Ivan mu se {*\bf toj} preg\textschwa \v{s}ta.\\
P Ivan DAT.M.SG NACT he hug.IMPF.3SG\\
`Ivan feels like hugging them.'
\label{covert subject}

In example $\ref{periphrastic subject}$, the pronoun {\it toy} can be the subject of the embedded clause.  However, in example $\ref{covert subject}$, the addition of the pronoun makes the sentence ungrammatical. The ability to have overt subject suggests that the complement of periphrastic impulsatives has a TP projection.  Moreover, the inability to have an overt subject in the complement clause indicates that covert impulsatives have a complement smaller than TP.  

  Furthermore, periphrastic impulsatives must have a subjunctive marker as in $\ref{overt passive}$ while covert impulsatives in Bulgarian cannot.   
 
 \exg. Na Ivan (*da) mu se pi\v{s}e\v{s}e kniga.\\
P Ivan SUBJ DAT.M.3SG SUBJ NACT write.3SG.IMPF book  \\
`Ivan felt like writing a book.'
\label{impu3}

Example $\ref{impu3}$ is ungrammatical with the presence of the subjunctive marker.  The subjunctive marker in $\ref{periphrastic subject}$ appears after the complement subject.  This indicates that the subjunctive marker is the head of a mood phrase that sits below TP \citep*{Stjepanovic:2004, Tomic:2004}.  This is further evidence that covert impulsatives must have a complement smaller than a TP.

In addition, covert Bulgarian impulsatives cannot have a passive embedded clause, while periphrastic impulsatives can.  If covert impulsatives were bi-clausal, they would have two voice projections and the embedded clause should be able to have the passive voice. 

\exg. Na Ivan mu se \v{s}te\v{s}e da bude pregurnat.\\
P Ivan DAT.M.SG NACT will.IMP SUBJ be.3SG hug.PART\\
`Ivan wanted to be hugged.'
\label{overt passive}

\exg.*Na Ivan mu se da bude pregurnat.\\
P Ivan DAT.M.SG NACT SUBJ be.3SG hug.PART\\
`Ivan felt like being hugged.' 
\label{covert passive}

In example, $\ref{overt passive}$, the periphrastic impulsative allows the lower clause to be passivized, however the covert counterpart in $\ref{covert passive}$ is ungrammatical.  This suggests that there is only one voice projection. If there is only one voice projection, then the structure must be smaller than the biclausal structure Mura\v{s}i\v{c} and \v{Z}aucer propose.  This indicates that the structure is actually monoclausal.



There are two more differences between periphrastic and covert impulsatives both dealing with the thematic object that do not straightforwardly indicate a monoclausal structure.  However, I will return to these differences when I introduce my monoclausal analysis and show that they fall out of the analysis I provide.  First, unlike periphrastic impulsatives, covert impulsatives assign nominative case to their logical object.

\exg. Na Ivan mu se \v{s}te\v{s}e da *(gi) pi\v{s}e (*te). \\
P Ivan DAT.M.SG NACT will.IMP SUBJ ACC.3PL write.3SG (them.NOM)\\
`Ivan felt like writing them.'
\label{overt object}

\exg. Na Ivan mu se (*gi) pi\v{s}exa te. \\
P Ivan DAT.M.SG NACT (ACC.3PL) write.3PL them.NOM\\
`Ivan felt like writing them.'
\label{covert object}

In example $\ref{covert object}$, the periphrastic impulsative is ungrammatical with a nominative object, but it's covert counterpart is grammatical only with a nominative object and is ungrammatical with an accusative object.   


Moreover, covert impulsatives cannot have reflexive logical objects, while the periphrastic ones can.  This is unexpected under any analysis where the covert version is analogous to the periphrastic version.

\exg.  Na Ivan mu se \v{s}te\v{s}e da pregurne sebe si.\\
P Ivan DAT.M.SG NACT will.IMP SUBJ hug.3SG self CL\\
`Ivan wants to hug himself.'
\label{overt reflexive}

\exg.  *Na Ivan mu se pregusta sebe si.\\
P Ivan DAT.M.SG NACT hug.3SG.IMPF self CL\\
`Ivan felt like hugging himself.'
\label{covert reflexive}

In example, $\ref{overt reflexive}$, the embedded verb {\it pregurne} `hug' takes the reflexive argument {\it sebesi}.  However, in $\ref{covert reflexive}$ the selection of {\it sebesi} results in ungrammaticality.  If covert impulsatives were fully biclausal there should not be any restrictions on the type of object the embedded predicate can take.  In section 4.2, I will discuss how my analysis would rule example $\ref{covert reflexive}$ out.




To conclude, periphrastic and  covert impulsatives behave very differently.  The fact that a covert impulsative cannot embed a passive or have separate tense and aspect for the embedded verb indicates that covert impulsatives in Bulgarian are monoclausal with only one set of functional projections like Voice, Tense and Aspect.  Furthermore, periphrastic and covert impulsatives assign different cases to their logical objects: while periphrastic impulsatives assign accusative case, covert impulsatives must assign nominative case to their logical objects and ban reflexive objects. These differences demonstrates that Bulgarian covert impulsatives are not analogous in structure to their periphrastic counterparts and that their complement is indeed smaller. 
 
 
 
 
 
 

\subsection{A Monoclausal Analysis}

In this section, I provide a monoclausal structure for Bulgarian impulsatives.  I extend Embick's \citeyearpar{Embick:2004a} analysis of non-active morphology to impulsative constructions. This extension has many advantages. First, it accounts for the differences between covert impulsatives and periphrastic impulsatives in Bulgarian.  Secondly, it explains the non-active morphology that obligatorily occurs in impulsative constructions.  And lastly, this unifies the analysis with that of impulsatives in Albanian which also hosts non-active morphology.

 Embick proposes an account of the morphological syncretism of the non-active voice in Greek and Albanian in the Distributed Morphology framework \citep*{Halle:1993b}. Non-Active voice morphology appears on passives, reflexives and unaccusatives.   Embick suggests that non-active morphology is a reflection of unaccusative syntax.  This assumes that external arguments are not introduced by the verb itself but by a higher functional head, Voice. \citep*{Kratzer:1996}  Unaccusative syntax is any syntactic structure where the external argument is not projected.  At spell-out, whenever there is a syntactic structure without the projection of the external argument, the non-active morphology is inserted.   In the tree below, v' is projected but not the full vP or spec vP where the external argument would be placed.

\ex. \Tree [.XP [.v' v  [.VP $\sqrt{ROOT}$ ] ] ]


Similar to Albanian and Greek, non-Active voice morphology in Bulgarian also appears on passives, unaccusatives  and reflexives, as shown in section 2.1.  Relevant examples are repeated below.

\exg.  Knigata se pisa.\\
book.the NACT write.AOR.3SG\\
`The book was written.'
\label{passive}

\exg. Ivan se poyavi.\\
Ivan NACT appear.AOR.3SG\\
`Ivan appeared.'
  \label{unaccusative}

\exg.  Ivan se mie.\\
Ivan NACT wash.3SG.PR\\
`Ivan washes himself.'
  \label{reflexive}
  

These are all cases in which the external argument is not projected in Bulgarian.  Passives are a clear case for unaccusative syntax because the external argument that occurs in active versions of the sentence is missing.     Unaccusative verbs are verbs where the subject has been shown to be an underlying object or internal object of the verb.  And lastly, Embick argues that reflexives also have unaccusative syntax in that the underlying object raises to the specifier of v, after the cliticization of an anaphoric external element.    



I propose that the null impulse head selects for v'.  Since the external argument  of the overt verb does not project, the non-active morphology is inserted.  However, since v' is a projection of v, agenthood is introduced into the semantics.

\ex. \Tree  [.ImpuP' {nonact impu}  [.v' v  [.VP dance ] ] ] 

In addition, the null impulse head passes on to its experiencer argument the agenthood of the internal predicate.  This is shown in the semantic denotation I propose below.

\ex. \doublebr{Impulse} = $\lambda$P$_{<e,vt>}$$\lambda$x.$\lambda$e.$\lambda$w.$\forall$w'[w' is compatible with what x has an impulse to do in e in w] $\rightarrow$[$\exists$e' in w'.P(x)(e')]\label{denotation}

This structure captures the monoclausal nature of Bulgarian impulsatives in addition to providing a template for which non-active morphology in Bulgarian is inserted uniformly.  Moreover, this unifies the analysis of both Bulgarian non-active morphology and Bulgarian impulsatives with that of their counterparts in Albanian.  Albanian impulsatives like their Bulgarian counterparts, are composed of a verb with non-active morphology and an argument with dative case:

Albanian
\exg.  Agimit k\"ercehet n\"e zyr\"e.\\
Agim.DAT dance.3SG.NACT.PR in office.SG.DEF \\
`Agim feels like dancing in the office.'  \footnote{Can also be interpreted as `There was dancing in the office and it affected Agim (e.g. it was Agim's office.)' }
\label{impulsative}


In example $\ref{impulsative}$, the impulsative in Albanian is formed by using the non-active form of the verb {\it k\"ercehet} `dance' and by marking the experiencer argument {\it Agim} with dative case. These similarities indicate that a good analysis of Bulgarian impulsatives should also extend to impulsatives in Albanian.  The analysis I have provided thus far extends smoothly to Albanian impulsatives.

Recall in the previous section, covert impulsatives differed from periphrastic impulsatives in that they assigned nominative case to the logical object.  This falls out of this monoclausal account as impulsatives have a similar structure as a passive, which also assigns nominative case to the logical object.  This is because when the full vP does not project and thus never has the ability to assign accusative case.  Therefore, the object is not licensed {\it in situ} and moves to spec of TP where it receives nominative case \citep*{Chomsky:1995, Chomsky:2000,Chomsky:2001}.


Moreover, it also explains why impulsatives cannot have a reflexive object.  Under Embick's analysis, reflexives consist of an underlying object in subject position that binds an anaphoric clitic that was base-generated in specifier of v but cliticizes to the v head.  In an impulsative, the anaphoric clitic would not be able to be base-generated in specifier of v, since that level is never projected.  Furthermore, the internal argument of the embedded predicate cannot be input as the experiencer argument.  In the denotation in $\ref{denotation}$, only the argument introduced by v, can be the experiencer argument.  Thus, my analysis can rule out $\ref{covert reflexive}$.


Thus, the selection of v' grants us four benefits.  Firstly, it unifies the analysis of Bulgarian impulsatives with that of Albanian impulsatives.  Secondly, it accounts for the non-active morphology, thirdly it derives a monoclausal structure and finally it explains the assignment of nominative case on the logical object and the ban on reflexive objects.

One objection to this type of analysis is that X-Bar theory, as it is usually understood, stipulates that selection of a bar level category is not allowed. However, there is no principled reason for this stipulation.  Moreover, the semantic selection of a predicate with an unsaturated argument has been used in analyses for reciprocals \citet*{Bruening:2004a} and reflexives \citet*{ Labelle:2008} respectively.  It is independently necessary to allow the selection of an open predicate for analysis of reflexives and reciprocals.  This semantic selection of an open predicate translates syntactically to a bar level category. Therefore, I argue that traditional X-Bar theory should allow for this type of selection.


\subsection{Conclusion}

In this section, I analyzed the null impulsative predicate as selecting for an unsaturated voice projection.  This was motivated by the fact that Bulgarian impulsatives have non-active morphology.  In addition, this analysis parallels impulsatives in Albanian.   The monoclausal properties were uncovered when I compared Bulgarian impulsatives with their periphrastic analogue.  This argues against the biclausal analysis proposed by \citep*{Murasic:2006}, despite having borrowed their suggestion of a null predicate.  I have taken their suggestion and modified it to an analysis with a null impulsative predicate that selects for the unsaturated voice projection. 


\section{Full Derivation}





 I posit a null impulsative modal with the following denotation

\ex. \doublebr{Impulse} = $\lambda$P$_{<e,vt>}$$\lambda$x.$\lambda$e.$\lambda$w.$\forall$w'[w' is compatible with what x has an impulse to do in e in w ] $\rightarrow$[$\exists$e' in w'.P(x)(e')]


 The null impulse head does the following things.  It provides modality, it Introduces another event; namely the `feeling like' event and an experiencer argument.  In addition, it adds this to the assertion of the sentence.  The following is a sample derivation

\singlespace

\ex. \Tree [.ImpulseP Ivan [.Impulse' Impulse [.v' v [.VP ] ] ] ] 


\ex. \a. \doublebr{write} = $\lambda$e. write(e) \\
 \doublebr{v} = $\lambda$x$\lambda$e. Agt(e,x) \\
 Event Identification \\
 \b. \doublebr{v'} = $\lambda$x.$\lambda$e.write(e) \& Agt(e,x) \\ 
   \doublebr{Impulse} = $\lambda$P$_{<e,vt>}$$\lambda$x$\lambda$e.$\lambda$w.$\forall$w' [w' is compatible with what x has an impulse to do in e in w] $\rightarrow$ [$\exists$e' in w'.P(x)(e')]\\
  Function Application \\
\d. \doublebr{ImpulseP'} = $\lambda$x$\lambda$e.$\lambda$w.$\forall$w' [w' is compatible with what x has an impulse to do in e in w] $\rightarrow$ [$\exists$e' in w'.write(e') \& Agent(e',x))] \\
 \doublebr{Ivan} = Ivan \\
 Function Application \\
\e. \doublebr{ImpulseP} = $\lambda$e.$\lambda$w.$\forall$w' [w' is compatible with what Ivan has an impulse to do in e in w] $\rightarrow$ [$\exists$e' in w' .write(e') \& Agent(e',Ivan))]\\


\doublespace



\section{Conclusion}



In conclusion, Bulgarian impulsatives are best analyzed by positing a null impulse head.  This null head is the same as the category seen in Cusco Quechua, namely an Impulse head.  Furthermore, the differences between fully biclausal psych predicates in Bulgarian gives us evidence that Bulgarian impulsatives are monoclausal.  In fact, the pattern of non-active morphology suggests that the lower clause does not have a full vP projection and instead that the impulse head selects for an unsaturated voice projection.   Lastly, the impulse head introduces the experiencer argument.  This analysis explains the non-active voice, the source of the intensionality and the bi-eventivity of Bulgarian Impulsatives.  















%
%\bibliographystyle{plainnat}
%\bibliography{/Users/mec/Documents/Resources/bibtex/mec}

%

%\end{document}